\documentclass[5pt]{article}

\usepackage{sectsty}
\usepackage{graphicx}
\usepackage{lipsum} % for generating dummy text
\usepackage[margin=1in]{geometry}
\usepackage{setspace}
\usepackage{array}
\usepackage{cellspace}
\usepackage{tabularx}
\usepackage[table]{xcolor}
\usepackage{tabularray}
\usepackage{wrapfig}
\usepackage{flafter} 


\usepackage{hyperref}
\usepackage{scrextend}
\graphicspath{ {../assets} }


% Margins
\topmargin=-0.45in
\evensidemargin=0in
\oddsidemargin=0in
\textwidth=6.5in
\textheight=9.0in
\headsep=0.25in

\title{Analisi dei Requisiti}
\author{Jackpot Coding}
\renewcommand*\contentsname{Indice}
\renewcommand{\figurename}{Immagine}
\date{\today}

%longtblr
\DefTblrTemplate{contfoot-text}{default}{}
\DefTblrTemplate{conthead-text}{default}{}
\DefTblrTemplate{caption}{default}{}
\DefTblrTemplate{conthead}{default}{}
\DefTblrTemplate{capcont}{default}{}

%STARTOF THE DOCUMENT
\begin{document}

%-------------------------

% Reduce top margin only on the first page
\newgeometry{top=0.5in}

%UNIPD LOGO
    \vspace{8pt}
    \includegraphics[scale=0.2]{UNIPDFull.png}
%END UNIPD LOGO

\vspace{30pt}

%COURSE INFO
\begin{minipage}[t]{0.48\textwidth}
    %COURSE TITLE
        \begin{flushleft}
            Informatica\\
            \vspace{5pt}
            \textbf{\LARGE Ingegneria del Software}\\
            Anno Accademico: 2023/2024
        \end{flushleft}
    %END COURSE TITLE
\end{minipage}
%END COURSE INFO


\vspace{5px}


%BLACK LINE
\rule{\textwidth}{5pt}

%JACKPOT CODING INFO
\begin{minipage}[t]{0.50\textwidth}
    %LOGO JACKPOT CODING
    \begin{flushleft}
        \hspace{10pt}
        \includegraphics[scale=0.65]{jackpot-logo.png} 
    \end{flushleft}
\end{minipage}
\hspace{-60pt} % This adds horizontal space between the minipages
\begin{flushright}
    \begin{minipage}[t]{0.50\textwidth}
        %INFO JACKPOT CODING
        \begin{flushright}
            Gruppo: {\Large Jackpot Coding}\\
            Email: \href{mailto:jackpotcoding@gmail.com}{jackpotcoding@gmail.com}
        \end{flushright}
    \end{minipage}
\end{flushright}
%END JACKPOT CODING INFO

\vspace{24pt}

%TITLE
\begin{center}
    \textbf{\LARGE ANALISI DEI REQUISITI}
\end{center}
%END TITLE

\vspace{13pt}

\begin{flushleft}
    \begin{spacing}{1.5}
        DESTINATARI: Prof. T. Vardanega, Prof. R. Cardin\\
    \end{spacing}
\end{flushleft}

\begin{flushright}
    \begin{spacing}{1}
        USO: ESTERNO\\
        VERSIONE: 1.1.2\\
    \end{spacing}
\end{flushright}


% Restore original margins from the second page onwards
\restoregeometry

\pagebreak

\textbf{\Large Registro delle modifiche}
\begin{longtblr}
	{
		colspec={|Q[0.10\linewidth]|Q[0.10\linewidth]|Q[0.15\linewidth]|Q[0.15\linewidth]|Q[0.45\linewidth]|},
		rows={halign=l},
		column{1}={halign=c},
		column{3}={halign=c},
		column{4}={halign=c},
		row{1}={halign=c},
		row{odd} = {gray!20},
		row{1}={teal!50},
	}
    \hline
    \textbf{Versione} & \textbf{Data} & \textbf{Autore} & \textbf{Verificatore} & \textbf{Modifica} \\
    \hline
    % versione & data & autore & verificatore & descrizioneModifica \\
    % \hline
    v.1.1.2 & 10/04/2024 & R. Simionato & - & Correzione UC21 - Caricamento Struttura Database tramite file \\
    \hline
    v.1.1.1 & 04/04/2024 & G. Moretto & - & Rimozione Redattori e Verificatori come da feedback RTB, correzione UC10.2 \\
    \hline
    v.1.1.0 & 15/03/2024 & - & E. Gallo & Verifica documento \\
    \hline
    v.1.0.4 & 15/03/2024 & G. Moretto & E. Gallo & Aggiunto RQ4, RV3, RV4, RV5, RV6 \\
    \hline
    v.1.0.3 & 29/02/2024 & G. Moretto & E. Gallo & Sistemazione link verifica G glossario, "il sistema visualizza un errore" diventa "il sistema mostra un errore" \\
    \hline
    v.1.0.2 & 27/02/2024 & G. Moretto & E. Gallo & Rimosso indicazione Glossario su titoli, Sinonimi opzionali su campo e tabella, RV3 diventa RF39, rimossa generalizzazione utente \\
    \hline
    v.1.0.1 & 21/02/2024 & G. Moretto & E. Gallo & Spostato UC25-29 sotto UC25, aggiornato numerazione e link \\
    \hline
    v.1.0.0 & 14/02/2024 & R. Simionato & G. Moretto & Approvazione \\
    \hline
    v.0.1.3 & 14/02/2024 & E. Gallo & R. Simionato & Aggiornata estetica del registro delle modifiche. Aggiunti Elenco delle immagini e Elenco delle tabelle\\
    \hline
    v.0.1.2 & 12/02/2024 & G.Moretto & E. Gallo & Aggiunti Tabella, Campi e Sinonimi  al glossario, correzione terminologia casi d'uso\\
    \hline
    v.0.1.1 & 12/02/2024 & G.Moretto & E. Gallo & Aggiunti riferimenti al Glossario \\
    \hline
    v.0.1.0 & 10/02/2024 & R. Simionato & G.Moretto & Verifica documento, aggiunta corsivo per termini in Inglese \\
    \hline
    v.0.0.14 & 11/01/2024 & R. Simionato & G.Moretto & Aggiunti Requisiti e Tracciamento \\
    \hline
    v.0.0.13 & 9/01/2024 & E. Gallo & G. Moretto & Aggiunti diagrammi UML per i casi d'uso \\
    \hline
    v.0.0.12 & 9/01/2024 & R. Simionato & G. Moretto & Aggiunta struttura tabelle dei Requisiti \\
    \hline
    v.0.0.11 & 9/01/2024 & R. Simionato & E. Gallo & Aggiunte sezioni Glossario e Riferimenti \\
    \hline
    v.0.0.10 & 8/01/2024 & E. Gallo & R. Simionato & Aggiunti UC 11. Modificati UC 7, 15.3, 16, 22, 30.2 \\
    \hline
    v.0.0.9 & 8/01/2024 & G. Moretto & E. Gallo & Aggiunti UC 26,28,28.1,28.2,30,30.1,30.2,31,32,32.1,32.2,33 \\
    \hline
    v.0.0.8 & 7/01/2024 & G. Moretto & E. Gallo & Aggiunti UC 5,7,11,17,19,21,22. Modificati UC 15, 15.1, 15.2, 15.3, 16, 16.1, 16.2, 16.3,18 \\
    \hline
    v.0.0.7 & 7/01/2024 & R. Simionato & G. Moretto & Aggiunti UC 11,12,14,15,17 \\
    \hline
    v.0.0.6 & 7/01/2024 & R. Simionato & G. Moretto & Aggiunti UC 18,19 alla struttura, diviso documento in sottofile \\
    \hline
    v.0.0.5 & 7/01/2024 & R. Simionato & G. Moretto & Aggiunti UC 3,4,7,8,9 e sottocasi  \\
    \hline
    v.0.0.4 & 5/01/2024 & R. Simionato & G. Moretto & Aggiunta nuova struttura per i Casi d'Uso \\
    \hline
    v.0.0.3 & 29/12/2023 & G. Moretto & R. Simionato  & Aggiunti UC 6,7,8,9,9.1,9.2, commentati altri \\
    \hline
    v.0.0.2 & 18/12/2023 & R. Simionato & G. Moretto & Aggiunte Introduzione, Descrizione e Casi d'uso \\
    \hline
    v.0.0.1 & 15/11/2023 & R. Simionato & G. Moretto  & Creata struttura del documento \\
  	\hline
\end{longtblr}



\pagebreak
\tableofcontents
\pagebreak
	\begin{flushleft}
\textbf{\Large Elenco delle immagini} 

\begin{spacing}{1.5} 

Immagine 1: Attori coinvolti \newline
Immagine 2: UC1,UC2 - Login \\
Immagine 3: UC1 - Inserimento nome utente e password \newline
Immagine 4: UC3,UC4 - Creazione struttura database \\
Immagine 5: UC3.1,UC3.2 - Inserimento nome e descrizione database \\
	Immagine 6: UC5 - Visualizzazione lista strutture database \\
	Immagine 7: UC6,UC7,UC8 - Visualizzazione, modifica ed eliminazione singola struttura database \\
	Immagine 8: UC9,UC10 - Creazione tabella \\
	Immagine 9: UC9,UC10 - Inserimento campi tabella \\
	Immagine 10: UC11 - Visualizzazione lista delle tabelle della struttura database \\
	Immagine 11: UC12,UC13,UC14 - Visualizzazione, modifica ed eliminazione della tabella \\
	Immagine 12: UC15,UC16 - Creazione campo tabella \\
	Immagine 13: UC15,UC16 - Inserimento singolo campo tabella \\
	Immagine 14: UC17 - Visualizzazione lista dei campi della tabella \\
	Immagine 15: UC18,UC19,UC20 - Visualizzazione, modifica ed eliminazione singolo campo tabella \\
	Immagine 16: UC21,UC22 - Caricamento struttura database tramite file \\
	Immagine 17: UC23,UC24 - Logout \\
	Immagine 18: UC25 - Richiestya di generazione prompt \\
	Immagine 19: UC25.4 - Inserimento frase in linguaggio naturale \\
	Immagine 21: UC25.6 - Visualizzazione prompt generato \\
	Immagine 22: UC26 - Errore comunicazione con LLM \\
	Immagine 23: UC27 - Richiesta generazione query SQL \\
	Immagine 24: UC28 - Errore comunicazione con API 
\end{spacing}
\end{flushleft}


\pagebreak

\begin{flushleft}
\textbf{\Large Elenco delle tabelle}
\begin{spacing}{1.5}

	Tabella 1: Requisiti funzionali \\
	Tabella 2: Requisiti di qualità \\
	Tabella 3: Requisiti di vincolo \\
	Tabella 4: Fonte-Requisiti \\
	Tabella 5: Requisito-Fonti \\
	Tabella 6: Riepilogo
\end{spacing}
\end{flushleft}

\pagebreak

\section{Introduzione}
\subsection{Scopo del documento}
Questo documento serve a fornire una descrizione dettagliata del funzionamento del prodotto, prestando particolare attenzione a come il prodotto potrà essere usato e a quanto richiesto dai requisiti presentati e discussi con il proponente.
Analizzando questi cerchiamo quindi di individuare ed illustrare i diversi attori e i casi d’uso presenti all’interno del prodotto.\\
Ogni caso d’uso rappresenta uno scenario di utilizzo del programma da parte di un attore, per descriverlo al meglio utilizzeremo la struttura seguente:
\begin{itemize}
	\item Descrizione: breve descrizione del caso d'uso;
	\item Attori: chi esegue l'azione descritta;
	\item Precondizioni: stato del programma prima del caso d'uso;
	\item Postcondizioni: stato del programma dopo il caso d'uso;
	\item Scenario principale: azioni svolte prima, durante e dopo il caso d'uso;
	\item Generalizzazioni: se presenti, scomposizione del caso d'uso in sottocasi;
	\item Estensioni: se presenti, casi d'uso collegati(es. visualizzazione di errori o avvisi).
\end{itemize}
Ogni attore rappresenta una persona o un sistema esterno al programma che si interfaccia con esso.
Nel nostro caso il programma verrà utilizzato da due tipi di attori che avranno accesso a diverse funzionalità del prodotto:
\begin{itemize}
	\item \textbf{Utente}: può scegliere il database a cui fare la richiesta e inserire il messaggio in linguaggio naturale che verrà utilizzato per la creazione del prompt;
	\item \textbf{Amministratore}: può aggiungere, modificare e eliminare i database selezionabili dagli utenti.
\end{itemize} % __DA VERIFICARE__ decidere se lasciare la descrizione degli attori qui o se spostarla

\subsection{Glossario}
\subsection{Riferimenti}

\section{Descrizione}
\subsection{Obiettivi del prodotto}
Il prodotto ha come obiettivo dare la possibilità di interrogare un \textit{database} partendo da una richiesta in linguaggio naturale, trasformandola in un \textit{prompt} da sottoporre ad un sistema di \textit{AI} per ottenere una \textit{query SQL} corretta in base alla struttura del \textit{database} interrogato.

\subsection{Funzioni del prodotto}
Il prodotto dovrà quindi, dati:
\begin{itemize}
	\item un file strutturato contenente le tabelle e le relazioni di un \textit{database};
	\item una frase in linguaggio naturale per interrogare suddetto \textit{database}.
\end{itemize}
Trovare nella frase le parole chiave, capendo quali sono i dati da visualizzare, in quali tabelle sono salvate, tenendo conto di eventuali condizioni imposte.
In seguito dovrà riscrivere la frase in modo da generare un \textit{prompt} che possa essere passato ad un sistema \textit{AI} che creerà la \textit{query} richiesta in linguaggio \textit{SQL}.
Il \textit{prompt} generato potrà essere semplicemente mostrato all’utente, che sarà tenuto ad inserirlo nel sistema \textit{AI} da lui scelto, oppure utilizzare le \textit{API} delle \textit{AI} per sottoporre direttamente il \textit{prompt} e mostrare all’utente il codice \textit{SQL}.
Il prodotto dovrà inoltre dare la possibilità agli utenti amministratori, dopo aver effettuato l’accesso, di:
\begin{itemize}
	\item aggiungere file strutturati selezionabili dagli utenti;
	\item modificare i file strutturati;
	\item eliminare file strutturati non più necessari.
\end{itemize}

\section{Casi d'uso}
% TEMPLATE
%   \subsection{UC - NomeUseCase}
%   \label{sec:UC}
%   \includegraphics[]{diagramma_UML}
%   \begin{itemize}
	%       \item \textbf{Descrizione:} 
	%       \item \textbf{Attori:} 
	%       \item \textbf{Precondizioni:} 
	%       \item \textbf{Postcondizioni:} 
	%       \item \textbf{Scenario principale:} 
	%       \item \textbf{Generalizzazioni:} 
	%       \item \textbf{Estensioni:} 
	%   \end{itemize}

%   \hyperref[sec:UC]{\textbf{UC}}

% Link ad altre sezioni usando hyperref - utile per linkare generalizzazioni e estensioni
% La label fa da segnalibro a dove dovrà andare il link
%   \label{sec:nomeSezione}
% Link sul quale cliccare per andare alla label
%   \hyperref[sec:nomeSezione]{testo}
% END TEMPLATE
\setcounter{secnumdepth}{0}

\subsection{UC1 - Login}
\label{sec:UC1}
%\includegraphics[]{diagramma_UML}
\begin{itemize}
	\item \textbf{Descrizione:} L’amministratore accede al pannello amministrativo con le sue credenziali;
	\item \textbf{Attori:} amministratore;
	\item \textbf{Precondizioni:} 
	\begin{itemize}
		\item L’amministratore possiede delle credenziali di accesso valide;
		\item L’amministratore non ha già effettuato l’accesso;
	\end{itemize}
	\item \textbf{Postcondizioni:} 
	\begin{itemize}
		\item L’utente Amministratore viene riconosciuto dal sistema;
	\end{itemize}
	\item \textbf{Scenario principale:} 
	\begin{itemize}
		\item L’ amministratore inserisce il proprio nome utente nel form di accesso (\hyperref[sec:UC1.1]{\textbf{UC1.1}});
		\item L’ amministratore inserisce la propria password nel form di accesso (\hyperref[sec:UC1.2]{\textbf{UC1.2}});
		\item Il sistema verifica che le credenziali ricevute siano corrette. 
	\end{itemize}
	\item \textbf{Estensioni:} Nel caso le credenziali non siano corrette:
	\begin{itemize}
		\item viene mostrato un errore - \hyperref[sec:UC2]{\textbf{UC2}}
	\end{itemize}
\end{itemize}

\subsubsection{UC1.1 - Inserimento nome utente}
\label{sec:UC1.1}
%\includegraphics[]{diagramma_UML}
\begin{itemize}
	\item \textbf{Descrizione:} L’amministratore inserisce il proprio nome utente;
	\item \textbf{Attori:} amministratore;
	\item \textbf{Precondizioni:} 
	\begin{itemize}
		\item L’amministratore possiede le credenziali di accesso;
		\item L’amministratore non ha già effettuato l’accesso;
		\item L’amministratore sta effettuando il login (\hyperref[sec:UC1]{\textbf{UC1}})
	\end{itemize}
	\item \textbf{Postcondizioni:} 
	\begin{itemize}
		\item L’amministratore ha inserito correttamente il proprio nome utente;
	\end{itemize}
	\item \textbf{Scenario principale:} 
	\begin{itemize}
		\item L’amministratore inserisce il proprio nome utente nel form di accesso.
	\end{itemize}
\end{itemize}

\subsubsection{UC1.2 - Inserimento password}
\label{sec:UC1.2}
%\includegraphics[]{diagramma_UML}
\begin{itemize}
	\item \textbf{Descrizione:} L’amministratore inserisce la propria password;
	\item \textbf{Attori:} amministratore;
	\item \textbf{Precondizioni:} 
	\begin{itemize}
		\item L’amministratore possiede le credenziali di accesso;
		\item L’amministratore non ha già effettuato l’accesso;
		\item L’amministratore sta effettuando il login (\hyperref[sec:UC1]{\textbf{UC1}})
	\end{itemize}
	\item \textbf{Postcondizioni:} 
	\begin{itemize}
		\item L’amministratore ha inserito correttamente la propria password;
	\end{itemize}
	\item \textbf{Scenario principale:} 
	\begin{itemize}
		\item L’amministratore inserisce la propria password nel form di accesso.
	\end{itemize}
\end{itemize}

\subsection{UC2 - Credenziali login errate}
\label{sec:UC2}
%\includegraphics[]{diagramma_UML}
\begin{itemize}
	\item \textbf{Descrizione:} L’amministratore visualizza un messaggio di errore di autenticazione;
	\item \textbf{Attori:} amministratore;
	\item \textbf{Precondizioni:} 
	\begin{itemize}
		\item L’amministratore possiede le credenziali di accesso;
		\item L’amministratore non ha già effettuato l’accesso;
		\item L’amministratore sta effettuando il login (\hyperref[sec:UC1]{\textbf{UC1}});
		\item L’amministratore ha inserito il proprio nome utente (\hyperref[sec:UC1.1]{\textbf{UC1.1}});
		\item L’amministratore ha inserito la propria password (\hyperref[sec:UC1.2]{\textbf{UC1.2}});
	\end{itemize}
	\item \textbf{Postcondizioni:}
	\begin{itemize}
		\item L’amministratore non viene riconosciuto dal sistema e deve reinserire le proprie credenziali;
		\item L'amministrazione visualizza un messaggio di errore;
	\end{itemize}
	\item \textbf{Scenario principale:} 
	\begin{itemize}
		\item Il sistema verifica le credenziali ricevute siano corrette;
		\item Il sistema visualizza un messaggio di errore per le credenziali inserite se non corrette.
	\end{itemize}
\end{itemize}

\subsection{UC3 - Creazione Database}
\label{sec:UC3}
%\includegraphics[]{diagramma_UML}
\begin{itemize}
	\item \textbf{Descrizione:} l’amministratore vuole aggiungere la struttura di un database da poter interrogare;
	\item \textbf{Attori:} amministratore;
	\item \textbf{Precondizioni:} 
	\begin{itemize}
		\item L’amministratore ha effettuato il login (\hyperref[sec:UC1]{\textbf{UC1}});
		\item L’amministratore si trova nel pannello amministrativo;
	\end{itemize}
	\item \textbf{Postcondizioni:} 
	\begin{itemize}
		\item La struttura del database viene salvata nel programma;
	\end{itemize}
	\item \textbf{Scenario principale:} 
	\begin{itemize}
		\item L’amministratore inserisce il nome (\hyperref[sec:UC3.1]{\textbf{UC3.1}}) e la descrizione (\hyperref[sec:UC3.2]{\textbf{UC3.2}}) del database;
	\end{itemize}
	\item \textbf{Estensioni:} nel caso in cui venga inserito un nome già esistente:
	\begin{itemize}
		\item \hyperref[sec:UC4]{\textbf{UC4}} - Errore: nome Database già presente
	\end{itemize}
\end{itemize}

\subsubsection{UC3.1 - Inserimento nome Database}
\label{sec:UC3.1}
%\includegraphics[]{diagramma_UML}
\begin{itemize}
	\item \textbf{Descrizione:} l’amministratore deve inserire il nome del nuovo database da aggiungere;
	\item \textbf{Attori:} amministratore;
	\item \textbf{Precondizioni:} 
	\begin{itemize}
		\item L’amministratore ha effettuato il login (\hyperref[sec:UC1]{\textbf{UC1}});
		\item L’amministratore si trova nel pannello amministrativo;
		\item L’amministratore sta creando un nuovo database (\hyperref[sec:UC3]{\textbf{UC3}});
	\end{itemize}
	\item \textbf{Postcondizioni:} 
	\begin{itemize}
		\item L'amministratore ha inserito correttamente il nome del nuovo database;
	\end{itemize}
	\item \textbf{Scenario principale:} 
	\begin{itemize}
		\item L’amministratore inserisce il nome del database;
	\end{itemize}
\end{itemize}

\subsubsection{UC3.2 - Inserimento descrizione Database}
\label{sec:UC3.2}
%\includegraphics[]{diagramma_UML}
\begin{itemize}
	\item \textbf{Descrizione:} l’amministratore deve inserire la descrizione del nuovo database da aggiungere;
	\item \textbf{Attori:} amministratore;
	\item \textbf{Precondizioni:} 
	\begin{itemize}
		\item L’amministratore ha effettuato il login (\hyperref[sec:UC1]{\textbf{UC1}});
		\item L’amministratore si trova nel pannello amministrativo;
		\item L’amministratore sta creando un nuovo database (\hyperref[sec:UC3]{\textbf{UC3}});
	\end{itemize}
	\item \textbf{Postcondizioni:} 
	\begin{itemize}
		\item L'amministratore ha inserito correttamente la descrizione del nuovo database;
	\end{itemize}
	\item \textbf{Scenario principale:} 
	\begin{itemize}
		\item L’amministratore inserisce la descrizione del database;
	\end{itemize}
\end{itemize}

\subsection{UC4 - Errore: nome Database già presente}
\label{sec:UC4}
%\includegraphics[]{diagramma_UML}
\begin{itemize}
	\item \textbf{Descrizione:} L’amministratore visualizza un errore di creazione del database;
	\item \textbf{Attori:} amministratore;
	\item \textbf{Precondizioni:} 
	\begin{itemize}
		\item L’amministratore ha effettuato il login (\hyperref[sec:UC1]{\textbf{UC1}});
		\item L’amministratore si trova nel pannello amministrativo;
		\item L’amministratore sta creando un nuovo database (\hyperref[sec:UC3]{\textbf{UC3}});
	\end{itemize}
	\item \textbf{Postcondizioni:} 
	\begin{itemize}
		\item La struttura del database non viene salvata nel programma e l'amministratore visualizza un messaggio di errore;
	\end{itemize}
	\item \textbf{Scenario principale:} 
	\begin{itemize}
		\item L’amministratore inserisce il nome del database (\hyperref[sec:UC3.1]{\textbf{UC3.1}});
		\item L’amministratore inserisce la descrizione del database  (\hyperref[sec:UC3.2]{\textbf{UC3.2}});
		\item Il sistema verifica che non esista già un database con lo stesso nome;
		\item Il sistema visualizza un messaggio di errore per il nome inserito.
	\end{itemize}
\end{itemize}

\subsection{UC5 - Visualizzazione lista Strutture Database}
\label{sec:UC5}
%\includegraphics[]{diagramma_UML}
\begin{itemize}
	\item \textbf{Descrizione:} l’amministratore visualizza tutte le Strutture Database disponibili;
	\item \textbf{Attori:} amministratore;
	\item \textbf{Precondizioni:} 
	\begin{itemize}
		\item L’amministratore ha effettuato il login (\hyperref[sec:UC1]{\textbf{UC1}});
		\item L’amministratore si trova nel pannello amministrativo;
	\end{itemize}
	\item \textbf{Postcondizioni:} 
	\begin{itemize}
		\item L'amministratore può vedere nome e descrizione dei database presenti;
	\end{itemize}
	\item \textbf{Scenario principale:} 
	\begin{itemize}
		\item Il programma visualizza la lista dei database presenti, con la possibilità di modificarli, visualizzarli o eliminarli;
	\end{itemize}
\end{itemize}

\subsection{UC6 - Visualizzazione singola Struttura Database}
\label{sec:UC6}
%\includegraphics[]{diagramma_UML}
\begin{itemize}
	\item \textbf{Descrizione:} l’amministratore visualizza una Struttura Database;
	\item \textbf{Attori:} amministratore;
	\item \textbf{Precondizioni:} 
	\begin{itemize}
		\item L’amministratore ha effettuato il login (\hyperref[sec:UC1]{\textbf{UC1}});
		\item L’amministratore si trova nel pannello amministrativo;
		\item L'amministratore ha selezionato una Struttura Database dalla lista;
	\end{itemize}
	\item \textbf{Postcondizioni:} 
	\begin{itemize}
		\item L'amministratore visualizza il nome della Struttura Database;
		\item L'amministratore visualizza la descrizione della Struttura Database;
		\item L'amministratore visualizza la lista delle tabelle della Struttura Database  (\hyperref[sec:UC12]{\textbf{UC12}};
	\end{itemize}
	\item \textbf{Scenario principale:} 
	\begin{itemize}
		\item Il programma mostra la lista dei database presenti, con la possibilità di modificarli, visualizzarli o eliminarli;
	\end{itemize}
\end{itemize}


\subsection{UC7 - Modifica Struttura Database}
\label{sec:UC7}
%\includegraphics[]{diagramma_UML}
\begin{itemize}
	\item \textbf{Descrizione:} l’amministratore modifica la Strutture Database selezionata;
	\item \textbf{Attori:} amministratore;
	\item \textbf{Precondizioni:} 
	\begin{itemize}
		\item L’amministratore sta visualizzando una struttura Database (\hyperref[sec:UC6]{\textbf{UC6}});
	\end{itemize}
	\item \textbf{Postcondizioni:} 
	\begin{itemize}
		\item Il sistema aggiorna la Struttura Database;
	\end{itemize}
	\item \textbf{Scenario principale:} 
	\begin{itemize}
		\item Il sistema verifica il nome ricevuto;
		\item Il sistema ve
	\end{itemize}
\end{itemize}

\subsection{UC8 - Elimina Database}
\label{sec:UC8}
%\includegraphics[]{diagramma_UML}
\begin{itemize}
	\item \textbf{Descrizione:} l’amministratore elimina la Struttura Database selezionata;
	\item \textbf{Attori:} amministratore;
	\item \textbf{Precondizioni:} 
	\begin{itemize}
		\item L’amministratore ha effettuato il login (\hyperref[sec:UC1]{\textbf{UC1}});
		\item L’amministratore si trova nel pannello amministrativo;
		\item L’amministratore sta visualizzando la lista dei database (\hyperref[sec:UC5]{\textbf{UC5}});
	\end{itemize}
	\item \textbf{Postcondizioni:} 
	\begin{itemize}
		\item La Struttura Database selezionata viene eliminata dal sistema;
	\end{itemize}
	\item \textbf{Scenario principale:} 
	\begin{itemize}
		\item L'amministratore seleziona il database da eliminare usando il pulsante di eliminazione apposito;
		\item Il sistema visualizza un messaggio per chiedere la conferma dell'eliminazione;
		\item Se l'amministrazione conferma l'eliminazione, il database e le tabelle collegate verranno rimossi dal sistema e verrà visualizzato un messaggio di avvenuta eliminazione.
	\end{itemize}
\end{itemize}

\subsection{UC9 - Creazione tabella Database}
\label{sec:UC9}
%\includegraphics[]{diagramma_UML}
\begin{itemize}
	\item \textbf{Descrizione:} l’amministratore vuole aggiungere una tabella alla struttura del database da interrogare;
	\item \textbf{Attori:} amministratore;
	\item \textbf{Precondizioni:} 
	\begin{itemize}
		\item L’amministratore ha effettuato il login (\hyperref[sec:UC1]{\textbf{UC1}});
		\item L’amministratore si trova nel pannello amministrativo;
		\item L’amministratore si trova nella sezione di creazione di una nuova tabella;
	\end{itemize}
	\item \textbf{Postcondizioni:} 
	\begin{itemize}
		\item La tabella viene aggiunta alla struttura del database;
	\end{itemize}
	\item \textbf{Scenario principale:} 
	\begin{itemize}
		\item L’amministratore inserisce il nome, i sinonimi del nome e la descrizione della tabella;
	\end{itemize}
	\item \textbf{Generalizzazioni:} 
	\begin{itemize}
		\item \hyperref[sec:UC9.1]{\textbf{UC9.1}} - Inserimento nome tabella
		\item \hyperref[sec:UC9.2]{\textbf{UC9.2}} - Inserimento sinonimi tabella
		\item \hyperref[sec:UC9.3]{\textbf{UC9.3}} - Inserimento descrizione tabella
	\end{itemize}
	\item \textbf{Estensioni:} nel caso in cui non vengano inseriti i sinonimi del nome della tabella, o il nome esisti già:
	\begin{itemize}
		\item \hyperref[sec:UC9]{\textbf{UC9}} - Errore nella creazione della tabella
	\end{itemize}
\end{itemize}

\subsubsection{UC9.1 - Inserimento nome tabella}
\label{sec:UC9.1}
%\includegraphics[]{diagramma_UML}
\begin{itemize}
	\item \textbf{Descrizione:} l’amministratore inserisce il nome della tabella da creare;
	\item \textbf{Attori:} amministratore;
	\item \textbf{Precondizioni:} 
	\begin{itemize}
		\item L’amministratore ha effettuato il login (\hyperref[sec:UC1]{\textbf{UC1}});
		\item L’amministratore sta creando una nuova tabella (\hyperref[sec:UC9]{\textbf{UC9}});
	\end{itemize}
	\item \textbf{Postcondizioni:} 
	\begin{itemize}
		\item Il nome della tabella viene inserito nel form;
	\end{itemize}
	\item \textbf{Scenario principale:} 
	\begin{itemize}
		\item L’amministratore inserisce il nome della tabella nell'apposito form di creazione;
	\end{itemize}
\end{itemize}

\subsubsection{UC9.2 - Inserimento sinonimi tabella}
\label{sec:UC9.2}
%\includegraphics[]{diagramma_UML}
\begin{itemize}
	\item \textbf{Descrizione:} l’amministratore inserisce i sinonimi associati al nome della tabella da creare;
	\item \textbf{Attori:} amministratore;
	\item \textbf{Precondizioni:} 
	\begin{itemize}
		\item L’amministratore ha effettuato il login (\hyperref[sec:UC1]{\textbf{UC1}});
		\item L’amministratore sta creando una nuova tabella (\hyperref[sec:UC9]{\textbf{UC9}});
	\end{itemize}
	\item \textbf{Postcondizioni:} 
	\begin{itemize}
		\item I sinonimi del nome della tabella vengono inseriti nel form;
	\end{itemize}
	\item \textbf{Scenario principale:} 
	\begin{itemize}
		\item L’amministratore inserisce i sinonimi del nome della tabella nell'apposito form di creazione;
	\end{itemize}
\end{itemize}

\subsubsection{UC9.3 - Inserimento descrizione tabella}
\label{sec:UC9.3}
%\includegraphics[]{diagramma_UML}
\begin{itemize}
	\item \textbf{Descrizione:} l’amministratore inserisce la descrizione della tabella da creare;
	\item \textbf{Attori:} amministratore;
	\item \textbf{Precondizioni:} 
	\begin{itemize}
		\item L’amministratore ha effettuato il login (\hyperref[sec:UC1]{\textbf{UC1}});
		\item L’amministratore sta creando una nuova tabella (\hyperref[sec:UC9]{\textbf{UC9}});
	\end{itemize}
	\item \textbf{Postcondizioni:} 
	\begin{itemize}
		\item La descrizione della tabella viene inserita nel form;
	\end{itemize}
	\item \textbf{Scenario principale:} 
	\begin{itemize}
		\item L’amministratore inserisce la descrizione della tabella nell'apposito form di creazione;
	\end{itemize}
\end{itemize}

\subsection{UC10 - Errore nella creazione della tabella}
\label{sec:UC10}
%\includegraphics[]{diagramma_UML}
\begin{itemize}
	\item \textbf{Descrizione:} L’amministratore visualizza un errore di creazione della tabella;
	\item \textbf{Attori:} amministratore;
	\item \textbf{Precondizioni:} 
	\begin{itemize}
		\item L’amministratore ha effettuato il login (\hyperref[sec:UC1]{\textbf{UC1}});
		\item L’amministratore sta creando una nuova tabella (\hyperref[sec:UC9]{\textbf{UC9}});
	\end{itemize}
	\item \textbf{Postcondizioni:} 
	\begin{itemize}
		\item La tabella non viene creata e il programma visualizza un messaggio di errore;
	\end{itemize}
	\item \textbf{Scenario principale:} 
	\begin{itemize}
		\item L’amministratore inserisce il nome della tabella nel form di creazione (\hyperref[sec:UC8.1]{\textbf{UC8.1}});
		\item L’amministratore inserisce i sinonimi del nome della tabella nel form di creazione (\hyperref[sec:UC8.2]{\textbf{UC8.2}});
		\item L’amministratore inserisce la descrizione della tabella nel form di creazione (\hyperref[sec:UC8.3]{\textbf{UC8.3}});
		\item Il sistema verifica che non esista già una tabella con lo stesso nome e che vengano inseriti sinonimi e descrizione della tabella;
		\item Il sistema visualizza il messaggio di errore opportuno.
	\end{itemize}
\end{itemize}

\subsubsection{UC10.1 - Errore nome tabella già presente}
\label{sec:UC10.1}
%\includegraphics[]{diagramma_UML}
\begin{itemize}
	\item \textbf{Descrizione:} L’amministratore visualizza un errore relativo al nome della tabella;
	\item \textbf{Attori:} amministratore;
	\item \textbf{Precondizioni:} 
	\begin{itemize}
		\item L’amministratore ha effettuato il login (\hyperref[sec:UC1]{\textbf{UC1}});
		\item L’amministratore sta creando una nuova tabella (\hyperref[sec:UC9]{\textbf{UC9}});
	\end{itemize}
	\item \textbf{Postcondizioni:} 
	\begin{itemize}
		\item La tabella non viene creata e il programma visualizza un messaggio di errore;
	\end{itemize}
	\item \textbf{Scenario principale:} 
	\begin{itemize}
		\item Il sistema verifica che non esista già una tabella con lo stesso nome;
		\item Il sistema visualizza il messaggio di errore per il nome inserito.
	\end{itemize}
\end{itemize}

\subsubsection{UC10.2 - Errore sinonimi non inseriti}
\label{sec:UC10.2}
%\includegraphics[]{diagramma_UML}
\begin{itemize}
	\item \textbf{Descrizione:} L’amministratore visualizza un errore relativo ai sinonimi del nome della tabella;
	\item \textbf{Attori:} amministratore;
	\item \textbf{Precondizioni:} 
	\begin{itemize}
		\item L’amministratore ha effettuato il login (\hyperref[sec:UC1]{\textbf{UC1}});
		\item L’amministratore sta creando una nuova tabella (\hyperref[sec:UC9]{\textbf{UC9}});
	\end{itemize}
	\item \textbf{Postcondizioni:} 
	\begin{itemize}
		\item La tabella non viene creata e il programma visualizza un messaggio di errore;
	\end{itemize}
	\item \textbf{Scenario principale:} 
	\begin{itemize}
		\item Il sistema verifica che il campo relativo ai sinonimi del nome della tabella non sia vuoto;
		\item Il sistema visualizza il messaggio di errore per il campo sinonimi vuoto.
	\end{itemize}
\end{itemize}

\subsubsection{UC10.3 - Errore descrizione non inserita}
\label{sec:UC10.3}
%\includegraphics[]{diagramma_UML}
\begin{itemize}
	\item \textbf{Descrizione:} L’amministratore visualizza un errore relativo alla descrizione della tabella;
	\item \textbf{Attori:} amministratore;
	\item \textbf{Precondizioni:} 
	\begin{itemize}
		\item L’amministratore ha effettuato il login (\hyperref[sec:UC1]{\textbf{UC1}});
		\item L’amministratore sta creando una nuova tabella (\hyperref[sec:UC9]{\textbf{UC9}});
	\end{itemize}
	\item \textbf{Postcondizioni:} 
	\begin{itemize}
		\item La tabella non viene creata e il programma visualizza un messaggio di errore;
	\end{itemize}
	\item \textbf{Scenario principale:} 
	\begin{itemize}
		\item Il sistema verifica che il campo relativo ai sinonimi del nome della tabella non sia vuoto;
		\item Il sistema visualizza il messaggio di errore per il campo descrizione vuoto.
	\end{itemize}
\end{itemize}

\subsection{UC11 - Modifica della tabella}
\label{sec:UC11}

\subsection{UC12 - Visualizzazione lista delle tabella della Struttura Database}
\label{sec:UC12}
%\includegraphics[]{diagramma_UML}
\begin{itemize}
	\item \textbf{Descrizione:} l’amministratore visualizza tutte le tabelle della Struttura Database selezionata;
	\item \textbf{Attori:} amministratore;
	\item \textbf{Precondizioni:} 
	\begin{itemize}
		\item L’amministratore ha effettuato il login (\hyperref[sec:UC1]{\textbf{UC1}});
		\item L’amministratore si trova nel pannello amministrativo;
		\item L'amministratore ha selezionato una Struttura Database dalla lista;
	\end{itemize}
	\item \textbf{Postcondizioni:} 
	\begin{itemize}
		\item L'amministratore può vedere il nome delle tabelle presenti;
	\end{itemize}
	\item \textbf{Scenario principale:} 
	\begin{itemize}
		\item Il programma visualizza la lista delle tabelle presenti, con la possibilità di modificarle, visualizzarle o eliminarle;
	\end{itemize}
\end{itemize}

\subsection{UC13 - Visualizzazione della singola tabella}
\label{sec:UC13}
%\includegraphics[]{diagramma_UML}
\begin{itemize}
	\item \textbf{Descrizione:} l’amministratore visualizza la tabella della Struttura Database selezionata;
	\item \textbf{Attori:} amministratore;
	\item \textbf{Precondizioni:} 
	\begin{itemize}
		\item L'amministratore ha visualizzato la lista delle tabelle della Struttura Database (\hyperref[sec:UC12]{\textbf{UC12}});
		\item L'amministratore ha selezionato una tabella dalla lista;
	\end{itemize}
	\item \textbf{Postcondizioni:} 
	\begin{itemize}
		\item L'amministratore può vedere nome, descrizione e sinonimi della tabella selezionata;
		\item L'amministratore può visualizzare i campi della tabella selezionata;
	\end{itemize}
	\item \textbf{Scenario principale:} 
	\begin{itemize}
		\item Il programma visualizza i campi della tabella selezionata;
	\end{itemize}
\end{itemize}

\subsection{UC14 - Eliminazione della tabella}
\label{sec:UC14}
%\includegraphics[]{diagramma_UML}
\begin{itemize}
	\item \textbf{Descrizione:} l’amministratore elimina la tabella selezionata;
	\item \textbf{Attori:} amministratore;
	\item \textbf{Precondizioni:} 
	\begin{itemize}
		\item L’amministratore ha effettuato il login (\hyperref[sec:UC1]{\textbf{UC1}});
		\item L’amministratore si trova nel pannello amministrativo;
		\item L’amministratore sta visualizzando la lista delle tabelle (\hyperref[sec:UC12]{\textbf{UC12}});
	\end{itemize}
	\item \textbf{Postcondizioni:} 
	\begin{itemize}
		\item La Struttura Database selezionata viene eliminata dal sistema;
	\end{itemize}
	\item \textbf{Scenario principale:} 
	\begin{itemize}
		\item L'amministratore seleziona la tabella da eliminare usando il pulsante di eliminazione apposito;
		\item Il sistema visualizza un messaggio per chiedere la conferma dell'eliminazione;
		\item Se l'l'amministratore conferma l'eliminazione, la tabella e i suoi campi verranno rimossi dal sistema e verrà visualizzato un messaggio di avvenuta eliminazione.
	\end{itemize}
\end{itemize}

\subsection{UC15 - Creazione campo tabella}
\label{sec:UC15}
%\includegraphics[]{diagramma_UML}
\begin{itemize}
	\item \textbf{Descrizione:} l’amministratore vuole aggiungere un campo alla tabella selezionata;
	\item \textbf{Attori:} amministratore;
	\item \textbf{Precondizioni:} 
	\begin{itemize}
		\item L’amministratore ha effettuato il login (\hyperref[sec:UC1]{\textbf{UC1}});
		\item L’amministratore si trova nel pannello amministrativo;
		\item L’amministratore si trova nella sezione di visualizzazione di una tabella (\hyperref[sec:UC13]{\textbf{UC13}});
		\item L’amministratore sta inserendo i campi che compongono la tabella;
	\end{itemize}
	\item \textbf{Postcondizioni:} 
	\begin{itemize}
		\item I campi vengono aggiunti alla tabella;
	\end{itemize}
	\item \textbf{Scenario principale:} 
	\begin{itemize}
		\item L’amministratore inserisce il nome del campo \hyperref[sec:UC15.1]{\textbf{UC15.1}};
		\item L'amministratore seleziona il tipo del campo \hyperref[sec:UC15.2]{\textbf{UC15.2}};
		\item L'amministratore inserisce i sinonimi del campo \hyperref[sec:UC15.3]{\textbf{UC15.3}};
	\end{itemize}
	\item \textbf{Estensioni:} nel caso in cui il nome inserito sia già esistente o non sia stato selezionato il tipo o inseriti i sinonimi:
	\begin{itemize}
		\item \hyperref[sec:UC16]{\textbf{UC16}} - Errore creazione campo
	\end{itemize}
\end{itemize}

\subsubsection{UC15.1 - Inserimento nome campo}
\label{sec:UC15.1}
%\includegraphics[]{diagramma_UML}
\begin{itemize}
	\item \textbf{Descrizione:} l’amministratore vuole inserire il nome del campo da inserire nella tabella;
	\item \textbf{Attori:} amministratore;
	\item \textbf{Precondizioni:} 
	\begin{itemize}
		\item L’amministratore ha effettuato il login (\hyperref[sec:UC1]{\textbf{UC1}});
		\item L’amministratore si trova nel pannello amministrativo;
		\item L’amministratore si trova nella sezione di visualizzazione di una tabella (\hyperref[sec:UC13]{\textbf{UC13}});
		\item L’amministratore sta inserendo i campi che compongono la tabella;
	\end{itemize}
	\item \textbf{Postcondizioni:} 
	\begin{itemize}
		\item Il nome del campo viene inserito;
	\end{itemize}
	\item \textbf{Scenario principale:} 
	\begin{itemize}
		\item L’amministratore inserisce il nome del campo nella casella di testo dedicata;
	\end{itemize}
	\item \textbf{Estensioni:} nel caso in cui il nome inserito sia già esistente:
	\begin{itemize}
		\item \hyperref[sec:UC16.1]{\textbf{UC16.1}} - Errore nome campo già esistente
	\end{itemize}
\end{itemize}

\subsubsection{UC15.2 - Inserimento tipo campo}
\label{sec:UC15.2}
%\includegraphics[]{diagramma_UML}
\begin{itemize}
	\item \textbf{Descrizione:} l’amministratore vuole selezionare il tipo del campo da inserire nella tabella;
	\item \textbf{Attori:} amministratore;
	\item \textbf{Precondizioni:} 
	\begin{itemize}
		\item L’amministratore ha effettuato il login (\hyperref[sec:UC1]{\textbf{UC1}});
		\item L’amministratore si trova nel pannello amministrativo;
		\item L’amministratore si trova nella sezione di visualizzazione di una tabella (\hyperref[sec:UC13]{\textbf{UC13}}) ;
		\item L’amministratore sta inserendo i campi che compongono la tabella;
	\end{itemize}
	\item \textbf{Postcondizioni:} 
	\begin{itemize}
		\item Il tipo del campo viene selezionato;
	\end{itemize}
	\item \textbf{Scenario principale:} 
	\begin{itemize}
		\item L’amministratore sceglie il tipo di campo, selezionandolo dalle scelte possibili;
	\end{itemize}
	\item \textbf{Estensioni:} nel caso in cui il tipo non venga selezionato:
	\begin{itemize}
		\item \hyperref[sec:UC16.2]{\textbf{UC16.2}} - Errore tipo campo non selezionato
	\end{itemize}
\end{itemize}

\subsubsection{UC15.3 - Inserimento sinonimi campo}
\label{sec:UC15.3}
%\includegraphics[]{diagramma_UML}
\begin{itemize}
	\item \textbf{Descrizione:} l’amministratore vuole selezionare il tipo del campo da inserire nella tabella;
	\item \textbf{Attori:} amministratore;
	\item \textbf{Precondizioni:} 
	\begin{itemize}
		\item L’amministratore ha effettuato il login (\hyperref[sec:UC1]{\textbf{UC1}});
		\item L’amministratore si trova nel pannello amministrativo;
		\item L’amministratore si trova nella sezione di visualizzazione di una tabella (\hyperref[sec:UC13]{\textbf{UC13}} ;
		\item L’amministratore sta inserendo i campi che compongono la tabella;
	\end{itemize}
	\item \textbf{Postcondizioni:} 
	\begin{itemize}
		\item I sinonimi del nome del campo vengono inseriti;
	\end{itemize}
	\item \textbf{Scenario principale:} 
	\begin{itemize}
		\item L’amministratore inserisce i sinonimi del nome del campo nella casella di testo dedicata;
	\end{itemize}
	\item \textbf{Estensioni:} nel caso in cui i sinonimi non vengano inseriti:
	\begin{itemize}
		\item \hyperref[sec:UC16.3]{\textbf{UC16.3}} - Errore mancato inserimento sinonimi campo
	\end{itemize}
\end{itemize}

\subsection{UC16 - Errore creazione campo}
\label{sec:UC16}
%\includegraphics[]{diagramma_UML}
\begin{itemize}
	\item \textbf{Descrizione:} L’amministratore visualizza un errore di creazione del campo;
	\item \textbf{Attori:} amministratore;
	\item \textbf{Precondizioni:} 
	\begin{itemize}
		\item L’amministratore ha effettuato il login (\hyperref[sec:UC1]{\textbf{UC1}});
		\item L’amministratore si trova nel pannello amministrativo;
		\item L’amministratore si trova nella sezione di visualizzazione di una tabella (\hyperref[sec:UC13]{\textbf{UC13}} ;
		\item L’amministratore sta creando un nuovo campo (\hyperref[sec:UC15]{\textbf{UC15}});
	\end{itemize}
	\item \textbf{Postcondizioni:} 
	\begin{itemize}
		\item Il campo non viene creato e il programma visualizza un messaggio di errore;
	\end{itemize}
	\item \textbf{Scenario principale:} 
	\begin{itemize}
		\item L’amministratore inserisce il nome del campo nel form di creazione (\hyperref[sec:UC13.1]{\textbf{UC13.1}});
		\item L’amministratore seleziona il tipo del campo nel form di creazione (\hyperref[sec:UC13.2]{\textbf{UC13.2}});
		\item L’amministratore inserisce i sinonimi del nome campo nel form di creazione (\hyperref[sec:UC13.3]{\textbf{UC13.3}});
		\item Il sistema verifica che non esista già una tabella con lo stesso nome, che venga selezionato il tipo e che vengano inseriti i sinonimi del campo;
		\item Il sistema visualizza il messaggio di errore opportuno.
	\end{itemize}
\end{itemize}

\subsubsection{UC16.1 - Errore nome campo già esistente}
\label{sec:UC16.1}
%\includegraphics[]{diagramma_UML}
\begin{itemize}
	\item \textbf{Descrizione:} L’amministratore visualizza un errore relativo al nome del campo;
	\item \textbf{Attori:} amministratore;
	\item \textbf{Precondizioni:} 
	\begin{itemize}
		\item L’amministratore ha effettuato il login (\hyperref[sec:UC1]{\textbf{UC1}});
		\item L’amministratore si trova nel pannello amministrativo;
		\item L’amministratore si trova nella sezione di visualizzazione di una tabella (\hyperref[sec:UC13]{\textbf{UC13}} ;
		\item L’amministratore sta creando un nuovo campo (\hyperref[sec:UC15]{\textbf{UC15}});
	\end{itemize}
	\item \textbf{Postcondizioni:} 
	\begin{itemize}
		\item Il campo non viene creato e il programma visualizza un messaggio di errore;
	\end{itemize}
	\item \textbf{Scenario principale:} 
	\begin{itemize}
		\item Il sistema verifica che non esista già un campo con lo stesso nome;
		\item Il sistema visualizza il messaggio di errore per il nome inserito.
	\end{itemize}
\end{itemize}

\subsubsection{UC16.2 - Errore tipo campo non selezionato}
\label{sec:UC16.2}
%\includegraphics[]{diagramma_UML}
\begin{itemize}
	\item \textbf{Descrizione:} L’amministratore visualizza un errore relativo al tipo del campo;
	\item \textbf{Attori:} amministratore;
	\item \textbf{Precondizioni:} 
	\begin{itemize}
		\item L’amministratore ha effettuato il login (\hyperref[sec:UC1]{\textbf{UC1}});
		\item L’amministratore si trova nel pannello amministrativo;
		\item L’amministratore si trova nella sezione di visualizzazione di una tabella (\hyperref[sec:UC13]{\textbf{UC13}} ;
		\item L’amministratore sta creando un nuovo campo (\hyperref[sec:UC15]{\textbf{UC15}});
	\end{itemize}
	\item \textbf{Postcondizioni:} 
	\begin{itemize}
		\item Il campo non viene creato e il programma visualizza un messaggio di errore;
	\end{itemize}
	\item \textbf{Scenario principale:} 
	\begin{itemize}
		\item Il sistema verifica che sia stato selezionato un tipo tra quelli disponibili per il campo;
		\item Il sistema visualizza il messaggio di errore per il tipo del campo.
	\end{itemize}
\end{itemize}

\subsubsection{UC16.3 - Errore mancato inserimento sinonimi campo}
\label{sec:UC16.3}
%\includegraphics[]{diagramma_UML}
\begin{itemize}
	\item \textbf{Descrizione:} L’amministratore visualizza un errore relativo ai sinonimi del nome del campo;
	\item \textbf{Attori:} amministratore;
	\item \textbf{Precondizioni:} 
	\begin{itemize}
		\item L’amministratore ha effettuato il login (\hyperref[sec:UC1]{\textbf{UC1}});
		\item L’amministratore si trova nel pannello amministrativo;
		\item L’amministratore si trova nella sezione di visualizzazione di una tabella (\hyperref[sec:UC13]{\textbf{UC13}} ;
		\item L’amministratore sta creando un nuovo campo (\hyperref[sec:UC15]{\textbf{UC15}});
	\end{itemize}
	\item \textbf{Postcondizioni:} 
	\begin{itemize}
		\item Il campo non viene creato e il programma visualizza un messaggio di errore;
	\end{itemize}
	\item \textbf{Scenario principale:} 
	\begin{itemize}
		\item Il sistema verifica che siano stati inseriti dei sinonimi per il nome del campo;
		\item Il sistema visualizza il messaggio di errore per i sinonimi.
	\end{itemize}
\end{itemize}

\subsection{UC17 - Visualizzazione lista dei campi della tabella}
\label{sec:UC17}
%\includegraphics[]{diagramma_UML}
\begin{itemize}
	\item \textbf{Descrizione:} l’amministratore visualizza tutti i campi della tabella selezionata;
	\item \textbf{Attori:} amministratore;
	\item \textbf{Precondizioni:} 
	\begin{itemize}
		\item L’amministratore ha effettuato il login (\hyperref[sec:UC1]{\textbf{UC1}});
		\item L’amministratore si trova nella sezione di visualizzazione di una tabella (\hyperref[sec:UC13]{\textbf{UC13}} ;
		\item L’amministratore si trova nel pannello amministrativo;
	\end{itemize}
	\item \textbf{Postcondizioni:} 
	\begin{itemize}
		\item L'amministratore può vedere nome, tipo e sinonimi dei campi presenti;
	\end{itemize}
	\item \textbf{Scenario principale:} 
	\begin{itemize}
		\item Il programma visualizza la lista dei campi presenti, con la possibilità di modificarli, visualizzarli o eliminarli;
	\end{itemize}
\end{itemize}

\subsection{UC18 - Visualizzazione singolo campo tabella}
\label{sec:UC18}
%\includegraphics[]{diagramma_UML}
\begin{itemize}
	\item \textbf{Descrizione:} l’amministratore visualizza tutti i campi della tabella selezionata;
	\item \textbf{Attori:} amministratore;
	\item \textbf{Precondizioni:} 
	\begin{itemize}
		\item L’amministratore ha effettuato il login (\hyperref[sec:UC1]{\textbf{UC1}});
		\item L’amministratore si trova nella sezione di visualizzazione di una tabella (\hyperref[sec:UC13]{\textbf{UC13}});
		\item L’amministratore si trova nel pannello amministrativo;
		\item L'amministratore ha selezionato un campo di una tabella;
	\end{itemize}
	\item \textbf{Postcondizioni:} 
	\begin{itemize}
		\item L'amministratore può vedere nome, tipo e sinonimi del campo selezionatoi;
	\end{itemize}
	\item \textbf{Scenario principale:} 
	\begin{itemize}
		\item Il programma visualizza la lista dei campi presenti, con la possibilità di modificarli, visualizzarli o eliminarli;
	\end{itemize}
\end{itemize}

\subsection{UC19 - Modifica campo tabella}
\label{sec:UC19}
%\includegraphics[]{diagramma_UML}
\begin{itemize}
	\item \textbf{Descrizione:} l’amministratore vuole modificare un campo alla tabella selezionata;
	\item \textbf{Attori:} amministratore;
	\item \textbf{Precondizioni:} 
	\begin{itemize}
		\item L’amministratore ha effettuato il login (\hyperref[sec:UC1]{\textbf{UC1}});
		\item L’amministratore si trova nel pannello amministrativo;
		\item L’amministratore si trova nella sezione di visualizzazione di una tabella (\hyperref[sec:UC13]{\textbf{UC13}});
		\item L’amministratore sta modificando i campi che compongono la tabella;
	\end{itemize}
	\item \textbf{Postcondizioni:} 
	\begin{itemize}
		\item I campi della tabella vengono modificati;
	\end{itemize}
	\item \textbf{Scenario principale:} 
	\begin{itemize}
		\item L’amministratore inserisce il nome del campo \hyperref[sec:UC15.1]{\textbf{UC15.1}};
		\item L'amministratore seleziona il tipo del campo \hyperref[sec:UC15.2]{\textbf{UC15.2}};
		\item L'amministratore inserisce i sinonimi del campo \hyperref[sec:UC15.3]{\textbf{UC15.3}};
	\end{itemize}

		\item \textbf{Estensioni:} nel caso in cui il nome inserito sia già esistente o non sia stato selezionato il tipo o inseriti i sinonimi:
		\begin{itemize}
			\item (\hyperref[sec:UC16]{\textbf{UC16}}) - Errore creazione campo
	\end{itemize}
\end{itemize}


\subsection{UC20 - Eliminazione campo tabella}
\label{sec:UC20}
%\includegraphics[]{diagramma_UML}
\begin{itemize}
	\item \textbf{Descrizione:} l’amministratore elimina il campo selezionato;
	\item \textbf{Attori:} amministratore;
	\item \textbf{Precondizioni:} 
	\begin{itemize}
		\item L’amministratore ha effettuato il login (\hyperref[sec:UC1]{\textbf{UC1}});
		\item L’amministratore si trova nel pannello amministrativo;
		\item L’amministratore sta visualizzando la lista dei campi (\hyperref[sec:UC15]{\textbf{UC15}});
	\end{itemize}
	\item \textbf{Postcondizioni:} 
	\begin{itemize}
		\item Il campo selezionato viene eliminato dal sistema;
	\end{itemize}
	\item \textbf{Scenario principale:} 
	\begin{itemize}
		\item L'amministratore seleziona il campo da eliminare;
		\item Il sistema visualizza un messaggio per chiedere la conferma dell'eliminazione;
		\item Se l'amministrazione conferma l'eliminazione, il campo verrà rimosso dal sistema e verrà visualizzato un messaggio di avvenuta eliminazione.
	\end{itemize}
\end{itemize}

\subsection{UC21 - Caricamento struttura Database tramite file}
\label{sec:UC21}
\begin{itemize}
	\item \textbf{Descrizione:} l’amministratore vuole caricare un file che descrive una Struttura Database;
	\item \textbf{Attori:} amministratore;
	\item \textbf{Precondizioni:} 
	\begin{itemize}
		\item L’amministratore ha effettuato il login (\hyperref[sec:UC1]{\textbf{UC1}});
		\item L’amministratore si trova nel pannello amministrativo;
	\end{itemize}
	\item \textbf{Postcondizioni:} 
	\begin{itemize}
		\item La Struttura Database viene caricata correttamente nel sistema;
	\end{itemize}
	\item \textbf{Scenario principale:} 
	\begin{itemize}
		\item L'amministrazione inserisce il file della Struttura Database;
	\end{itemize}
	\item \textbf{Estensioni:} nel caso il file non sia correttoi:
	\begin{itemize}
		\item \hyperref[sec:UC22]{\textbf{UC22}} - Errore caricamento file
	\end{itemize}
\end{itemize}

\subsection{UC22 - Errore caricamento file}
\label{sec:UC22}
\begin{itemize}
	\item \textbf{Descrizione:} L’amministratore visualizza un errore di caricamento del file;
	\item \textbf{Attori:} amministratore;
	\item \textbf{Precondizioni:} 
	\begin{itemize}
		\item L’amministratore ha effettuato il login (\hyperref[sec:UC1]{\textbf{UC1}});
		\item L’amministratore si trova nel pannello amministrativo;
		\item L’amministratore sta caricando un file Struttura del Databese (\hyperref[sec:UC21]{\textbf{UC21}});
	\end{itemize}
	\item \textbf{Postcondizioni:} 
	\begin{itemize}
		\item La tabella non viene creata e il programma visualizza un messaggio di errore;
	\end{itemize}
	\item \textbf{Scenario principale:} 
	\begin{itemize}
		\item Il sistema verifica che il formato del file sia corretto;
		\item Il sistema mostra un messaggio di errore se il formato non è corretto;
	\end{itemize}
\end{itemize}


\subsection{UC23 - Logout}
\label{sec:UC23}
%\includegraphics[]{diagramma_UML}
\begin{itemize}
	\item \textbf{Descrizione: l'amministratore vuole fare il logout dall'area amministrativa} 
	\item \textbf{Attori:} amministratore;
	\item \textbf{Precondizioni:} 
	\begin{itemize}
		\item L’amministratore ha effettuato il login (\hyperref[sec:UC1]{\textbf{UC1}});
	\end{itemize}
	\item \textbf{Postcondizioni:} 
	\begin{itemize}
		\item Viene visualizzata la pagina iniziale;
	\end{itemize}
	\item \textbf{Scenario principale:} 
	\begin{itemize}
		\item L’amministratore clicca il pulsante di logout;
		\item L’amministratore viene reindirizzato alla pagina iniziale.
	\end{itemize}
	\item \textbf{Estensioni:} nel caso in cui l'utente non fosse loggato:
	\begin{itemize}
		\item Viene visualizzato un errore - \hyperref[sec:UC24]{\textbf{UC24}}.
	\end{itemize}
\end{itemize}

\subsection{UC24 - Errore logout}
\label{sec:UC24}

\subsection{UC25 - Selezione Struttura Database da interrogare}
\label{sec:UC25}
%\includegraphics[]{diagramma_UML}
\begin{itemize}
	\item \textbf{Descrizione:} l’utente seleziona la Struttura Database che vuole interrogare da una lista;
	\item \textbf{Attori:} utente;
	\item \textbf{Precondizioni:}
	\begin{itemize}
		\item Sono presenti una o più Strutture Database da poter selezionare;
	\end{itemize}
	\item \textbf{Postcondizioni:}
	\begin{itemize}
		\item L’utente ha selezionato una Struttura Database;
	\end{itemize}
	\item \textbf{Scenario principale:}
	\begin{itemize}
		\item L’utente ha il programma aperto;
		\item L’utente seleziona uno dei database presenti nella lista.
	\end{itemize}
	\item \textbf{Estensioni:} In caso l'utente non selezioni una Strttura Database;
	\begin{itemize}
		\item Errore Struttura Database non selezionata (\hyperref[sec:UC26]{\textbf{UC26}});
	\end{itemize}
\end{itemize}

\subsection{UC26 - Errore Struttura Database non selezionata}
\label{sec:UC26}
\begin{itemize}
	\item \textbf{Descrizione:} l’utente non ha selezionato una Struttura Database;
	\item \textbf{Attori:} utente;
	\item \textbf{Precondizioni:}
	\begin{itemize}
		\item Sono presenti una o più Strutture Database da poter selezionare;
	\end{itemize}
	\item \textbf{Postcondizioni:}
	\begin{itemize}
		\item L’utente visualizza un messaggio di errore;
	\end{itemize}
	\item \textbf{Scenario principale:}
	\begin{itemize}
		\item Il sistema non riceve una Struttura Database;
		\item Il sistema mostra un errore;
	\end{itemize}
\end{itemize}

\subsection{UC27 - Inserimento frase in linguaggio naturale}
\label{sec:UC27}
%\includegraphics[]{diagramma_UML}
\begin{itemize}
	\item \textbf{Descrizione:} L’utente scrive una frase in linguaggio naturale;
	\item \textbf{Attori:} utente;
	\item \textbf{Precondizioni:} 
	\begin{itemize}
		\item L’utente ha selezionato un file di struttura Database (\hyperref[sec:UC25]{\textbf{UC25}});
	\end{itemize}
	\item \textbf{Postcondizioni:} 
	\begin{itemize}
		\item L’utente riceve un prompt per la creazione della query richiesta (\hyperref[sec:UC29]{\textbf{UC29}});
	\end{itemize}
	\item \textbf{Scenario principale:} 
	\begin{itemize}
		\item L’utente ha il programma aperto;
		\item L’utente seleziona la casella di testo;
		\item L’utente scrive la frase per interrogare il database;
		\item L’utente clicca il pulsante apposito per ottenere il prompt.
	\end{itemize}
	\item \textbf{Estensioni:} nel caso in cui venga inserita una frase non inerente al database, o non comprensibile:
	\begin{itemize}
		\item Viene visualizzato un errore - \hyperref[sec:UC28]{\textbf{UC28}}.
	\end{itemize}
\end{itemize}

\subsection{UC28 - Errore frase naturale}
\label{sec:UC28}
\begin{itemize}
	\item \textbf{Descrizione:} L’utente visualizza un errore inerente alla frase inserita;
	\item \textbf{Attori:} utente;
	\item \textbf{Precondizioni:} 
	\begin{itemize}
		\item L’utente ha selezionato un file di struttura Database (\hyperref[sec:UC25]{\textbf{UC25}});
		\item L'utente ha inserito una frase in linguaggio naturale (\hyperref[sec:UC27]{\textbf{UC27}});
	\end{itemize}
	\item \textbf{Postcondizioni:} 
	\begin{itemize}
		\item L’utente visualizza un messaggio di errore;
	\end{itemize}
	\item \textbf{Scenario principale:} 
	\begin{itemize}
		\item Il sistema verifica la frase ricevuta;
		\item Il sistema non riescie ad elaborare la frase;
		\item Il sistema visualizza un messaggio di errore.
	\end{itemize}
	\end{itemize}


\subsubsection{UC28.1 - Errore frase non inserita}
\label{sec:UC28.1}
\begin{itemize}
	\item \textbf{Descrizione:} L’utente visualizza un errore relativo alla frase inserita;
	\item \textbf{Attori:} utente;
	\item \textbf{Precondizioni:} 
	\begin{itemize}
		\item L’utente ha selezionato un file di struttura Database (\hyperref[sec:UC25]{\textbf{UC25}});
		\item L'utente ha inserito una frase in linguaggio naturale (\hyperref[sec:UC27]{\textbf{UC27}});
	\end{itemize}
	\item \textbf{Postcondizioni:} 
	\begin{itemize}
		\item Il programma mostra un messaggio di errore;
		\item Il prompt non viene generato;
	\end{itemize}
	\item \textbf{Scenario principale:} 
	\begin{itemize}
		\item Il sistema non ha ricevuto una frase;
		\item Il sistema visualizza un messaggio di errore.
	\end{itemize}
\end{itemize}


\subsubsection{UC28.2 - Errore frase non compresa}
\label{sec:UC28.2}
\begin{itemize}
	\item \textbf{Descrizione:} L’utente inserisce una frase in linguaggio naturale non comprensibile dal sistema;
	\item \textbf{Attori:} utente;
	\item \textbf{Precondizioni:} 
	\begin{itemize}
		\item L’utente ha selezionato un file di struttura Database (\hyperref[sec:UC25]{\textbf{UC25}});
		\item L'utente ha inserito una frase in linguaggio naturale (\hyperref[sec:UC27]{\textbf{UC27}});
	\end{itemize}
	\item \textbf{Postcondizioni:} 
	\begin{itemize}
		\item Il programma mostra un messaggio di errore;
		\item Il prompt non viene generato;
	\end{itemize}
	\item \textbf{Scenario principale:} 
	\begin{itemize}
		\item Il sistema verifica la frase ricevuta;
		\item Il sistema non comprende la frase;
		\item Il sistema visualizza un messaggio di errore.
	\end{itemize}
\end{itemize}


\subsubsection{UC28.3 - Errore frase non inerente}
\label{sec:UC28.3}
%\includegraphics[]{diagramma_UML}
\begin{itemize}
	\item \textbf{Descrizione:} l’utente inserisce una frase in linguaggio naturale non interpretabile dal sistema come inerente al database;
	\item \textbf{Attori:} utente;
	\item \textbf{Precondizioni:} 
	\begin{itemize}
		\item L’utente ha selezionato un file di struttura Database (\hyperref[sec:UC25]{\textbf{UC22}});
		\item L’utente ha scritto una frase in linguaggio naturale nella casella di testo apposita e ne ha confermato l’inserimento (\hyperref[sec:UC27]{\textbf{UC24}});
	\end{itemize}
	\item \textbf{Postcondizioni:} 
	\begin{itemize}
		\item Il programma mostra un messaggio di errore;
		\item Il prompt non viene generato;
	\end{itemize}
	\item \textbf{Scenario principale:} 
	\begin{itemize}
		\item Il sistema verifica la frase ricevuta;
		\item Il sistema non trova componenti inerenti alla Struttura Database nella frase;
		\item Il sistema visualizza un messaggio di errore.
	\end{itemize}
\end{itemize}

\subsection{UC29 - Visualizzazione prompt generato}
\label{sec:UC29}
%\includegraphics[]{diagramma_UML}
\begin{itemize}
	\item \textbf{Descrizione:} L’utente riceve il prompt per la generazione della query;
	\item \textbf{Attori:} utente;
	\item \textbf{Precondizioni:} 
	\begin{itemize}
		\item L’utente ha selezionato un file di struttura Database (\hyperref[sec:UC25]{\textbf{UC22}});
		\item L’utente ha scritto una frase in linguaggio naturale (\hyperref[sec:UC27]{\textbf{UC24}});
	\end{itemize}
	\item \textbf{Postcondizioni:} 
	\begin{itemize}
		\item Il programma visualizza il prompt per la creazione della query richiesta;
	\end{itemize}
	\item \textbf{Scenario principale:} 
	\begin{itemize}
		\item L’utente ha il programma aperto;
		\item L’utente ha selezionato la Struttura Database (\hyperref[sec:UC25]{\textbf{UC25}})
		\item L'utente ha inserito la frase in linguaggio naturale (\hyperref[sec:UC27]{\textbf{UC27}});
		\item L’utente richiede il prompt;
		\item Il programma visualizza il prompt elaborato.
	\end{itemize}
	\item \textbf{Estensioni:} nel caso in cui non sia possibile generare il prompt:
	\begin{itemize}
		\item Viene visualizzato un errore - \hyperref[sec:UC30]{\textbf{UC30}}.
	\end{itemize}
\end{itemize}

\subsection{UC30 - Errore generazione prompt}
\label{sec:UC30}
\begin{itemize}
	\item \textbf{Descrizione:} L’utente visualizza un errore inerente alla generazione del prompt;
	\item \textbf{Attori:} utente;
	\item \textbf{Precondizioni:} 
	\begin{itemize}
		\item L’utente ha selezionato un file di struttura Database (\hyperref[sec:UC25]{\textbf{UC25}});
		\item L’utente ha scritto una frase in linguaggio naturale (\hyperref[sec:UC27]{\textbf{UC27}});
		\item L’utente richiede il prompt;
	\end{itemize}
	\item \textbf{Postcondizioni:} 
	\begin{itemize}
		\item L’utente visualizza un messaggio di errore;
	\end{itemize}
	\item \textbf{Scenario principale:} 
	\begin{itemize}
		\item Il sistema verifica il prompt e ritorna l'errore adeguato;
	\end{itemize}
\end{itemize}

\subsubsection{UC30.1 - Errore comunicazione con LLM}
\label{sec:UC30.1}
\begin{itemize}
	\item \textbf{Descrizione:} L’utente visualizza un errore inerente alla comunicazione con il modello LLM;
	\item \textbf{Attori:} utente;
	\item \textbf{Precondizioni:} 
	\begin{itemize}
		\item L’utente ha selezionato un file di struttura Database (\hyperref[sec:UC25]{\textbf{UC25}});
		\item L’utente ha scritto una frase in linguaggio naturale (\hyperref[sec:UC27]{\textbf{UC27}});
		\item L’utente richiede il prompt;
	\end{itemize}
	\item \textbf{Postcondizioni:} 
	\begin{itemize}
		\item L’utente visualizza un messaggio di errore;
	\end{itemize}
	\item \textbf{Scenario principale:} 
	\begin{itemize}
		\item Il sistema riceve il prompt;
		\item Il sistema non riesce a comunicare con il modello LLM;
		\item Il sistema mostra un errore;
	\end{itemize}
\end{itemize}

\subsubsection{UC30.2 - Errore dati mancanti per creazione prompt}
\label{sec:UC30.2}
\begin{itemize}
	\item \textbf{Descrizione:} L’utente visualizza un errore inerente alla frase in linguaggio naturale;
	\item \textbf{Attori:} utente;
	\item \textbf{Precondizioni:} 
	\begin{itemize}
		\item L’utente ha selezionato un file di struttura Database (\hyperref[sec:UC22]{\textbf{UC22}});
		\item L’utente ha scritto una frase in linguaggio naturale (\hyperref[sec:UC24]{\textbf{UC24}});
		\item L’utente richiede il prompt;
	\end{itemize}
	\item \textbf{Postcondizioni:} 
	\begin{itemize}
		\item L’utente visualizza un messaggio di errore;
	\end{itemize}
	\item \textbf{Scenario principale:} 
	\begin{itemize}
		\item Il sistema riceve il prompt;
		\item Il sistema non trova tutti i dati necessari per la generazione del prompt ;
		\item Il sistema mostra un errore;
	\end{itemize}
\end{itemize}

\subsection{UC31 - Richiesta generazione query SQL}
\label{sec:UC31}
%   \includegraphics[]{diagramma_UML}
\begin{itemize}
    \item \textbf{Descrizione:} L'utente richiede la generazione della query SQL;
    \item \textbf{Attori:} utente;
    \item \textbf{Precondizioni:} 
    \begin{itemize}
    	\item L'utente ha visualizzato il prompt (\hyperref[sec:UC29]{\textbf{UC29}});
    \end{itemize}
    \item \textbf{Postcondizioni:} 
    \begin{itemize}
    	\item L'utente visualizza la query SQL richiesta (\hyperref[sec:UC33]{\textbf{UC33}});
    \end{itemize}
    \item \textbf{Scenario principale:}
    \begin{itemize}
    	\item L'utente richiede la generazione della query SQL a partire dal prompt ricevuto;
    	\item Il sistema verifica il prompt ricevuto;
    	\item Il sistema invia il prompt al servizio LLM esterno;
    \end{itemize}
    \item \textbf{Estensioni:} In caso non sia possibile generare la query SQL:
    \begin{itemize}
    	\item Errore generazione query SQL (\hyperref[sec:UC32]{\textbf{UC32}});
    \end{itemize}
\end{itemize}

\subsection{UC32 - Errore generazione query SQL}
\label{sec:UC32}
%   \includegraphics[]{diagramma_UML}
\begin{itemize}
	\item \textbf{Descrizione:} L'utente visualizza un messaggio di errore inerente generazione della query SQL;
	\item \textbf{Attori:} utente;
	\item \textbf{Precondizioni:} 
	\begin{itemize}
		\item L'utente ha visualizzato il prompt (\hyperref[sec:UC29]{\textbf{UC29}});
		\item L'utente richiede la generazione della query SQL (\hyperref[sec:UC31]{\textbf{UC31}});
	\end{itemize}
	\item \textbf{Postcondizioni:} 
	\begin{itemize}
		\item L'utente un messaggio di errore;
	\end{itemize}
	\item \textbf{Scenario principale:}
	\begin{itemize}
		\item L'utente richiede la generazione della query SQL a partire dal prompt ricevuto;
		\item Il sistema verifica il prompt ricevuto;
		\item Il sistema invia il prompt al servizio LLM esterno;
		\item Il sistema mostra un messaggio di errore;
	\end{itemize}
\end{itemize}

\subsubsection{UC32.1 - Errore comunicazione con API}
\label{sec:UC32.1}
\begin{itemize}
	\item \textbf{Descrizione:} L'utente visualizza un messaggio di errore inerente alla comunicazione con il servizio di API esterno;
	\item \textbf{Attori:} utente;
	\item \textbf{Precondizioni:} 
	\begin{itemize}
		\item L'utente ha visualizzato il prompt (\hyperref[sec:UC29]{\textbf{UC29}});
		\item L'utente richiede la generazione della query SQL (\hyperref[sec:UC31]{\textbf{UC31}});
	\end{itemize}
	\item \textbf{Postcondizioni:} 
	\begin{itemize}
		\item L'utente un messaggio di errore;
	\end{itemize}
	\item \textbf{Scenario principale:}
	\begin{itemize}
		\item L'utente richiede la generazione della query SQL a partire dal prompt ricevuto;
		\item Il sistema verifica il prompt ricevuto;
		\item Il sistema invia il prompt al servizio LLM esterno;
		\item Il sistema non riescie a comunicare con il servizio LLM esterno;
		\item Il sistema mostra un messaggio di errore;
	\end{itemize}
\end{itemize}

\subsubsection{UC32.2 - Errore formattazione prompt}
\label{sec:UC32.2}
\begin{itemize}
	\item \textbf{Descrizione:} L'utente visualizza un messaggio di errore inerente alla formattazione del prompt creato;
	\item \textbf{Attori:} utente;
	\item \textbf{Precondizioni:} 
	\begin{itemize}
		\item L'utente ha visualizzato il prompt (\hyperref[sec:UC29]{\textbf{UC29}});
		\item L'utente richiede la generazione della query SQL (\hyperref[sec:UC31]{\textbf{UC31}});
	\end{itemize}
	\item \textbf{Postcondizioni:} 
	\begin{itemize}
		\item L'utente un messaggio di errore;
	\end{itemize}
	\item \textbf{Scenario principale:}
	\begin{itemize}
		\item L'utente richiede la generazione della query SQL a partire dal prompt ricevuto;
		\item Il sistema verifica il prompt ricevuto;
		\item Il sistema invia il prompt al servizio LLM esterno;
		\item Il sistema riceve un errore dal servizio LLM esterno;
		\item Il sistema mostra un messaggio di errore;
	\end{itemize}
\end{itemize}

\subsection{UC33 - Visualizzazione query SQL}
\label{sec:UC33}
\begin{itemize}
	\item \textbf{Descrizione:} L’utente visualizza la query SQL ricevuta dal servizio LLM esterno;
	\item \textbf{Attori:} utente;
	\item \textbf{Precondizioni:} 
	\begin{itemize}
		\item L’utente ha selezionato un file di struttura Database (\hyperref[sec:UC25]{\textbf{UC25}});
		\item L’utente ha inserito una frase in linguaggio naturale (\hyperref[sec:UC27]{\textbf{UC27}});
		\item Il programma visualizza il prompt elaborato (\hyperref[sec:UC29]{\textbf{UC29}});
		\item L’utente richiesto la generazione della query SQL (\hyperref[sec:UC31]{\textbf{UC31}});
	\end{itemize}
	\item \textbf{Postcondizioni:} 
	\begin{itemize}
		\item Il programma visualizza la query SQL richiesta;
	\end{itemize}
	\item \textbf{Scenario principale:} 
	\begin{itemize}
		\item L'utente richiede la generazione della query SQL (\hyperref[sec:UC31]{\textbf{UC31}});
		\item L'utente visualizza la query SQL richiesta;
	\end{itemize}
\end{itemize}


\section{Requisiti}
In questa sezione vengono descritti e organizzati in forma tabellare i requisiti che il prodotto finale dovrà soddisfare. La struttura usata viene descritta all'interno del documento Norme di Progetto.

\subsection{Requisiti funzionali}
\begin{center}
	\rowcolors{2}{gray!25}{white}
	\begin{tabular}{ |p{0.10\linewidth}|p{0.50\linewidth}|p{0.15\linewidth}|p{0.15\linewidth}| }
		\hline
		\textbf{Codice} & \textbf{Descrizione} & \textbf{Classificazione} & \textbf{Fonti} \\
		\hline
		% RF& descrizione & classificazione & fonti \\
		% \hline
		RF1 & L'amministratore deve poter effettuare il login per modificare i Database & Obbligatorio & UC1 - Verbale esterno \\
		\hline
		RF2 & Visualizzazione di errore in caso di credenziali errate durante il login & Obbligatorio & UC2 \\
		\hline
		RF3 & L'amministratore deve poter creare un nuovo database specificando nome e descrizione & Obbligatorio & UC3 \\
		\hline
		RF4 & Visualizzazione di errore in caso di tentativo di creazione di un database con un nome già in uso & Obbligatorio & UC4 \\
		\hline
		RF5 & L'amministratore deve poter visualizzare la lista dei database presenti & Obbligatorio & UC5 \\
		\hline
		RF6 & L'amministratore deve poter visualizzare i dettagli di un database & Obbligatorio & UC6 \\
		\hline
		RF7 & L'amministratore deve poter effettuare modifiche a nome e descrizione di un database & Obbligatorio & UC7 \\
		\hline
		RF8 & L'amministratore deve poter eliminare un database presente & Obbligatorio & UC8 \\
		\hline
		RF9 & L'amministratore deve poter creare una nuova tabella specificando nome, sinonimi e descrizione & Obbligatorio & UC9 \\
		\hline
		RF10& Visualizzazione di errore in caso di tentativo di creazione di una tabella con un nome già in uso o dati mancanti & Obbligatorio & UC10 \\
		\hline
		RF11 & L'amministratore deve poter effettuare modifiche a nome, sinonimi e descrizione della tabella & Obbligatorio & UC11 \\
		\hline
		RF12 & L'amministratore deve poter visualizzare la lista delle tabelle presenti in un database & Obbligatorio & UC12 \\
		\hline
		RF13 & L'amministratore deve poter visualizzare i dettagli di una tabella & Obbligatorio & UC13 \\
		\hline
		RF14 & L'amministratore deve poter eliminare una tabella presente in un database & Obbligatorio & UC14 \\
		\hline
		RF15 & L'amministratore deve poter creare un nuovo campo all'interno di una tabella specificando nome, tipo e sinonimi & Obbligatorio & UC15 \\
		\hline
		RF16 & Visualizzazione di errore in caso di tentativo di creazione di un campo con un nome già in uso o dati mancanti & Obbligatorio & UC16 \\
		\hline
		RF17 & L'amministratore deve poter visualizzare la lista dei campi presenti in una tabella & Obbligatorio & UC17 \\
		\hline
		RF18 & L'amministratore deve poter visualizzare i dettagli di un campo & Obbligatorio & UC18 \\
		\hline
		RF19 & L'amministratore deve poter effettuare modifiche a nome, tipo e sinonimi di un campo & Obbligatorio & UC19 \\
		\hline
		RF20 & L'amministratore deve poter eliminare un campo presente in una tabella & Obbligatorio & UC20 \\
		\hline
		RF21 & L'amministratore deve poter caricare la struttura di un nuovo database tramite file & Obbligatorio & UC21 - Capitolato \\
		\hline
		RF22 & Visualizzazione di errore in caso di problemi durante il caricamento del file & Desiderabile & UC22 \\
		\hline
		RF23 & L'amministratore deve poter effettuare il logout & Obbligatorio & UC23 \\
		\hline
		RF24 & Visualizzazione errore in caso di problemi in fase di logout & Desiderabile & UC24 \\
		\hline
		RF25 & L'utente deve poter scegliere il database da interrogare tra quelli presenti nel sistema & Obbligatorio & UC25 - Capitolato \\
		\hline
		RF26 & Visualizzazione di errore nel caso in cui l'utente non selezioni un database da interrogare & Desiderabile & UC26 \\
		\hline
		RF27 & L'utente deve poter inserire la richiesta in linguaggio naturale per interrogare il database & Obbligatorio & UC27 - Capitolato \\
		\hline
		RF28 & Visualizzazione di errore in caso di problemi con la frase inserita dall'utente & Desiderabile & UC28 \\
		\hline
		RF29 & Visualizzazione del prompt generato da fornire poi all'LLM & Obbligatorio & UC29 - Capitolato \\
		\hline
		RF30 & Visualizzazione di errore in caso di problemi con la generazione del prompt & Desiderabile & UC30 \\
		\hline
		RF31 & L'utente deve poter richiedere la generazione della query utilizzando un servizio LLM esterno al sistema & Opzionale & UC31 - Capitolato \\
		\hline
		RF32 & Visualizzazione di errore in caso di problemi con la generazione della query & Opzionale & UC32 - Capitolato \\
		\hline
		RF33 & Visualizzazione della query generata dal LLM con il prompt & Opzionale & UC33 - Capitolato \\
		\hline
		RF34 & Dare la possibilità all'utente di inserire la frase in linguaggio naturale trampite input vocale & Opzionale & Capitolato \\
		\hline
		RF35 & Verificare la correttezza della frase SQL prodotta & Opzionale & Capitolato \\
		\hline
		RF36 & Implementare la gestione di più basi di dati & Opzionale & Capitolato - UC5, UC25 \\
		\hline
		RF37 & Utilizzare modelli LLM alternativi a ChatGPT & Opzionale & Capitolato \\
		\hline
		RF38 & Avere a disposizione un modello addestrato appositamente per tradurre le frasi di interrogazione in italiano a SQL & Opzionale & Capitolato \\
		\hline
	\end{tabular}
\end{center}

\subsection{Requisiti di qualità}
\begin{center}
	\rowcolors{2}{gray!25}{white}
	\begin{tabular}{ |p{0.10\linewidth}|p{0.50\linewidth}|p{0.15\linewidth}|p{0.15\linewidth}| }
		\hline
		\textbf{Codice} & \textbf{Descrizione} & \textbf{Classificazione} & \textbf{Fonti} \\
		\hline
		% RQ& descrizione & classificazione & fonti \\
		% \hline
		RQ1 & Il prodotto deve essere sviluppato seguendo le Norme di Progetto definite & Obbligatorio & Norme di Progetto \\
		\hline
		RQ2 & Il codice sorgente deve essere presente su GitHub all'interno di una repository & Obbligatorio & Norme di Progetto \\
		\hline
		RQ3 & Viene fornita la documentazione dell'applicazione, presente all'interno della repository & Desiderabile & Norme di Progetto \\
		\hline
	\end{tabular}
\end{center}

\subsection{Requisiti di vincolo}
\begin{center}
	\rowcolors{2}{gray!25}{white}
	\begin{tabular}{ |p{0.10\linewidth}|p{0.50\linewidth}|p{0.15\linewidth}|p{0.15\linewidth}| }
		\hline
		\textbf{Codice} & \textbf{Descrizione} & \textbf{Classificazione} & \textbf{Fonti} \\
		\hline
		% RV& descrizione & classificazione & fonti \\
		% \hline
		RV1 & Sviluppo di un interfaccia tramite HTML e CSS & Opzionale & Verbale esterno \\
		\hline
		RV2 & Sviluppo dell'applicazione in Python & Opzionale & Capitolato \\
		\hline
		RV3 & Il prodotto deve essere in grado di analizzare un file strutturato & Obbligatorio & Capitolato \\
		\hline
	\end{tabular}
\end{center}

\subsection{Requisiti prestazionali}
Non sono stati individuati vincoli prestazionali.

\subsection{Tracciamento}
\subsubsection{Fonte - Requisiti}
\begin{center}
	\rowcolors{2}{gray!25}{white}
	\begin{tabular}{ |p{0.25\linewidth}|p{0.25\linewidth}| } 
		\hline
		\textbf{Fonte} & \textbf{Requisiti} \\
		\hline
		% fonte & requisiti \\
		% \hline
		Capitolato & RF21, RF25, RF27, RF29, RF31, RF32, RF33, RF34, RF35, RF36, RF37, RF38, RV2, RV3 \\
		\hline
		UC1 & RF1 \\
		\hline
		UC2 & RF2 \\
		\hline
		UC3 & RF3 \\
		\hline
		UC4 & RF4 \\
		\hline
		UC5 & RF5 \\
		\hline
		UC6 & RF6 \\
		\hline
		UC7 & RF7 \\
		\hline
		UC8 & RF8 \\
		\hline
		UC9 & RF9 \\
		\hline
		UC10 & RF10 \\
		\hline
		UC11 & RF11 \\
		\hline
		UC12 & RF12 \\
		\hline
		UC13 & RF13 \\
		\hline
		UC14 & RF14 \\
		\hline
		UC15 & RF15 \\
		\hline
		UC16 & RF16 \\
		\hline
		UC17 & RF17 \\
		\hline
		UC18 & RF18 \\
		\hline
		UC19 & RF19 \\
		\hline
		UC20 & RF20 \\
		\hline
		UC21 & RF21 \\
		\hline
		UC22 & RF22 \\
		\hline
		UC23 & RF23 \\
		\hline
		UC24 & RF24 \\
		\hline
		UC25 & RF25 \\
		\hline
		UC26 & RF26 \\
		\hline
		UC27 & RF27 \\
		\hline
		UC28 & RF28 \\
		\hline
		UC29 & RF29 \\
		\hline
		UC30 & RF30 \\
		\hline
		UC31 & RF31 \\
		\hline
		UC32 & RF32 \\
		\hline
		UC33 & RF33 \\
		\hline
		Norme di Progetto  & RQ1, RQ2, RQ3 \\
		\hline
		Verbali esterni  & RF1, RV1 \\
		\hline
	\end{tabular}
\end{center}

\subsubsection{Requisito - Fonti}
\begin{center}
	\rowcolors{2}{gray!25}{white}
	\begin{tabular}{ |p{0.25\linewidth}|p{0.25\linewidth}| } 
		\hline
		\textbf{Requisito} & \textbf{Fonti} \\
		\hline
		% requisito & fonti \\
		% \hline
		RF1 & UC1, Verbale esterno \\
		\hline
		RF2 & UC2 \\
		\hline
		RF3 & UC3 \\
		\hline
		RF4 & UC4 \\
		\hline
		RF5 & UC5 \\
		\hline
		RF6 & UC6 \\
		\hline
		RF7 & UC7 \\
		\hline
		RF8 & UC8 \\
		\hline
		RF9 & UC9 \\
		\hline
		RF10 & UC10 \\
		\hline
		RF11 & UC11 \\
		\hline
		RF12 & UC12 \\
		\hline
		RF13 & UC13 \\
		\hline
		RF14 & UC14 \\
		\hline
		RF15 & UC15 \\
		\hline
		RF16 & UC16 \\
		\hline
		RF17 & UC17 \\
		\hline
		RF18 & UC18 \\
		\hline
		RF19 & UC19 \\
		\hline
		RF20 & UC20 \\
		\hline
		RF21 & UC21, Capitolato \\
		\hline
		RF22 & UC22 \\
		\hline
		RF23 & UC23 \\
		\hline
		RF24 & UC24 \\
		\hline
		RF25 & UC25, Capitolato \\
		\hline
		RF26 & UC26 \\
		\hline
		RF27 & UC27, Capitolato \\
		\hline
		RF28 & UC28 \\
		\hline
		RF29 & UC29, Capitolato \\
		\hline
		RF30 & UC30 \\
		\hline
		RF31 & UC31, Capitolato \\
		\hline
		RF32 & UC32, Capitolato \\
		\hline
		RF33 & UC33, Capitolato \\
		\hline
		RF34 & Capitolato \\
		\hline
		RF35 & Capitolato \\
		\hline
		RF36 & UC5, UC25, Capitolato \\
		\hline
		RF37 & Capitolato \\
		\hline
		RF38 & Capitolato \\
		\hline
		RQ1 & Norme di Progetto \\
		\hline
		RQ2 & Norme di Progetto \\
		\hline
		RQ3 & Norme di Progetto \\
		\hline
		RV1 & Verbale esterno \\
		\hline
		RV2 & Capitolato \\
		\hline
		RV3 & Capitolato \\
		\hline
	\end{tabular}
\end{center}

\subsection{Riepilogo}
\begin{center}

	\rowcolors{2}{gray!25}{white}
	\begin{tabular}{ |c|c|c|c|c| } 
		\hline
		\textbf{Tipologia} & \textbf{Obbligatorio} & \textbf{Desiderabile} & \textbf{Opzionale} & \textbf{Totale}\\
		\hline
		Funzionale & - & - & - & - \\
		\hline
		Di Qualità & - & - & - & - \\
		\hline
		Di Vincolo & - & - & - & - \\
		\hline
		Prestazionale & - & - & - & - \\
		\hline
	\end{tabular}

\end{center}



\end{document}