\section{Casi d'uso}
% TEMPLATE
%   \subsection{UC - NomeUseCase}
%   \label{sec:UC}
%   \includegraphics[]{diagramma_UML}
%   \begin{itemize}
	%       \item \textbf{Descrizione:} 
	%       \item \textbf{Attori:} 
	%       \item \textbf{Precondizioni:} 
	%       \item \textbf{Postcondizioni:} 
	%       \item \textbf{Scenario principale:} 
	%       \item \textbf{Generalizzazioni:} 
	%       \item \textbf{Estensioni:} 
	%   \end{itemize}

%   \hyperref[sec:UC]{\textbf{UC}}

% Link ad altre sezioni usando hyperref - utile per linkare generalizzazioni e estensioni
% La label fa da segnalibro a dove dovrà andare il link
%   \label{sec:nomeSezione}
% Link sul quale cliccare per andare alla label
%   \hyperref[sec:nomeSezione]{testo}
% END TEMPLATE


\iffalse

\subsection{UC8xx - Visualizzazione errore: caricamento struttura (formato non accettato)??}
\label{sec:UC8xxx}
%\includegraphics[]{diagramma_UML}
\begin{itemize}
	\item \textbf{Descrizione:} l’amministratore carica un file con un formato errato;
	\item \textbf{Attori:} amministratore;
	\item \textbf{Precondizioni:} 
	\begin{itemize}
		\item L’amministratore ha selezionato un file con formato diverso dal ???(da decidere);
	\end{itemize}
	\item \textbf{Postcondizioni:} 
	\begin{itemize}
		\item Il programma visualizza un messaggio di errore;
		\item Il file non viene elaborato;
	\end{itemize}
	\item \textbf{Scenario principale:} 
	\begin{itemize}
		\item L’amministratore si trova nel pannello amministrativo;
		\item L’amministratore sceglie un file in un formato diverso dal ??? tra i suoi file locali e lo carica nel programma;
		\item Il programma visualizza un messaggio di errore.
	\end{itemize}
\end{itemize}

\subsection{UC9xx - Modifica struttura database}
\label{sec:UC9xxx}
%\includegraphics[]{diagramma_UML}
\begin{itemize}
	\item \textbf{Descrizione:}  l’amministratore modifica una delle strutture database già presenti;
	\item \textbf{Attori:} amministratore;
	\item \textbf{Precondizioni:} 
	\begin{itemize}
		\item L’amministratore si trova nel pannello amministrativo;
		\item ???
	\end{itemize}
	\item \textbf{Postcondizioni:} 
	\begin{itemize}
		\item La struttura è stata modificata e salvata.
	\end{itemize}
	\item \textbf{Scenario principale:} 
	\begin{itemize}
		\item ???
	\end{itemize}
	\item \textbf{Generalizzazioni:} 
	\begin{itemize}
		\item ???
	\end{itemize}
\end{itemize}

\subsection{UC10xx - Eliminazione struttura database}
\label{sec:10xxx}
%\includegraphics[]{diagramma_UML}
\begin{itemize}
	\item \textbf{Descrizione:}  l’amministratore elimina una delle strutture database già presenti;
	\item \textbf{Attori:} amministratore;
	\item \textbf{Precondizioni:} 
	\begin{itemize}
		\item L’amministratore si trova nel pannello amministrativo;
		\item ???
	\end{itemize}
	\item \textbf{Postcondizioni:} 
	\begin{itemize}
		\item La struttura è stata eliminata.
	\end{itemize}
	\item \textbf{Scenario principale:} 
	\begin{itemize}
		\item ???
	\end{itemize}
	\item \textbf{Generalizzazioni:} 
	\begin{itemize}
		\item ???
	\end{itemize}
\end{itemize}

\subsection{UC - }
\label{sec:UC}

\fi

\subsection{UC1 - Login}
\label{sec:UC1}
%\includegraphics[]{diagramma_UML}
\begin{itemize}
	\item \textbf{Descrizione:} L’amministratore accede al pannello amministrativo con le sue credenziali;
	\item \textbf{Attori:} amministratore;
	\item \textbf{Precondizioni:} 
	\begin{itemize}
		\item L’amministratore possiede delle credenziali di accesso valide;
		\item L’amministratore non ha già effettuato l’accesso;
	\end{itemize}
	\item \textbf{Postcondizioni:} 
	\begin{itemize}
		\item L’utente Amministratore viene riconosciuto dal sistema;
	\end{itemize}
	\item \textbf{Scenario principale:} 
	\begin{itemize}
		\item L’ amministratore inserisce il proprio nome utente nel form di accesso (\hyperref[sec:UC1.1]{\textbf{UC1.1}});
		\item L’ amministratore inserisce la propria password nel form di accesso (\hyperref[sec:UC1.2]{\textbf{UC1.2}});
		\item Il sistema verifica che le credenziali ricevute siano corrette. 
	\end{itemize}
	\item \textbf{Generalizzazioni:} 
	\begin{itemize}
		\item \hyperref[sec:UC1.1]{\textbf{UC1.1}} - Inserimento nome utente
		\item \hyperref[sec:UC1.2]{\textbf{UC1.2}} - Inserimento password
	\end{itemize}
	\item \textbf{Estensioni:} Nel caso le credenziali non siano corrette:
	\begin{itemize}
		\item viene mostrato un errore - \hyperref[sec:UC5]{\textbf{UC5}}
	\end{itemize}
\end{itemize}

\subsubsection{UC1.1 - Inserimento nome utente}
\label{sec:UC1.1}
%\includegraphics[]{diagramma_UML}
\begin{itemize}
	\item \textbf{Descrizione:} L’amministratore inserisce il proprio nome utente;
	\item \textbf{Attori:} amministratore;
	\item \textbf{Precondizioni:} 
	\begin{itemize}
		\item L’amministratore possiede le credenziali di accesso;
		\item L’amministratore non ha già effettuato l’accesso;
		\item L’amministratore sta effettuando il login (\hyperref[sec:UC1]{\textbf{UC1}})
	\end{itemize}
	\item \textbf{Postcondizioni:} 
	\begin{itemize}
		\item L’amministratore ha inserito correttamente il proprio nome utente;
	\end{itemize}
	\item \textbf{Scenario principale:} 
	\begin{itemize}
		\item L’amministratore inserisce il proprio nome utente nel form di accesso.
	\end{itemize}
\end{itemize}

\subsubsection{UC1.2 - Inserimento password}
\label{sec:UC1.2}
%\includegraphics[]{diagramma_UML}
\begin{itemize}
	\item \textbf{Descrizione:} L’amministratore inserisce la propria password;
	\item \textbf{Attori:} amministratore;
	\item \textbf{Precondizioni:} 
	\begin{itemize}
		\item L’amministratore possiede le credenziali di accesso;
		\item L’amministratore non ha già effettuato l’accesso;
		\item L’amministratore sta effettuando il login (\hyperref[sec:UC1]{\textbf{UC1}})
	\end{itemize}
	\item \textbf{Postcondizioni:} 
	\begin{itemize}
		\item L’amministratore ha inserito correttamente la propria password;
	\end{itemize}
	\item \textbf{Scenario principale:} 
	\begin{itemize}
		\item L’amministratore inserisce la propria password nel form di accesso.
	\end{itemize}
\end{itemize}

\subsection{UC2 - Credenziali login errate}
\label{sec:UC2}
%\includegraphics[]{diagramma_UML}
\begin{itemize}
	\item \textbf{Descrizione:} L’amministratore visualizza un errore di autenticazione;
	\item \textbf{Attori:} amministratore;
	\item \textbf{Precondizioni:} 
	\begin{itemize}
		\item L’amministratore possiede le credenziali di accesso;
		\item L’amministratore non ha già effettuato l’accesso;
		\item L’amministratore sta effettuando il login (\hyperref[sec:UC1]{\textbf{UC1}});
	\end{itemize}
	\item \textbf{Postcondizioni:}
	\begin{itemize}
		\item L’amministratore non viene riconosciuto dal sistema e deve reinserire le proprie credenziali;
	\end{itemize}
	\item \textbf{Scenario principale:} 
	\begin{itemize}
		\item L’amministratore inserisce il proprio nome utente nel form di accesso (\hyperref[sec:UC1.1]{\textbf{UC1.1}});
		\item L’amministratore inserisce la propria password nel form di accesso (\hyperref[sec:UC1.2]{\textbf{UC1.2}});
		\item Il sistema verifica le credenziali ricevute siano corrette;
		\item Il sistema visualizza un messaggio di errore per le credenziali inserite.
	\end{itemize}
\end{itemize}

\subsection{UC3 - Creazione database}
\label{sec:UC3}
%\includegraphics[]{diagramma_UML}
\begin{itemize}
	\item \textbf{Descrizione:} l’amministratore vuole aggiungere la struttura di un database da poter interrogare;
	\item \textbf{Attori:} amministratore;
	\item \textbf{Precondizioni:} 
	\begin{itemize}
		\item L’amministratore ha effettuato il login (\hyperref[sec:UC1]{\textbf{UC1}});
		\item L’amministratore si trova nel pannello amministrativo;
	\end{itemize}
	\item \textbf{Postcondizioni:} 
	\begin{itemize}
		\item La struttura del database viene salvata nel programma;
	\end{itemize}
	\item \textbf{Scenario principale:} 
	\begin{itemize}
		\item L’amministratore inserisce il nome e la descrizione del database;
	\end{itemize}
	\item \textbf{Generalizzazioni:} 
	\begin{itemize}
		\item \hyperref[sec:UC3.1]{\textbf{UC3.1}} - Inserimento nome DB
		\item \hyperref[sec:UC3.2]{\textbf{UC3.2}} - Inserimento descrizione DB
	\end{itemize}
	\item \textbf{Estensioni:} nel caso in cui venga inserito un nome già esistente:
	\begin{itemize}
		\item \hyperref[sec:UC4]{\textbf{UC4}} - Errore: nome DB già presente
	\end{itemize}
\end{itemize}

\subsubsection{UC3.1 - Inserimento nome DB}
\label{sec:UC3.1}
%\includegraphics[]{diagramma_UML}
\begin{itemize}
	\item \textbf{Descrizione:} l’amministratore deve inserire il nome del nuovo database da aggiungere;
	\item \textbf{Attori:} amministratore;
	\item \textbf{Precondizioni:} 
	\begin{itemize}
		\item L’amministratore ha effettuato il login (\hyperref[sec:UC1]{\textbf{UC1}});
		\item L’amministratore si trova nel pannello amministrativo;
		\item L’amministratore sta creando un nuovo database (\hyperref[sec:UC3]{\textbf{UC3}});
	\end{itemize}
	\item \textbf{Postcondizioni:} 
	\begin{itemize}
		\item L'amministratore ha inserito correttamente il nome del nuovo database;
	\end{itemize}
	\item \textbf{Scenario principale:} 
	\begin{itemize}
		\item L’amministratore inserisce il nome del database nel form di creazione;
	\end{itemize}
\end{itemize}

\subsubsection{UC3.2 - Inserimento descrizione DB}
\label{sec:UC3.2}
%\includegraphics[]{diagramma_UML}
\begin{itemize}
	\item \textbf{Descrizione:} l’amministratore deve inserire la descrizione del nuovo database da aggiungere;
	\item \textbf{Attori:} amministratore;
	\item \textbf{Precondizioni:} 
	\begin{itemize}
		\item L’amministratore ha effettuato il login (\hyperref[sec:UC1]{\textbf{UC1}});
		\item L’amministratore si trova nel pannello amministrativo;
		\item L’amministratore sta creando un nuovo database (\hyperref[sec:UC3]{\textbf{UC3}});
	\end{itemize}
	\item \textbf{Postcondizioni:} 
	\begin{itemize}
		\item L'amministratore ha inserito correttamente la descrizione del nuovo database;
	\end{itemize}
	\item \textbf{Scenario principale:} 
	\begin{itemize}
		\item L’amministratore inserisce la descrizione del database nel form di creazione;
	\end{itemize}
\end{itemize}

\subsection{UC4 - Errore: nome DB già presente}
\label{sec:UC4}
%\includegraphics[]{diagramma_UML}
\begin{itemize}
	\item \textbf{Descrizione:} L’amministratore visualizza un errore di creazione del database;
	\item \textbf{Attori:} amministratore;
	\item \textbf{Precondizioni:} 
	\begin{itemize}
		\item L’amministratore ha effettuato il login (\hyperref[sec:UC1]{\textbf{UC1}});
		\item L’amministratore si trova nel pannello amministrativo;
		\item L’amministratore sta creando un nuovo database (\hyperref[sec:UC3]{\textbf{UC3}});
	\end{itemize}
	\item \textbf{Postcondizioni:} 
	\begin{itemize}
		\item La struttura del database non viene salvata nel programma e visualizza un messaggio di errore;
	\end{itemize}
	\item \textbf{Scenario principale:} 
	\begin{itemize}
		\item L’amministratore inserisce il nome del database nel form di creazione (\hyperref[sec:UC3.1]{\textbf{UC3.1}});
		\item L’amministratore inserisce la descrizione del database nel form di creazione (\hyperref[sec:UC3.2]{\textbf{UC3.2}});
		\item Il sistema verifica che non esista già un database con lo stesso nome;
		\item Il sistema visualizza un messaggio di errore per il nome inserito.
	\end{itemize}
\end{itemize}

\subsection{UC5 - Visualizzazione DB}
\label{sec:UC5}
%\includegraphics[]{diagramma_UML}
\begin{itemize}
	\item \textbf{Descrizione:} l’amministratore visualizza tutte le strutture database disponibili;
	\item \textbf{Attori:} amministratore;
	\item \textbf{Precondizioni:} 
	\begin{itemize}
		\item L’amministratore ha effettuato il login (\hyperref[sec:UC1]{\textbf{UC1}});
		\item L’amministratore si trova nel pannello amministrativo;
	\end{itemize}
	\item \textbf{Postcondizioni:} 
	\begin{itemize}
		\item L'amministrazione naviga il pannello amministrativo e può vedere nome e descrizione dei database presenti;
	\end{itemize}
	\item \textbf{Scenario principale:} 
	\begin{itemize}
		\item Il programma visualizza la lista dei database presenti, con la possibilità di modificarli, visualizzarli o eliminarli; ???
	\end{itemize}
\end{itemize}

\subsection{UC6 - Modifica DB}
\label{sec:UC6}

\subsection{UC7 - Elimina DB}
\label{sec:UC7}
%\includegraphics[]{diagramma_UML}
\begin{itemize}
	\item \textbf{Descrizione:} l’amministratore elimina la struttura database selezionata;
	\item \textbf{Attori:} amministratore;
	\item \textbf{Precondizioni:} 
	\begin{itemize}
		\item L’amministratore ha effettuato il login (\hyperref[sec:UC1]{\textbf{UC1}});
		\item L’amministratore si trova nel pannello amministrativo;
		\item L’amministratore sta visualizzando la lista dei database; ???
	\end{itemize}
	\item \textbf{Postcondizioni:} 
	\begin{itemize}
		\item La struttura database selezionata viene eliminata dal sistema;
	\end{itemize}
	\item \textbf{Scenario principale:} 
	\begin{itemize}
		\item L'amministratore sta visualizzando la lista dei database presenti nel sistema (\hyperref[sec:UC5]{\textbf{UC5}});
		\item L'amministratore seleziona il database da eliminare usando il pulsante di eliminazione apposito;
		\item Il sistema visualizza un messaggio di conferma dell'eliminazione;
		\item Se l'amministrazione conferma l'eliminazione, il database e le tabelle collegate verranno rimossi dal sistema e verrà visualizzato un messaggio di avvenuta eliminazione.
	\end{itemize}
\end{itemize}

\subsection{UC8 - Creazione tabella DB}
\label{sec:UC8}
%\includegraphics[]{diagramma_UML}
\begin{itemize}
	\item \textbf{Descrizione:} l’amministratore vuole aggiungere una tabella alla struttura del database da interrogare;
	\item \textbf{Attori:} amministratore;
	\item \textbf{Precondizioni:} 
	\begin{itemize}
		\item L’amministratore ha effettuato il login (\hyperref[sec:UC1]{\textbf{UC1}});
		\item L’amministratore si trova nel pannello amministrativo;
		\item L’amministratore si trova nella sezione di creazione di una nuova tabella;
	\end{itemize}
	\item \textbf{Postcondizioni:} 
	\begin{itemize}
		\item La tabella viene aggiunta alla struttura del database;
	\end{itemize}
	\item \textbf{Scenario principale:} 
	\begin{itemize}
		\item L’amministratore inserisce il nome, i sinonimi del nome e la descrizione della tabella;
	\end{itemize}
	\item \textbf{Generalizzazioni:} 
	\begin{itemize}
		\item \hyperref[sec:UC8.1]{\textbf{UC8.1}} - Inserimento nome tabella
		\item \hyperref[sec:UC8.2]{\textbf{UC8.2}} - Inserimento sinonimi tabella
		\item \hyperref[sec:UC8.3]{\textbf{UC8.3}} - Inserimento descrizione tabella
	\end{itemize}
	\item \textbf{Estensioni:} nel caso in cui non vengano inseriti i sinonimi del nome della tabella, o il nome esisti già:
	\begin{itemize}
		\item \hyperref[sec:UC9]{\textbf{UC9}} - Errore nella creazione della tabella
	\end{itemize}
\end{itemize}

\subsubsection{UC8.1 - Inserimento nome tabella}
\label{sec:UC8.1}
%\includegraphics[]{diagramma_UML}
\begin{itemize}
	\item \textbf{Descrizione:} l’amministratore inserisce il nome della tabella da creare;
	\item \textbf{Attori:} amministratore;
	\item \textbf{Precondizioni:} 
	\begin{itemize}
		\item L’amministratore ha effettuato il login (\hyperref[sec:UC1]{\textbf{UC1}});
		\item L’amministratore sta creando una nuova tabella (\hyperref[sec:UC1]{\textbf{UC8}});
	\end{itemize}
	\item \textbf{Postcondizioni:} 
	\begin{itemize}
		\item Il nome della tabella viene inserito nel form;
	\end{itemize}
	\item \textbf{Scenario principale:} 
	\begin{itemize}
		\item L’amministratore inserisce il nome della tabella nell'apposito form di creazione;
	\end{itemize}
\end{itemize}

\subsubsection{UC8.2 - Inserimento sinonimi tabella}
\label{sec:UC8.2}
%\includegraphics[]{diagramma_UML}
\begin{itemize}
	\item \textbf{Descrizione:} l’amministratore inserisce i sinonimi associati al nome della tabella da creare;
	\item \textbf{Attori:} amministratore;
	\item \textbf{Precondizioni:} 
	\begin{itemize}
		\item L’amministratore ha effettuato il login (\hyperref[sec:UC1]{\textbf{UC1}});
		\item L’amministratore sta creando una nuova tabella (\hyperref[sec:UC1]{\textbf{UC8}});
	\end{itemize}
	\item \textbf{Postcondizioni:} 
	\begin{itemize}
		\item I sinonimi del nome della tabella vengono inseriti nel form;
	\end{itemize}
	\item \textbf{Scenario principale:} 
	\begin{itemize}
		\item L’amministratore inserisce i sinonimi del nome della tabella nell'apposito form di creazione;
	\end{itemize}
\end{itemize}

\subsubsection{UC8.3 - Inserimento descrizione tabella}
\label{sec:UC8.3}
%\includegraphics[]{diagramma_UML}
\begin{itemize}
	\item \textbf{Descrizione:} l’amministratore inserisce la descrizione della tabella da creare;
	\item \textbf{Attori:} amministratore;
	\item \textbf{Precondizioni:} 
	\begin{itemize}
		\item L’amministratore ha effettuato il login (\hyperref[sec:UC1]{\textbf{UC1}});
		\item L’amministratore sta creando una nuova tabella (\hyperref[sec:UC1]{\textbf{UC8}});
	\end{itemize}
	\item \textbf{Postcondizioni:} 
	\begin{itemize}
		\item La descrizione della tabella viene inserita nel form;
	\end{itemize}
	\item \textbf{Scenario principale:} 
	\begin{itemize}
		\item L’amministratore inserisce la descrizione della tabella nell'apposito form di creazione;
	\end{itemize}
\end{itemize}

\subsection{UC9 - Errore nella creazione della tabella}
\label{sec:UC9}
%\includegraphics[]{diagramma_UML}
\begin{itemize}
	\item \textbf{Descrizione:} L’amministratore visualizza un errore di creazione della tabella;
	\item \textbf{Attori:} amministratore;
	\item \textbf{Precondizioni:} 
	\begin{itemize}
		\item L’amministratore ha effettuato il login (\hyperref[sec:UC1]{\textbf{UC1}});
		\item L’amministratore sta creando una nuova tabella (\hyperref[sec:UC1]{\textbf{UC8}});
	\end{itemize}
	\item \textbf{Postcondizioni:} 
	\begin{itemize}
		\item La tabella non viene creata e il programma visualizza un messaggio di errore;
	\end{itemize}
	\item \textbf{Scenario principale:} 
	\begin{itemize}
		\item L’amministratore inserisce il nome della tabella nel form di creazione (\hyperref[sec:UC8.1]{\textbf{UC8.1}});
		\item L’amministratore inserisce i sinonimi del nome della tabella nel form di creazione (\hyperref[sec:UC8.2]{\textbf{UC8.2}});
		\item L’amministratore inserisce la descrizione della tabella nel form di creazione (\hyperref[sec:UC8.3]{\textbf{UC8.3}});
		\item Il sistema verifica che non esista già una tabella con lo stesso nome e che vengano inseriti sinonimi e descrizione della tabella;
		\item Il sistema visualizza il messaggio di errore opportuno.
	\end{itemize}
\end{itemize}

\subsubsection{UC9.1 - Errore nome tabella già presente}
\label{sec:UC9.1}
%\includegraphics[]{diagramma_UML}
\begin{itemize}
	\item \textbf{Descrizione:} L’amministratore visualizza un errore relativo al nome della tabella;
	\item \textbf{Attori:} amministratore;
	\item \textbf{Precondizioni:} 
	\begin{itemize}
		\item L’amministratore ha effettuato il login (\hyperref[sec:UC1]{\textbf{UC1}});
		\item L’amministratore sta creando una nuova tabella (\hyperref[sec:UC1]{\textbf{UC8}});
	\end{itemize}
	\item \textbf{Postcondizioni:} 
	\begin{itemize}
		\item La tabella non viene creata e il programma visualizza un messaggio di errore;
	\end{itemize}
	\item \textbf{Scenario principale:} 
	\begin{itemize}
		\item Il sistema verifica che non esista già una tabella con lo stesso nome;
		\item Il sistema visualizza il messaggio di errore per il nome inserito.
	\end{itemize}
\end{itemize}

\subsubsection{UC9.2 - Errore sinonimi non inseriti}
\label{sec:UC9.2}
%\includegraphics[]{diagramma_UML}
\begin{itemize}
	\item \textbf{Descrizione:} L’amministratore visualizza un errore relativo ai sinonimi del nome della tabella;
	\item \textbf{Attori:} amministratore;
	\item \textbf{Precondizioni:} 
	\begin{itemize}
		\item L’amministratore ha effettuato il login (\hyperref[sec:UC1]{\textbf{UC1}});
		\item L’amministratore sta creando una nuova tabella (\hyperref[sec:UC1]{\textbf{UC8}});
	\end{itemize}
	\item \textbf{Postcondizioni:} 
	\begin{itemize}
		\item La tabella non viene creata e il programma visualizza un messaggio di errore;
	\end{itemize}
	\item \textbf{Scenario principale:} 
	\begin{itemize}
		\item Il sistema verifica che il campo relativo ai sinonimi del nome della tabella non sia vuoto;
		\item Il sistema visualizza il messaggio di errore per il campo sinonimi vuoto.
	\end{itemize}
\end{itemize}

\subsubsection{UC9.3 - Errore descrizione non inserita}
\label{sec:UC9.3}
%\includegraphics[]{diagramma_UML}
\begin{itemize}
	\item \textbf{Descrizione:} L’amministratore visualizza un errore relativo alla descrizione della tabella;
	\item \textbf{Attori:} amministratore;
	\item \textbf{Precondizioni:} 
	\begin{itemize}
		\item L’amministratore ha effettuato il login (\hyperref[sec:UC1]{\textbf{UC1}});
		\item L’amministratore sta creando una nuova tabella (\hyperref[sec:UC1]{\textbf{UC8}});
	\end{itemize}
	\item \textbf{Postcondizioni:} 
	\begin{itemize}
		\item La tabella non viene creata e il programma visualizza un messaggio di errore;
	\end{itemize}
	\item \textbf{Scenario principale:} 
	\begin{itemize}
		\item Il sistema verifica che il campo relativo ai sinonimi del nome della tabella non sia vuoto;
		\item Il sistema visualizza il messaggio di errore per il campo descrizione vuoto.
	\end{itemize}
\end{itemize}

\subsection{UC10 - Modifica della tabella}
\label{sec:UC10}

\subsection{UC11 - Visualizzazione della tabella}
\label{sec:UC11}

\subsection{UC12 - Eliminazione della tabella}
\label{sec:UC12}

\subsection{UC13 - Creazione campo tabella}
\label{sec:UC13}
%\includegraphics[]{diagramma_UML}
\begin{itemize}
	\item \textbf{Descrizione:} l’amministratore vuole aggiungere i campi che compongono le tabelle alla struttura del database;
	\item \textbf{Attori:} amministratore;
	\item \textbf{Precondizioni:} 
	\begin{itemize}
		\item L’amministratore ha effettuato il login (\hyperref[sec:UC1]{\textbf{UC1}});
		\item L’amministratore si trova nel pannello amministrativo;
		\item L’amministratore si trova nella sezione di inserimento della struttura database;
		\item L’amministratore sta inserendo i campi che compongono la tabella;
	\end{itemize}
	\item \textbf{Postcondizioni:} 
	\begin{itemize}
		\item I campi vengono aggiunti alla tabella;
	\end{itemize}
	\item \textbf{Scenario principale:} 
	\begin{itemize}
		\item L’amministratore inserisce il nome del campo, ne seleziona il tipo e inserisce i sinonimi;
	\end{itemize}
	\item \textbf{Generalizzazioni:} 
	\begin{itemize}
		\item \hyperref[sec:UC13.1]{\textbf{UC13.1}} - Inserimento nome campo
		\item \hyperref[sec:UC13.2]{\textbf{UC13.2}} - Inserimento tipo campo
		\item \hyperref[sec:UC13.3]{\textbf{UC13.3}} - Inserimento sinonimi campo
	\end{itemize}
	\item \textbf{Estensioni:} nel caso in cui il nome inserito sia già esistente o non sia stato selezionato il tipo o inseriti i sinonimi:
	\begin{itemize}
		\item \hyperref[sec:UC14]{\textbf{UC14}} - Errore creazione campo
	\end{itemize}
\end{itemize}

\subsubsection{UC13.1 - Inserimento nome campo}
\label{sec:UC13.1}
%\includegraphics[]{diagramma_UML}
\begin{itemize}
	\item \textbf{Descrizione:} l’amministratore vuole inserire il nome del campo da inserire nella tabella;
	\item \textbf{Attori:} amministratore;
	\item \textbf{Precondizioni:} 
	\begin{itemize}
		\item L’amministratore ha effettuato il login (\hyperref[sec:UC1]{\textbf{UC1}});
		\item L’amministratore si trova nel pannello amministrativo;
		\item L’amministratore si trova nella sezione di inserimento della struttura database;
		\item L’amministratore sta inserendo i campi che compongono la tabella;
	\end{itemize}
	\item \textbf{Postcondizioni:} 
	\begin{itemize}
		\item Il nome del campo viene inserito;
	\end{itemize}
	\item \textbf{Scenario principale:} 
	\begin{itemize}
		\item L’amministratoer inserisce il nome del campo nella casella di testo dedicata;
	\end{itemize}
	\item \textbf{Estensioni:} nel caso in cui il nome inserito sia già esistente:
	\begin{itemize}
		\item \hyperref[sec:UC14.1]{\textbf{UC14.1}} - Errore nome campo già esistente
	\end{itemize}
\end{itemize}

\subsubsection{UC13.2 - Inserimento tipo campo}
\label{sec:UC13.2}
%\includegraphics[]{diagramma_UML}
\begin{itemize}
	\item \textbf{Descrizione:} l’amministratore vuole selezionare il tipo del campo da inserire nella tabella;
	\item \textbf{Attori:} amministratore;
	\item \textbf{Precondizioni:} 
	\begin{itemize}
		\item L’amministratore ha effettuato il login (\hyperref[sec:UC1]{\textbf{UC1}});
		\item L’amministratore si trova nel pannello amministrativo;
		\item L’amministratore si trova nella sezione di inserimento della struttura database;
		\item L’amministratore sta inserendo i campi che compongono la tabella;
	\end{itemize}
	\item \textbf{Postcondizioni:} 
	\begin{itemize}
		\item Il tipo del campo viene selezionato;
	\end{itemize}
	\item \textbf{Scenario principale:} 
	\begin{itemize}
		\item L’amministratore sceglie il tipo di campo, selezionandolo dalle scelte possibili;
	\end{itemize}
	\item \textbf{Estensioni:} nel caso in cui il tipo non venga selezionato:
	\begin{itemize}
		\item \hyperref[sec:UC14.1]{\textbf{UC14.1}} - Errore tipo campo non selezionato
	\end{itemize}
\end{itemize}

\subsubsection{UC13.3 - Inserimento sinonimi campo}
\label{sec:UC13.3}

\subsection{UC14 - Errore creazione campo}
\label{sec:UC14}

\subsubsection{UC14.1 - Errore nome campo già esistente}
\label{sec:UC14.1}

\subsubsection{UC14.2 - Errore tipo campo non selezionato}
\label{sec:UC14.2}

\subsubsection{UC14.3 - Errore mancato inserimento sinonimi campo}
\label{sec:UC14.3}

\subsection{UC15 - Visualizzazione campo tabella}
\label{sec:UC15}

\subsection{UC16 - Modifica campo tabella}
\label{sec:UC16}

\subsection{UC17 - Eliminazione campo tabella}
\label{sec:UC17}

\subsection{UC18 - Caricamento struttura DB tramite file}
\label{sec:UC18}

\subsection{UC19 - Errore caricamento file(formato non accettato)}
\label{sec:UC19}

\subsection{UC20 - Logout}
\label{sec:UC20}
%\includegraphics[]{diagramma_UML}
\begin{itemize}
	\item \textbf{Descrizione: l'amministratore vuole fare il logout dall'area amministrativa} 
	\item \textbf{Attori:} amministratore;
	\item \textbf{Precondizioni:} 
	\begin{itemize}
		\item L’amministratore ha effettuato il login (\hyperref[sec:UC1]{\textbf{UC1}});
	\end{itemize}
	\item \textbf{Postcondizioni:} 
	\begin{itemize}
		\item Viene visualizzata la pagina iniziale;
	\end{itemize}
	\item \textbf{Scenario principale:} 
	\begin{itemize}
		\item L’amministratore clicca il pulsante di logout;
		\item L’amministratore viene reindirizzato alla pagina iniziale.
	\end{itemize}
	\item \textbf{Estensioni:} nel caso in cui l'utente non fosse loggato:
	\begin{itemize}
		\item Viene visualizzato un errore - \hyperref[sec:UC21]{\textbf{UC21}}.
	\end{itemize}
\end{itemize}

\subsection{UC21 - Errore logout (not logged in)}
\label{sec:UC21}

\subsection{UC22 - Selezione database da interrogare}
\label{sec:UC22}
%\includegraphics[]{diagramma_UML}
\begin{itemize}
	\item \textbf{Descrizione:} l’utente seleziona il database che vuole interrogare da una lista;
	\item \textbf{Attori:} utente;
	\item \textbf{Precondizioni:}
	\begin{itemize}
		\item Sono presenti una o più strutture database da poter selezionare;
	\end{itemize}
	\item \textbf{Postcondizioni:}
	\begin{itemize}
		\item L’utente ha selezionato una struttura database e può procedere con \hyperref[sec:UC2]{\textbf{UC2}};
	\end{itemize}
	\item \textbf{Scenario principale:}
	\begin{itemize}
		\item L’utente ha il programma aperto;
		\item L’utente seleziona uno dei database presenti nella lista.
	\end{itemize}
\end{itemize}

\subsection{UC23 - Errore DB non selezionato}
\label{sec:UC23}

\subsection{UC24 - Inserimento frase in linguaggio naturale}
\label{sec:UC24}
%\includegraphics[]{diagramma_UML}
\begin{itemize}
	\item \textbf{Descrizione:} L’utente scrive una frase in linguaggio naturale nella casella di testo disponibile nell’interfaccia e ne conferma l’inserimento;
	\item \textbf{Attori:} utente;
	\item \textbf{Precondizioni:} 
	\begin{itemize}
		\item L’utente ha selezionato un file di struttura Database (\hyperref[sec:UC22]{\textbf{UC22}});
	\end{itemize}
	\item \textbf{Postcondizioni:} 
	\begin{itemize}
		\item L’utente riceve un prompt per la creazione della query richiesta (\hyperref[sec:UC26]{\textbf{UC26}});
	\end{itemize}
	\item \textbf{Scenario principale:} 
	\begin{itemize}
		\item L’utente ha il programma aperto;
		\item L’utente seleziona la casella di testo;
		\item L’utente scrive la frase per interrogare il database;
		\item L’utente clicca il pulsante apposito per ottenere il prompt.
	\end{itemize}
	\item \textbf{Estensioni:} nel caso in cui venga inserita una frase non inerente al database, o non comprensibile:
	\begin{itemize}
		\item Viene visualizzato un errore - \hyperref[sec:UC25]{\textbf{UC25}}.
	\end{itemize}
\end{itemize}

\subsection{UC25 - Errore frase naturale}
\label{sec:UC25}

\subsubsection{UC25.1 - Errore frase non inserita}
\label{sec:UC25.1}

\subsubsection{UC25.2 - Errore frase non compresa}
\label{sec:UC25.2}

\subsubsection{UC25.3 - Errore frase non inerente}
\label{sec:UC25.3}
%\includegraphics[]{diagramma_UML}
\begin{itemize}
	\item \textbf{Descrizione:} l’utente inserisce una frase in linguaggio naturale non interpretabile dal sistema come inerente al database;
	\item \textbf{Attori:} utente;
	\item \textbf{Precondizioni:} 
	\begin{itemize}
		\item L’utente ha selezionato un file di struttura Database (\hyperref[sec:UC22]{\textbf{UC22}});
		\item L’utente ha scritto una frase in linguaggio naturale nella casella di testo apposita e ne ha confermato l’inserimento (\hyperref[sec:UC24]{\textbf{UC24}});
	\end{itemize}
	\item \textbf{Postcondizioni:} 
	\begin{itemize}
		\item Il programma visualizza un messaggio di errore;
		\item Il prompt non viene generato;
	\end{itemize}
	\item \textbf{Scenario principale:} 
	\begin{itemize}
		\item L’utente ha il programma aperto;
		\item L’utente ha selezionato il database(\hyperref[sec:UC22]{\textbf{UC22}}) e inserito la frase(\hyperref[sec:UC24]{\textbf{UC24}});
		\item Il programma visualizza un messaggio di errore.
	\end{itemize}
\end{itemize}

\subsection{UC26 - Visualizzazione prompt generato}
\label{sec:UC26}
%\includegraphics[]{diagramma_UML}
\begin{itemize}
	\item \textbf{Descrizione:} L’utente riceve il prompt per la generazione della query;
	\item \textbf{Attori:} utente;
	\item \textbf{Precondizioni:} 
	\begin{itemize}
		\item L’utente ha selezionato un file di struttura Database (\hyperref[sec:UC22]{\textbf{UC22}});
		\item L’utente ha scritto una frase in linguaggio naturale nella casella di testo apposita e ne ha confermato l’inserimento (\hyperref[sec:UC24]{\textbf{UC24}});
	\end{itemize}
	\item \textbf{Postcondizioni:} 
	\begin{itemize}
		\item Il programma visualizza il prompt per la creazione della query richiesta;
	\end{itemize}
	\item \textbf{Scenario principale:} 
	\begin{itemize}
		\item L’utente ha il programma aperto;
		\item L’utente ha selezionato il database(\hyperref[sec:UC22]{\textbf{UC22}}) e inserito la frase(\hyperref[sec:UC24]{\textbf{UC24}});
		\item L’utente preme il bottone per ottenere il prompt;
		\item Il programma visualizza il prompt elaborato.
	\end{itemize}
\end{itemize}

\subsection{UC27 - Errore generazione prompt}
\label{sec:UC27}

\subsubsection{UC27.1 - Errore comunicazione con LLM}
\label{sec:UC27.1}

\subsubsection{UC27.2 - Errore dati mancanti per creazione prompt}
\label{sec:UC27.2}

\subsection{UC28 - Richiesta generazione query SQL}
\label{sec:UC28}

\subsection{UC29 - Errore generazione query SQL}
\label{sec:UC29}

\subsubsection{UC29.1 - Errore comunicazione con API}
\label{sec:UC29.1}

\subsubsection{UC29.2 - Errore formattazione prompt}
\label{sec:UC29.2}

\subsection{UC30 - Visualizzazione query SQL}
\label{sec:UC30}

%\subsection{UC - }
%\label{sec:UC}