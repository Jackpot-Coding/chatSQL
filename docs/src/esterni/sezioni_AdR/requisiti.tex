\DefTblrTemplate{contfoot-text}{default}{}
\DefTblrTemplate{conthead-text}{default}{}
\DefTblrTemplate{caption}{default}{}
\DefTblrTemplate{conthead}{default}{}
\DefTblrTemplate{capcont}{default}{}


\section{Requisiti}
In questa sezione vengono descritti e organizzati in forma tabellare i requisiti che il prodotto finale dovrà soddisfare. La struttura usata viene descritta all'interno del documento Norme di Progetto.


\subsection{Requisiti funzionali}
\begin{longtblr}
	{
		colspec={|Q[0.10\linewidth]|Q[0.50\linewidth]|Q[0.15\linewidth]|Q[0.15\linewidth]|},
		rows={halign=l},
		column{1}={halign=c},
		column{3}={halign=c},
		column{4}={halign=c},
		row{1}={halign=c},
		row{odd} = {gray!20},
		row{1}={teal!50},
	}
	% CONTENUTO %
	\hline
	\textbf{Codice} & \textbf{Descrizione} & \textbf{Classificazione} & \textbf{Fonti} \\
	\hline
	% RF& descrizione & classificazione & fonti \\
	% \hline
	RF1 & L'amministratore deve poter effettuare il \textit{login} per modificare i \textit{database} & Obbligatorio & UC1 - Verbale esterno \\
	\hline
	RF2 & Visualizzazione di errore in caso di credenziali errate durante il \textit{login} & Obbligatorio & UC2 \\
	\hline
	RF3 & L'amministratore deve poter creare un nuovo \textit{database} specificando nome e descrizione & Obbligatorio & UC3 \\
	\hline
	RF4 & Visualizzazione di errore in caso di tentativo di creazione di un \textit{database} con un nome già in uso & Obbligatorio & UC4 \\
	\hline
	RF5 & L'amministratore deve poter visualizzare la lista dei \textit{database} presenti & Obbligatorio & UC5 \\
	\hline
	RF6 & L'amministratore deve poter visualizzare i dettagli di un \textit{database} & Obbligatorio & UC6 \\
	\hline
	RF7 & L'amministratore deve poter effettuare modifiche a nome e descrizione di un \textit{database} & Obbligatorio & UC7 \\
	\hline
	RF8 & L'amministratore deve poter eliminare un \textit{database} presente & Obbligatorio & UC8 \\
	\hline
	RF9 & L'amministratore deve poter creare una nuova tabella specificando nome, sinonimi e descrizione & Obbligatorio & UC9 \\
	\hline
	RF10 & Visualizzazione di errore in caso di tentativo di creazione di una tabella con un nome già in uso o dati mancanti & Obbligatorio & UC10 \\
	\hline
	RF11 & L'amministratore deve poter effettuare modifiche a nome, sinonimi e descrizione della tabella & Obbligatorio & UC11 \\
	\hline
	RF12 & L'amministratore deve poter visualizzare la lista delle tabelle presenti in un \textit{database} & Obbligatorio & UC12 \\
	\hline
	RF13 & L'amministratore deve poter visualizzare i dettagli di una tabella & Obbligatorio & UC13 \\
	\hline
	RF14 & L'amministratore deve poter eliminare una tabella presente in un \textit{database} & Obbligatorio & UC14 \\
	\hline
	RF15 & L'amministratore deve poter creare un nuovo campo all'interno di una tabella specificando nome, tipo e sinonimi & Obbligatorio & UC15 \\
	\hline
	RF16 & Visualizzazione di errore in caso di tentativo di creazione di un campo con un nome già in uso o dati mancanti & Obbligatorio & UC16 \\
	\hline
	RF17 & L'amministratore deve poter visualizzare la lista dei campi presenti in una tabella & Obbligatorio & UC17 \\
	\hline
	RF18 & L'amministratore deve poter visualizzare i dettagli di un campo & Obbligatorio & UC18 \\
	\hline
	RF19 & L'amministratore deve poter effettuare modifiche a nome, tipo e sinonimi di un campo & Obbligatorio & UC19 \\
	\hline
	RF20 & L'amministratore deve poter eliminare un campo presente in una tabella & Obbligatorio & UC20 \\
	\hline
	RF21 & L'amministratore deve poter caricare la struttura di un nuovo \textit{database} tramite file & Obbligatorio & UC21 - Capitolato \\
	\hline
	RF22 & Visualizzazione di errore in caso di problemi durante il caricamento del file & Desiderabile & UC22 \\
	\hline
	RF23 & L'amministratore deve poter effettuare il \textit{logout} & Obbligatorio & UC23 \\
	\hline
	RF24 & Visualizzazione errore in caso di problemi in fase di \textit{logout} & Desiderabile & UC24 \\
	\hline
	RF25 & L'utente deve poter scegliere il \textit{database} da interrogare tra quelli presenti nel sistema & Obbligatorio & UC25 - Capitolato \\
	\hline
	RF26 & Visualizzazione di errore nel caso in cui l'utente non selezioni un \textit{database} da interrogare & Desiderabile & UC26 \\
	\hline
	RF27 & L'utente deve poter inserire la richiesta in linguaggio naturale per interrogare il \textit{database} & Obbligatorio & UC27 - Capitolato \\
	\hline
	RF28 & Visualizzazione di errore in caso di problemi con la frase inserita dall'utente & Desiderabile & UC28 \\
	\hline
	RF29 & Visualizzazione del \textit{prompt} generato da fornire poi all'LLM & Obbligatorio & UC29 - Capitolato \\
	\hline
	RF30 & Visualizzazione di errore in caso di problemi con la generazione del \textit{prompt} & Desiderabile & UC30 \\
	\hline
	RF31 & L'utente deve poter richiedere la generazione della \textit{query} utilizzando un servizio LLM esterno al sistema & Opzionale & UC31 - Capitolato \\
	\hline
	RF32 & Visualizzazione di errore in caso di problemi con la generazione della \textit{query} & Opzionale & UC32 - Capitolato \\
	\hline
	RF33 & Visualizzazione della \textit{query} generata dal LLM con il \textit{prompt} & Opzionale & UC33 - Capitolato \\
	\hline
	RF34 & Dare la possibilità all'utente di inserire la frase in linguaggio naturale tramite \textit{input} vocale & Opzionale & Capitolato \\
	\hline
	RF35 & Verificare la correttezza della frase SQL prodotta & Opzionale & Capitolato \\
	\hline
	RF36 & Implementare la gestione di più basi di dati & Opzionale & Capitolato - UC5, UC25 \\
	\hline
	RF37 & Utilizzare modelli LLM alternativi a \textit{ChatGPT} & Opzionale & Capitolato \\
	\hline
	RF38 & Avere a disposizione un modello addestrato appositamente per tradurre le frasi di interrogazione in italiano a SQL & Opzionale & Capitolato \\
	\hline
	% END_CONTENUTO %
\end{longtblr}

\newpage
\subsection{Requisiti di qualità}
\begin{longtblr}
	{
		colspec={|Q[0.10\linewidth]|Q[0.50\linewidth]|Q[0.15\linewidth]|Q[0.15\linewidth]|},
		rows={halign=l},
		column{1}={halign=c},
		column{3}={halign=c},
		column{4}={halign=c},
		row{1}={halign=c},
		row{odd} = {gray!20},
		row{1}={teal!50},
	}
	% CONTENUTO %
	\hline
	\textbf{Codice} & \textbf{Descrizione} & \textbf{Classificazione} & \textbf{Fonti} \\
	\hline
	% RQ& descrizione & classificazione & fonti \\
	% \hline
	RQ1 & Il prodotto deve essere sviluppato seguendo le Norme di Progetto definite & Obbligatorio & Norme di Progetto \\
	\hline
	RQ2 & Il codice sorgente deve essere presente su \textit{GitHub} all'interno di una \textit{repository} & Obbligatorio & Norme di Progetto \\
	\hline
	RQ3 & Viene fornita la documentazione dell'applicazione, presente all'interno della \textit{repository} & Desiderabile & Norme di Progetto \\
	\hline
	% END_CONTENUTO %
\end{longtblr}

\subsection{Requisiti di vincolo}
\begin{longtblr}
	{
		colspec={|Q[0.10\linewidth]|Q[0.50\linewidth]|Q[0.15\linewidth]|Q[0.15\linewidth]|},
		rows={halign=l},
		column{1}={halign=c},
		column{3}={halign=c},
		column{4}={halign=c},
		row{1}={halign=c},
		row{odd} = {gray!20},
		row{1}={teal!50},
	}
	% CONTENUTO %
	\hline
	\textbf{Codice} & \textbf{Descrizione} & \textbf{Classificazione} & \textbf{Fonti} \\
	\hline
	% RV& descrizione & classificazione & fonti \\
	% \hline
	RV1 & Sviluppo di un interfaccia tramite HTML e CSS & Opzionale & Verbale esterno \\
	\hline
	RV2 & Sviluppo dell'applicazione in \textit{Python} & Opzionale & Capitolato \\
	\hline
	RV3 & Il prodotto deve essere in grado di analizzare un file strutturato & Obbligatorio & Capitolato \\
	\hline
	% END_CONTENUTO %
\end{longtblr}

\subsection{Requisiti prestazionali}
Non sono stati individuati vincoli prestazionali.

\subsection{Tracciamento}
\subsubsection{Fonte - Requisiti}
\begin{longtblr}
	{
		colspec={|Q[0.25\linewidth]|Q[0.25\linewidth]|},
		rows={halign=c},
		row{odd} = {gray!20},
		row{1}={teal!50},
	}
	% CONTENUTO %
	\hline
	\textbf{Fonte} & \textbf{Requisiti} \\
	\hline
	% fonte & requisiti \\
	% \hline
	Capitolato & RF21, RF25, RF27, RF29, RF31, RF32, RF33, RF34, RF35, RF36, RF37, RF38, RV2, RV3 \\
	\hline
	UC1 & RF1 \\
	\hline
	UC2 & RF2 \\
	\hline
	UC3 & RF3 \\
	\hline
	UC4 & RF4 \\
	\hline
	UC5 & RF5 \\
	\hline
	UC6 & RF6 \\
	\hline
	UC7 & RF7 \\
	\hline
	UC8 & RF8 \\
	\hline
	UC9 & RF9 \\
	\hline
	UC10 & RF10 \\
	\hline
	UC11 & RF11 \\
	\hline
	UC12 & RF12 \\
	\hline
	UC13 & RF13 \\
	\hline
	UC14 & RF14 \\
	\hline
	UC15 & RF15 \\
	\hline
	UC16 & RF16 \\
	\hline
	UC17 & RF17 \\
	\hline
	UC18 & RF18 \\
	\hline
	UC19 & RF19 \\
	\hline
	UC20 & RF20 \\
	\hline
	UC21 & RF21 \\
	\hline
	UC22 & RF22 \\
	\hline
	UC23 & RF23 \\
	\hline
	UC24 & RF24 \\
	\hline
	UC25 & RF25 \\
	\hline
	UC26 & RF26 \\
	\hline
	UC27 & RF27 \\
	\hline
	UC28 & RF28 \\
	\hline
	UC29 & RF29 \\
	\hline
	UC30 & RF30 \\
	\hline
	UC31 & RF31 \\
	\hline
	UC32 & RF32 \\
	\hline
	UC33 & RF33 \\
	\hline
	Norme di Progetto  & RQ1, RQ2, RQ3 \\
	\hline
	Verbali esterni  & RF1, RV1 \\
	\hline
	% END_CONTENUTO %
\end{longtblr}


\subsubsection{Requisito - Fonti}
\begin{longtblr}
	{
		colspec={|Q[0.25\linewidth]|Q[0.25\linewidth]|},
		rows={halign=c},
		row{odd} = {gray!20},
		row{1}={teal!50},
	}
	% CONTENUTO %
	\hline
	\textbf{Requisito} & \textbf{Fonti} \\
	\hline
	% requisito & fonti \\
	% \hline
	RF1 & UC1, Verbale esterno \\
	\hline
	RF2 & UC2 \\
	\hline
	RF3 & UC3 \\
	\hline
	RF4 & UC4 \\
	\hline
	RF5 & UC5 \\
	\hline
	RF6 & UC6 \\
	\hline
	RF7 & UC7 \\
	\hline
	RF8 & UC8 \\
	\hline
	RF9 & UC9 \\
	\hline
	RF10 & UC10 \\
	\hline
	RF11 & UC11 \\
	\hline
	RF12 & UC12 \\
	\hline
	RF13 & UC13 \\
	\hline
	RF14 & UC14 \\
	\hline
	RF15 & UC15 \\
	\hline
	RF16 & UC16 \\
	\hline
	RF17 & UC17 \\
	\hline
	RF18 & UC18 \\
	\hline
	RF19 & UC19 \\
	\hline
	RF20 & UC20 \\
	\hline
	RF21 & UC21, Capitolato \\
	\hline
	RF22 & UC22 \\
	\hline
	RF23 & UC23 \\
	\hline
	RF24 & UC24 \\
	\hline
	RF25 & UC25, Capitolato \\
	\hline
	RF26 & UC26 \\
	\hline
	RF27 & UC27, Capitolato \\
	\hline
	RF28 & UC28 \\
	\hline
	RF29 & UC29, Capitolato \\
	\hline
	RF30 & UC30 \\
	\hline
	RF31 & UC31, Capitolato \\
	\hline
	RF32 & UC32, Capitolato \\
	\hline
	RF33 & UC33, Capitolato \\
	\hline
	RF34 & Capitolato \\
	\hline
	RF35 & Capitolato \\
	\hline
	RF36 & UC5, UC25, Capitolato \\
	\hline
	RF37 & Capitolato \\
	\hline
	RF38 & Capitolato \\
	\hline
	RQ1 & Norme di Progetto \\
	\hline
	RQ2 & Norme di Progetto \\
	\hline
	RQ3 & Norme di Progetto \\
	\hline
	RV1 & Verbale esterno \\
	\hline
	RV2 & Capitolato \\
	\hline
	RV3 & Capitolato \\
	\hline
	% END_CONTENUTO %
\end{longtblr}

\newpage
\subsection{Riepilogo}
\begin{longtblr}
	{
		colspec={|Q[0.15\linewidth]|Q[0.15\linewidth]|Q[0.15\linewidth]|Q[0.15\linewidth]|Q[0.10\linewidth]|},
		rows={halign=c},
		row{odd} = {gray!20},
		row{1}={teal!50},
	}
	% CONTENUTO %
	\hline
	\textbf{Tipologia} & \textbf{Obbligatorio} & \textbf{Desiderabile} & \textbf{Opzionale} & \textbf{Totale}\\
	\hline
	Funzionale & - & - & - & - \\
	\hline
	Di Qualità & - & - & - & - \\
	\hline
	Di Vincolo & - & - & - & - \\
	\hline
	Prestazionale & - & - & - & - \\
	\hline
	% END_CONTENUTO %
\end{longtblr}

