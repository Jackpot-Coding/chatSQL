\section{Introduzione}
\subsection{Scopo del documento}
Questo documento serve a fornire una descrizione dettagliata del funzionamento del prodotto, prestando particolare attenzione a come potrà essere usato e a quanto richiesto dai requisiti\textsuperscript{G} presentati e discussi con il proponente\textsuperscript{G}.
Analizzando questi cerchiamo quindi di individuare ed illustrare i diversi attori\textsuperscript{G} e i casi d’uso\textsuperscript{G} presenti all’interno del prodotto.\\
Ogni caso d’uso rappresenta uno scenario di utilizzo del programma da parte di un attore\textsuperscript{G}, per descriverlo al meglio utilizzeremo la struttura seguente:
\begin{itemize}
	\item Descrizione: breve descrizione del caso d'uso;
	\item Attori: chi esegue l'azione descritta;
	\item Precondizioni: stato del programma prima del caso d'uso;
	\item Postcondizioni: stato del programma dopo il caso d'uso;
	\item Scenario principale: azioni svolte prima, durante e dopo il caso d'uso;
	\item Generalizzazioni: se presenti, scomposizione del caso d'uso\textsuperscript{G}in sottocasi;
	\item Estensioni: se presenti, casi d'uso collegati (es. visualizzazione di errori o avvisi).
\end{itemize}
Ogni attore\textsuperscript{G} rappresenta una persona o un sistema esterno al programma che si interfaccia con esso.
Nel nostro caso il programma verrà utilizzato da due tipi di attori\textsuperscript{G} che avranno accesso a diverse funzionalità del prodotto:
\begin{itemize}
	\item \textbf{Utente\textsuperscript{G}}: può scegliere il \textit{database\textsuperscript{G}} a cui fare la richiesta e inserire il messaggio in linguaggio naturale\textsuperscript{G} che verrà utilizzato per la creazione del \textit{prompt\textsuperscript{G}};
	\item \textbf{Amministratore\textsuperscript{G}}: può aggiungere, modificare e eliminare i \textit{database\textsuperscript{G}} selezionabili dagli utenti\textsuperscript{G}.
\end{itemize} % __DA VERIFICARE__ decidere se lasciare la descrizione degli attori  qui o se spostarla

\subsection{Glossario}
Al fine di evitare incomprensioni riguardo la terminologia usata e per aiutare la comprensione del documento, viene fornito un Glossario nel file omonimo con la definizione precisa di ogni vocabolo potenzialmente ambiguo. Su questi termini verrà apposta un \textsuperscript{G} in apice per indicare la presenza della definizione nel Glossario.

\subsection{Riferimenti}
\subsubsection{Riferimenti normativi}
\begin{itemize}
	\item Capitolato\textsuperscript{G} C9 - \textit{ChatSQL} \\ \url{https://www.math.unipd.it/~tullio/IS-1/2023/Progetto/C9.pdf}
	\item Norme di progetto\textsuperscript{G}
\end{itemize}

\subsubsection{Riferimenti informativi}
\begin{itemize}
	\item Slide T5 - Analisi dei requisiti\textsuperscript{G} - Corso di Ingegneria del Software \\ \url{https://www.math.unipd.it/~tullio/IS-1/2023/Dispense/T5.pdf}
	\item Diagrammi dei casi d'uso - Corso di Ingegneria del Software \\ \url{https://www.math.unipd.it/~rcardin/swea/2022/Diagrammi%20Use%20Case.pdf}
\end{itemize}