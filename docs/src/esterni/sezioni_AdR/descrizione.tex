\section{Descrizione}
\subsection{Obiettivi del prodotto}
Il prodotto ha come obiettivo dare la possibilità di interrogare un \textit{database} partendo da una richiesta in linguaggio naturale, trasformandola in un \textit{prompt} da sottoporre ad un sistema di \textit{AI} per ottenere una \textit{query SQL} corretta in base alla struttura del \textit{database} interrogato.

\subsection{Funzioni del prodotto}
Il prodotto dovrà quindi, dati:
\begin{itemize}
	\item un file strutturato contenente le tabelle e le relazioni di un \textit{database};
	\item una frase in linguaggio naturale per interrogare suddetto \textit{database}.
\end{itemize}
Trovare nella frase le parole chiave, capendo quali sono i dati da visualizzare, in quali tabelle sono salvate, tenendo conto di eventuali condizioni imposte.
In seguito dovrà riscrivere la frase in modo da generare un \textit{prompt} che possa essere passato ad un sistema \textit{AI} che creerà la \textit{query} richiesta in linguaggio \textit{SQL}.
Il \textit{prompt} generato potrà essere semplicemente mostrato all’utente, che sarà tenuto ad inserirlo nel sistema \textit{AI} da lui scelto, oppure utilizzare le \textit{API} delle \textit{AI} per sottoporre direttamente il \textit{prompt} e mostrare all’utente il codice \textit{SQL}.
Il prodotto dovrà inoltre dare la possibilità agli utenti amministratori, dopo aver effettuato l’accesso, di:
\begin{itemize}
	\item aggiungere file strutturati selezionabili dagli utenti;
	\item modificare i file strutturati;
	\item eliminare file strutturati non più necessari.
\end{itemize}