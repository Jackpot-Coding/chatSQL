\section{Descrizione}
\subsection{Obiettivi del prodotto}
Il prodotto ha come obiettivo di dare la possibilità di interrogare un database partendo da una richiesta in linguaggio naturale e trasformandola in un prompt da sottoporre ad un sistema di AI per ottenere una query SQL corretta in base alla struttura del database interrogato.

\subsection{Funzioni del prodotto}
Il prodotto dovrà quindi, dati:
\begin{itemize}
	\item un file strutturato contenente le tabelle e le relazioni di un database;
	\item una frase in linguaggio naturale per interrogare suddetto database.
\end{itemize}
Trovare nella frase le parole chiave, capendo quali sono i dati da visualizzare, in quali tabelle sono salvate, tenendo conto di eventuali condizioni imposte.
In seguito riscrivere la frase in modo da generare un prompt che possa essere passato ad un sistema AI che creerà la query richiesta in linguaggio SQL.
Il prompt generato potrà essere semplicemente mostrato all’utente, che sarà tenuto ad inserirlo nel sistema AI da lui scelto, oppure utilizzare le API delle AI per sottoporre direttamente il prompt e mostrare all’utente il codice SQL.
Il prodotto dovrà inoltre dare la possibilità agli utenti amministratori, dopo aver effettuato l’accesso, di:
\begin{itemize}
	\item aggiungere file strutturati selezionabili dagli utenti;
	\item modificare i file strutturati;
	\item eliminare file strutturati non più necessari.
\end{itemize}