

\documentclass[5pt]{article}

\usepackage{sectsty}
\usepackage{graphicx}
\usepackage{lipsum} % for generating dummy text
\usepackage[margin=1in]{geometry}
\usepackage{setspace}
\usepackage{array}
\usepackage{cellspace}
\usepackage{tabularx}
\usepackage[table]{xcolor}
\usepackage{tabularray}
\usepackage{pgfplots}


\usepackage{hyperref}
\usepackage{scrextend}
\graphicspath{ {../assets} }


% package and setup for tables
\usepackage{float}
\usepackage[table]{xcolor}
\renewcommand{\arraystretch}{1.5}
\arrayrulecolor{black}


% Margins
\topmargin=-0.45in
\evensidemargin=0in
\oddsidemargin=0in
\textwidth=6.5in
\textheight=9.0in
\headsep=0.25in

\setlength{\parindent}{0pt}

\title{Piano di Qualifica}
\author{Jackpot Coding}
\renewcommand*\contentsname{Indice}
\date{\today}

%STARTOF THE DOCUMENT
\begin{document}
	
	%-------------------------
	
	% Reduce top margin only on the first page
	\newgeometry{top=0.5in}
	
	%UNIPD LOGO
	\vspace{8pt}
	%	\includegraphics[scale=0.2]{UNIPDFull.png}
	%END UNIPD LOGO
	
	\vspace{30pt}
	
	%COURSE INFO
	\begin{minipage}[t]{0.48\textwidth}
		%COURSE TITLE
		\begin{flushleft}
			Informatica\\
			\vspace{5pt}
			\textbf{\LARGE Ingegneria del Software}\\
			Anno Accademico: 2023/2024
		\end{flushleft}
		%END COURSE TITLE
	\end{minipage}
	%END COURSE INFO
	
	
	\vspace{5px}
	
	
	%BLACK LINE
	\rule{\textwidth}{5pt}
	
	%JACKPOT CODING INFO
	\begin{minipage}[t]{0.50\textwidth}
		%LOGO JACKPOT CODING
		\begin{flushleft}
			\hspace{10pt}
			%	\includegraphics[scale=0.65]{jackpot-logo.png} 
		\end{flushleft}
	\end{minipage}
	\hspace{-60pt} % This adds horizontal space between the minipages
	\begin{flushright}
		\begin{minipage}[t]{0.50\textwidth}
			%INFO JACKPOT CODING
			\begin{flushright}
				Gruppo: {\Large Jackpot Coding}\\
				Email: \href{mailto:jackpotcoding@gmail.com}{jackpotcoding@gmail.com}
			\end{flushright}
		\end{minipage}
	\end{flushright}
	%END JACKPOT CODING INFO
	
	\vspace{24pt}
	
	%TITLE
	\begin{center}
		\textbf{\LARGE SPECIFICA ARCHITETTURALE}
	\end{center}
	%END TITLE
	
	\vspace{13pt}
	
	\begin{flushleft}
		\begin{spacing}{1.5}
			REDATTORI: G. Moretto\\%INSERT HERE THE NAMES
			VERIFICATORI:\\
			\vspace{7pt}
			DESTINATARI: Prof. T. Vardanega, Prof. R. Cardin\\%INSERT HERE THE NAMES
		\end{spacing}
	\end{flushleft}
	
	\begin{flushright}
		\begin{spacing}{1}
			USO: ESTERNO\\
			VERSIONE: 0.0.2\\
		\end{spacing}
	\end{flushright}
	
	
	% Restore original margins from the second page onwards
	\restoregeometry
	
	\pagebreak
	
	\textbf{\Large Registro delle modifiche}
	\begin{table}[H]
		\centering
		\rowcolors{2}{black!15}{}
		\resizebox{\linewidth}{!}{
			\begin{tabular}{|c|c|c|c|c|}
				\rowcolor{teal!50}
				\hline
				\textbf{Versione} & \textbf{Data} & \textbf{Autore} & \textbf{Verificatore} & \textbf{Modifica} \\
				\hline
				0.0.2 & 02/04/2024 & G. Moretto & - & Aggiunta tabelle tecnologie codifica e testing \\
				\hline
				0.0.1 & 27/03/2024 & G. Moretto & M. Gobbo & Aggiunta struttura documento \\
				\hline
			\end{tabular}
		}
		\label{tab:conference}
	\end{table}
	
	\pagebreak
	\tableofcontents
	\pagebreak
	
	\section{Introduzione}
	
	\subsection{Scopo del Documento}
	
	\subsection{Scopo del Prodotto}
	
	\subsection{Glossario}
	
	\subsection{Riferimenti}
	
	\section{Tecnologie}
	
	\subsection{Codifica}
	\begin{longtblr}
		{
			colspec={|Q[0.15\linewidth]|Q[0.15\linewidth]|Q[0.70\linewidth]|},
			rows={halign=l},
			column{1}={halign=c},
			column{2}={halign=c},
			column{3}={halign=l},
			row{1}={halign=c},
			row{odd} = {gray!20},
			row{1}={teal!50},
			row{2}={teal!50},
			row{6}={teal!50},
			row{10}={teal!50}
		}
		\hline
		\textbf{Tecnologia} & \textbf{Versione} & \textbf{Descrizione} \\
		\hline
		\SetCell[c=3]{c} \textbf{Linguaggi} \\
		\hline
		HTML & 5 & Linguaggio di markup utilizzato per la definizione della struttura di pagine web \\
		\hline
		CSS & 3 & Linguaggio utilizzato per applicare stile a elementi presenti in una pagina HTML \\
		\hline
		Python & 3.11.8 & Linguaggio di programmazione ad alto livello, orientato agli oggetti. Viene utilizzato per la creazione del server. \\
		\hline
		\SetCell[c=3]{c} \textbf{Framework e Librerie} \\
		\hline
		Django & 5.0.3 & Framework per la creazione di applicazioni web scritto in linguaggio Python. \\
		\hline
		TensorFlow & 2.15.0 & Libreria Python per l'apprendimento automatico. \\
		\hline
		Transformers & 4.29.3 & Libreria Python per l'utilizzo di modelli del portale Hugging Face utilizzando TensorFlow\\
		\hline
		\SetCell[c=3]{c} \textbf{Strumenti} \\
		\hline
		Pip & 24.0 & Strumento per la gestione dei pacchetti utilizzati da applicazioni Python.\\
		\hline
		Git & 2.44.0 & Strumento per il controllo di versione utilizzato per la gestione della repository remota presente su GitHub. \\
		\hline
	\end{longtblr}
	
	\subsection{Testing}
	\begin{longtblr}
		{
			colspec={|Q[0.15\linewidth]|Q[0.15\linewidth]|Q[0.70\linewidth]|},
			rows={halign=l},
			column{1}={halign=c},
			column{2}={halign=c},
			column{3}={halign=l},
			row{1}={halign=c},
			row{odd} = {gray!20},
			row{1}={teal!50},
			row{2}={teal!50},
			row{7}={teal!50}
		}
		\hline
		\textbf{Tecnologia} & \textbf{Versione} & \textbf{Descrizione} \\
		\hline
		\SetCell[c=3]{c} \textbf{Framework e Librerie} \\
		\hline
		Unittest & 3.11.8 & Framework incluso nel linguaggio Python utilizzato per il testing di unità, utilizzato dal framework Django.\\
		\hline
		Django Test Client & 5.0.3 & Client per il testing di un applicazione web simulando un browser, integrato nel framework Django.\\
		\hline
		coverage.py & 7.4.4 & Tool per misurare il code coverage in applicazioni Python, integrabile nel framework Django. \\
		\hline
		Prospector & 1.10.3 & Tool per l'analisi statica di codice scritto nel linguaggio Python. \\
		\hline
		\SetCell[c=3]{c} \textbf{Strumenti} \\
		\hline
		GitHub Actions & - & Servizio di Github per la Continuous Integration, automatizza i processi di build, test e deploy del prodotto software.\\
		\hline
	\end{longtblr}
	
	\section{Architettura}
	
	\subsection{Introduzione}
	
	\subsection{Diagrammi delle classi}
	
	\subsection{Design pattern utilizzati}
	
	\section{Elenco Figure}
	
	\section{Eleco tabelle}
	
\end{document}