\documentclass[5pt]{article}

\usepackage{sectsty}
\usepackage{graphicx}
\usepackage{lipsum} % for generating dummy text
\usepackage[margin=1in]{geometry}
\usepackage{setspace}
\usepackage{array}
\usepackage{cellspace}


\usepackage{hyperref}
\usepackage{scrextend}
\graphicspath{ {./img} }


% Margins
\topmargin=-0.45in
\evensidemargin=0in
\oddsidemargin=0in
\textwidth=6.5in
\textheight=9.0in
\headsep=0.25in

\title{ Verbale - data }
\date{\today}

%STARTOF THE DOCUMENT
\begin{document}

%-------------------------

% Reduce top margin only on the first page
\newgeometry{top=0.5in}

%UNIPD LOGO
    \vspace{8pt}
    \includegraphics[scale=0.2]{assets/UNIPDFull.png}
%END UNIPD LOGO

\vspace{30pt}

%COURSE INFO
\begin{minipage}[t]{0.48\textwidth}
    %COURSE TITLE
        \begin{flushleft}
            Informatica\\
            \vspace{5pt}
            \textbf{\LARGE Ingegneria del Software}\\
            Anno Accademico: 2023/2024
        \end{flushleft}
    %END COURSE TITLE
\end{minipage}
%END COURSE INFO


\vspace{5px}


%BLACK LINE
\textcolor{}{\rule{\textwidth}{5pt}}

%JACKPOT CODING INFO
\begin{minipage}[t]{0.50\textwidth}
    %LOGO JACKPOT CODING
    \begin{flushleft}
        \hspace{10pt}
        \includegraphics[scale=0.65]{assets/jackpot-logo.png} 
    \end{flushleft}
\end{minipage}
\hspace{-60pt} % This adds horizontal space between the minipages
\begin{flushright}
    \begin{minipage}[t]{0.50\textwidth}
        %INFO JACKPOT CODING
        \begin{flushright}
            Gruppo: {\Large Jackpot Coding}\\
            Email: \href{mailto:jackpotcoding@gmail.com}{jackpotcoding@gmail.com}
        \end{flushright}
    \end{minipage}
\end{flushright}
%END JACKPOT CODING INFO

\vspace{24pt}

%TITLE
\begin{center}
    \textbf{\large VERBALE }
    \textbf{\large 27/10/2023} \\
    \textbf{\LARGE INCONTRO IMOLA INFORMATICA}
\end{center}
%END TITLE

\vspace{13pt}

\begin{flushleft}
    \begin{spacing}{1.5}
        REDATTORE:  G. Moretto, R. Simionato\\%INSERT HERE THE NAMES
        VERIFICATORE:   E. Gallo\\%INSERT HERE THE NAMES
        RESPONSABILE:   G. Moretto\\%INSERT HERE THE NAMES
        \vspace{7pt}
        SCRIBA: M. Gobbo\\%INSERT HERE THE NAMES
        \vspace{7pt}
        DESTINATARI:  Sig. A. Staffolani, Prof. T. Vardanega, Prof. R. Cardin\\%INSERT HERE THE NAMES
    \end{spacing}
\end{flushleft}

\begin{flushright}
    \begin{spacing}{1}
        USO: ESTERNO\\
        VERSIONE: 1.0\\
    \end{spacing}
\end{flushright}


% Restore original margins from the second page onwards
\restoregeometry

\pagebreak




% Optional TOC
% \tableofcontents
% \pagebreak

%--Paper--

\section{\Large ORARIO}
\begin{spacing}{1.5}
    {\large Inizio incontro: 17:30}\\
    {\large Fine incontro: 18:00}
\end{spacing}

\section{PARTECIPANTI}
% Define minimal spacing at the top and bottom of cells
\setlength\cellspacetoplimit{6pt}
\setlength\cellspacebottomlimit{6pt}

\begin{table}[ht]
  \begin{tabular}{|Sc|Sc|}
    \hline
    \textbf{Nome Componente} & \textbf{Durata della presenza} \\
    \hline
    Camillo Matteo & 30 minuti \\
    Favaretto Marco & 30 minuti \\
    Gallo Edoardo & 30 minuti \\
    Gobbo Marco & 30 minuti \\
    Moretto Giulio & 30 minuti \\
    Simionato Riccardo & 25 minuti \\
    Alessandro Staffolani (Imola Informatica S.p.A.) & 30 minuti\\
    \hline
  \end{tabular}
  \label{tab:conference}
\end{table}

\section{SINTESI DELL'INCONTRO}

Presente alla riunione per conto dell’azienda, Alessandro Staffolani, rappresentante di Imola
Informatica S.p.A.\\
Sono state poste le seguenti domande:
\begin{enumerate}
    \item Conferma dell'obiettivo del progetto.
    \item Tecnologie consigliate.
    \item Linguaggi di programmazione consigliati.
    \item Caratteristiche delle macchine virtuali a disposizione.
    \item Protocolli da utilizzare per la chat (caso d’uso 4).
    \item Cifratura della comunicazione app-server.
\end{enumerate}

\vspace{10pt}
Il progetto rispetta le nostre intuizioni come descritto dal capitolato, una web app responsive per la gestione di prenotazioni.

\vspace{10pt}
\noindent Non vi sono vincoli posti dall’azienda sulle tecnologie da utilizzare. Fondamentali html, css e javascript per lo sviluppo di una web app, l’azienda è maggiormente interessata alla parte documentale dove verranno descritte e motivate le scelte prese per lo sviluppo.

\vspace{10pt}
\noindent L’azienda mette a disposizione macchine virtuali Linux installate sui loro server a cui avremo libero accesso per testare la nostra applicazione. Le VM hanno inoltre installato Docker, utilizzato dall’azienda e applicazione scelta per il deploy dell’applicativo.\\
Se necessari, Imola Informatica si rende disponibile a tenere seminari riguardo tecnologie utilizzabili nel progetto come Docker o React.


\vspace{10pt}
\noindent L’implementazione della chat può essere semplice (senza crittografia), fino ad arrivare ad una crittografia end-to-end (“stile Whatsapp” con utilizzo del protocollo Signal), sta a noi la scelta.

\vspace{10pt}
\noindent Non ci sono ulteriori vincoli sulla crittografia all’interno dell’app. Se sviluppata, viene richiesta per la chat. Ulteriori implementazioni saranno a nostra discrezione.

%--/Paper--

\end{document}