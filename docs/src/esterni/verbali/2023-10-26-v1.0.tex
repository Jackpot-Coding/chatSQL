\documentclass[5pt]{article}

\usepackage{sectsty}
\usepackage{graphicx}
\usepackage{lipsum} % for generating dummy text
\usepackage[margin=1in]{geometry}
\usepackage{setspace}
\usepackage{array}
\usepackage{cellspace}


\usepackage{hyperref}
\usepackage{scrextend}
\graphicspath{ {./img} }


% Margins
\topmargin=-0.45in
\evensidemargin=0in
\oddsidemargin=0in
\textwidth=6.5in
\textheight=9.0in
\headsep=0.25in

\title{ Verbale - data }
\date{\today}

%STARTOF THE DOCUMENT
\begin{document}

%-------------------------

% Reduce top margin only on the first page
\newgeometry{top=0.5in}

%UNIPD LOGO
    \vspace{8pt}
    \includegraphics[scale=0.2]{assets/UNIPDFull.png}
%END UNIPD LOGO

\vspace{30pt}

%COURSE INFO
\begin{minipage}[t]{0.48\textwidth}
    %COURSE TITLE
        \begin{flushleft}
            Informatica\\
            \vspace{5pt}
            \textbf{\LARGE Ingegneria del Software}\\
            Anno Accademico: 2023/2024
        \end{flushleft}
    %END COURSE TITLE
\end{minipage}
%END COURSE INFO


\vspace{5px}


%BLACK LINE
\textcolor{}{\rule{\textwidth}{5pt}}

%JACKPOT CODING INFO
\begin{minipage}[t]{0.50\textwidth}
    %LOGO JACKPOT CODING
    \begin{flushleft}
        \hspace{10pt}
        \includegraphics[scale=0.65]{assets/jackpot-logo.png} 
    \end{flushleft}
\end{minipage}
\hspace{-60pt} % This adds horizontal space between the minipages
\begin{flushright}
    \begin{minipage}[t]{0.50\textwidth}
        %INFO JACKPOT CODING
        \begin{flushright}
            Gruppo: {\Large Jackpot Coding}\\
            Email: \href{mailto:jackpotcoding@gmail.com}{jackpotcoding@gmail.com}
        \end{flushright}
    \end{minipage}
\end{flushright}
%END JACKPOT CODING INFO

\vspace{24pt}

%TITLE
\begin{center}
    \textbf{\large VERBALE }
    \textbf{\large 26/10/2023} \\
    \textbf{\LARGE INCONTRO ZUCCHETTI}
\end{center}
%END TITLE

\vspace{13pt}

\begin{flushleft}
    \begin{spacing}{1.5}
        REDATTORE:  G. Moretto, R. Simionato\\%INSERT HERE THE NAMES
        VERIFICATORE:   E. Gallo\\%INSERT HERE THE NAMES
        RESPONSABILE:   G. Moretto\\%INSERT HERE THE NAMES
        \vspace{7pt}
        SCRIBA: M. Gobbo\\%INSERT HERE THE NAMES
        \vspace{7pt}
        DESTINATARI:   Sig. G. Piccoli, Prof. T. Vardanega, Prof. R. Cardin\\%INSERT HERE THE NAMES
    \end{spacing}
\end{flushleft}

\begin{flushright}
    \begin{spacing}{1}
        USO: ESTERNO\\
        VERSIONE: 1.0\\
    \end{spacing}
\end{flushright}


% Restore original margins from the second page onwards
\restoregeometry

\pagebreak




% Optional TOC
% \tableofcontents
% \pagebreak

%--Paper--

\section{\Large ORARIO}
\begin{spacing}{1.5}
    {\large Inizio incontro: 16:00}\\
    {\large Fine incontro: 16:40}
\end{spacing}

\section{PARTECIPANTI}
% Define minimal spacing at the top and bottom of cells
\setlength\cellspacetoplimit{6pt}
\setlength\cellspacebottomlimit{6pt}

\begin{table}[ht]
  \begin{tabular}{|Sc|Sc|}
    \hline
    \textbf{Nome Componente} & \textbf{Durata della presenza} \\
    \hline
    Camillo Matteo & 40 minuti \\
    Favaretto Marco & 40 minuti \\
    Gallo Edoardo & 40 minuti \\
    Gobbo Marco & 40 minuti \\
    Moretto Giulio & 40 minuti \\
    Simionato Riccardo & 30 minuti \\
    Gregorio Piccoli (Zucchetti S.p.A.) & 40 minuti \\
    \hline
  \end{tabular}
  \label{tab:conference}
\end{table}

\section{SINTESI DELL'INCONTRO}
Presente alla riunione per conto dell’azienda, Gregorio Piccoli, rappresentante di Zucchetti
S.p.A.
Sono state poste le seguenti domande:\\
\begin{enumerate}
    \item Conferma dell'obiettivo del progetto.
    \item Tecnologie consigliate.
    \item Se realizzare l’interfaccia utente come una pagina web o un applicativo eseguibile.
    \item Risorse messe a disposizione dall’azienda.
    \item Consigli sui modelli LLM utilizzabili.
\end{enumerate}

\vspace{10pt}
Il progetto rispetta le nostre intuizioni come descritto da documentazione fornita.

\vspace{10pt}
\noindent Non vi sono vincoli sulle tecnologie da utilizzare. Data la sua popolarità negli ambienti di
intelligenza artificiale, Python è consigliato come linguaggio di programmazione.\\
Consigliato inoltre il sito \href{https://huggingface.co}{huggingface.co} come solida base per l’apprendimento dei modelli LLM
e come strumento per testarli e selezionare il miglior approccio al progetto.\\
Viene inoltre
segnalata la libreria \href{https://github.com/neuml/txtai}{neuml/txtai} che l’azienda stessa utilizza.

\vspace{10pt}
\noindent Per l’interfaccia utente non vi è preferenza tra interfaccia web o applicazione nativa, in quanto
l’obiettivo del progetto è la creazione del prompt da poi utilizzare in un modello LLM.

\vspace{10pt}
\noindent Per la comunicazione asincrona e le riunioni l’azienda ha confermato l’utilizzo rispettivamente di
email e Zoom, con il quale si è trovata particolarmente bene negli anni scorsi.

\vspace{10pt}
\noindent Sono stati fatti notare i limiti sui modelli LLM, soprattutto per quanto riguarda l’accuratezza delle
informazioni restituite, ciò nonostante si manifesta un grande interesse per l’argomento in
quanto questi rappresentano un nuovo modo di interagire con i computer.

\vspace{10pt}
\noindent L’incontro è stato positivo, i nostri dubbi sono stati chiariti e si è manifestato interesse da tutti i
componenti del gruppo, confermando la scelta del capitolato precedentemente espressa negli
incontri interni.
%--/Paper--

\end{document}