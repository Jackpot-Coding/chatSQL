\documentclass[5pt]{article}

\usepackage{sectsty}
\usepackage{graphicx}
\usepackage{lipsum} % for generating dummy text
\usepackage[margin=1in]{geometry}
\usepackage{setspace}
\usepackage{array}
\usepackage{cellspace}


\usepackage{hyperref}
\usepackage{scrextend}
\graphicspath{ {../../assets} }


% Margins
\topmargin=-0.45in
\evensidemargin=0in
\oddsidemargin=0in
\textwidth=6.5in
\textheight=9.0in
\headsep=0.25in

\title{ Verbale - 12/04/2024 }
\date{\today}

%STARTOF THE DOCUMENT
\begin{document}

%-------------------------

% Reduce top margin only on the first page
\newgeometry{top=0.5in}

%UNIPD LOGO
    \vspace{8pt}
    \includegraphics[scale=0.2]{UNIPDFull.png}
%END UNIPD LOGO

\vspace{10pt}

%COURSE INFO
\begin{minipage}[t]{0.48\textwidth}
    %COURSE TITLE
        \begin{flushleft}
            Informatica\\
            \vspace{5pt}
            \textbf{\LARGE Ingegneria del Software}\\
            Anno Accademico: 2023/2024
        \end{flushleft}
    %END COURSE TITLE
\end{minipage}
%END COURSE INFO


\vspace{5px}


%BLACK LINE
\rule{\textwidth}{5pt}

%JACKPOT CODING INFO
\begin{minipage}[t]{0.50\textwidth}
    %LOGO JACKPOT CODING
    \begin{flushleft}
        \hspace{10pt}
        \includegraphics[scale=0.65]{jackpot-logo.png} 
    \end{flushleft}
\end{minipage}
\hspace{-60pt} % This adds horizontal space between the minipages
\begin{flushright}
    \begin{minipage}[t]{0.50\textwidth}
        %INFO JACKPOT CODING
        \begin{flushright}
            Gruppo: {\Large Jackpot Coding}\\
            Email: \href{mailto:jackpotcoding@gmail.com}{jackpotcoding@gmail.com}
        \end{flushright}
    \end{minipage}
\end{flushright}
%END JACKPOT CODING INFO

\vspace{20pt}

%TITLE
\begin{center}
    \textbf{\large VERBALE }
    \textbf{\large 12/04/2024} \\
    \textbf{\LARGE INCONTRO IN MERITO ALL'ARCHITETTURA DEL SISTEMA}
\end{center}
%END TITLE

\vspace{13pt}

\begin{flushleft}
    \begin{spacing}{1.5}
        REDATTORE: M. Favaretto\\ 
        VERIFICATORE: \\ 
        RESPONSABILE: \\ 
        \vspace{7pt}
        SCRIBA: M. Favaretto\\ 
        \vspace{7pt}
        DESTINATARI: Sig. G. Piccoli, Prof. T. Vardanega, Prof. R. Cardin\\ 
    \end{spacing}
\end{flushleft}

\begin{flushright}
    \begin{spacing}{1}
        USO: ESTERNO\\
        VERSIONE: 0.1\\
    \end{spacing}
\end{flushright}


% Restore original margins from the second page onwards
\restoregeometry

\pagebreak

% Optional TOC
% \tableofcontents
% \pagebreak

\section{ORARIO}
\begin{spacing}{1.5}
    {\large Inizio incontro: 16:00}\\
    {\large Fine incontro: 16:30} 
\end{spacing}

\section{PARTECIPANTI}
% Define minimal spacing at the top and bottom of cells
\setlength\cellspacetoplimit{6pt}
\setlength\cellspacebottomlimit{6pt}

\begin{table}[ht]
  \begin{tabular}{|Sc|Sc|}
    \hline
    \textbf{Nome Componente} & \textbf{Durata della presenza} \\
    \hline
    Camillo Matteo & 30 minuti \\
    Favaretto Marco & 30 minuti \\
    Gallo Edoardo & 30 minuti \\
    Gobbo Marco & 30 minuti \\
    Moretto Giulio & 30 minuti \\
    Simionato Riccardo & - \\
    Gregorio Piccoli (Zucchetti S.p.A.) & 30 minuti \\
    \hline
  \end{tabular}
  \label{tab:conference}
\end{table}

\section{SINTESI DELL'INCONTRO}
Durante l'incontro è stato descritto lo stato attuale del progetto in corso. 
Il primo punto discusso sono state le norme di qualifica, mostrando i \textit{test} fino ad ora realizzati e il \textit{code coverage} raggiunto da questi.
Successivamente è stata esposta l'architettura del \textit{software}, seguita da una descrizione del \textit{framework} adotatto per il progetto. \\
La fase finale dell'incontro si è concentrata sui confronti dei risultati ottenuti dai vari test della funzionalità principale del \textit{software}, 
sia eseguiti su macchina locale, che remota; con particolare attenzione in merito alla qualità delle risposte ottenute dai vari \textit{LLM}.



\section{CONCLUSIONI}
Dall'incontro sono emerse le seguenti attività: 
\begin{itemize}
    \item Possibilità di aumento del numero degli incontri, qualora necessario
    \item Aumento \textit{test} di qualità delle risposte degli \textit{LLM} usati dall'applicazione
    \item Il \textit{software} finale dovrà eseguire in locale, non è necessario che sia presente \textit{online}
\end{itemize}

\vspace{3em}
\begin{flushright}
	\begin{spacing}{1.5}
		Firma per presa visione\\
	\end{spacing}
\end{flushright}

\end{document}