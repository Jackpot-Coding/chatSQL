

\documentclass[5pt]{article}

\usepackage{sectsty}
\usepackage{graphicx}
\usepackage{lipsum} % for generating dummy text
\usepackage[margin=1in]{geometry}
\usepackage{setspace}
\usepackage{array}
\usepackage{cellspace}
\usepackage{tabularx}
\usepackage[table]{xcolor}
\usepackage{tabularray}
\usepackage{pgfplots}


\usepackage{hyperref}
\usepackage{scrextend}
\graphicspath{ {../assets} }

\usepackage[italian]{babel}


% package and setup for tables
\usepackage{float}
\usepackage[table]{xcolor}
\renewcommand{\arraystretch}{1.5}
\arrayrulecolor{black}


% Margins
\topmargin=-0.45in
\evensidemargin=0in
\oddsidemargin=0in
\textwidth=6.5in
\textheight=9.0in
\headsep=0.25in

\setlength{\parindent}{0pt}

\title{Specifica Tecnica}
\author{Jackpot Coding}
\renewcommand*\contentsname{Indice}
\date{\today}
%STARTOF THE DOCUMENT
\begin{document}
	
	%-------------------------
	
	% Reduce top margin only on the first page
	\newgeometry{top=0.5in}
	
	%UNIPD LOGO
	\vspace{8pt}
		\includegraphics[scale=0.2]{UNIPDFull.png}
	%END UNIPD LOGO
	
	\vspace{30pt}
	
	%COURSE INFO
	\begin{minipage}[t]{0.48\textwidth}
		%COURSE TITLE
		\begin{flushleft}
			Informatica\\
			\vspace{5pt}
			\textbf{\LARGE Ingegneria del Software}\\
			Anno Accademico: 2023/2024
		\end{flushleft}
		%END COURSE TITLE
	\end{minipage}
	%END COURSE INFO
	
	
	\vspace{5px}
	
	
	%BLACK LINE
	\rule{\textwidth}{5pt}
	
	%JACKPOT CODING INFO
	\begin{minipage}[t]{0.50\textwidth}
		%LOGO JACKPOT CODING
		\begin{flushleft}
			\hspace{10pt}
				\includegraphics[scale=0.65]{jackpot-logo.png} 
		\end{flushleft}
	\end{minipage}
	\hspace{-60pt} % This adds horizontal space between the minipages
	\begin{flushright}
		\begin{minipage}[t]{0.50\textwidth}
			%INFO JACKPOT CODING
			\begin{flushright}
				Gruppo: {\Large Jackpot Coding}\\
				Email: \href{mailto:jackpotcoding@gmail.com}{jackpotcoding@gmail.com}
			\end{flushright}
		\end{minipage}
	\end{flushright}
	%END JACKPOT CODING INFO
	
	\vspace{24pt}
	
	%TITLE
	\begin{center}
		\textbf{\LARGE SPECIFICA TECNICA}
	\end{center}
	%END TITLE
	
	\vspace{13pt}
	
	\begin{flushleft}
		\begin{spacing}{1.5}
			DESTINATARI: Prof. T. Vardanega, Prof. R. Cardin\\%INSERT HERE THE NAMES
		\end{spacing}
	\end{flushleft}
	
	\begin{flushright}
		\begin{spacing}{1}
			USO: ESTERNO\\
			VERSIONE: 0.0.11\\
		\end{spacing}
	\end{flushright}
	
	
	% Restore original margins from the second page onwards
	\restoregeometry
	
	\pagebreak
	
	\textbf{\Large Registro delle modifiche}
	\begin{table}[H]
		\centering
		\rowcolors{2}{black!15}{}
		\resizebox{\linewidth}{!}{
			\begin{tabular}{|c|c|c|c|c|}
				\rowcolor{teal!50}
				\hline
				\textbf{Versione} & \textbf{Data} & \textbf{Autore} & \textbf{Verificatore} & \textbf{Modifica} \\
				\hline
				0.0.11 & 18/04/2024 & G. Moretto & - & Aggiunta dei digrammi delle classi\\
				\hline
				0.0.10 & 10/04/2024 & G. Moretto & - & Aggiunta dello strumento di testing coveralls.io\\
				\hline
				0.0.9 & 09/04/2024 & - & G. Moretto & Documento rinominato in "Specifica Tecnica"\\
				\hline
				0.0.8 & 09/04/2024 & E. Gallo & G. Moretto & Aggiunti riferimenti normativi ed informativi \\ 
				\hline
				0.0.7 & 06/04/2024 & M. Favaretto & G. Moretto & Aggiunta sezione \textit{Design pattern} utilizzati - \textit{M.V.T.} \\ 
    			\hline
    			0.0.6 & 05/04/2024 & E. Gallo & M. Favaretto & Aggiunta sezione Design pattern utilizzati - Strategy \\
    			\hline
				0.0.5 & 04/04/2024 & M. Camillo & E. Gallo & Introduzione Documento \\
                \hline
				0.0.4 & 04/04/2024 & M. Gobbo & E. Gallo & Introduzione all'architettura \\
				\hline
				0.0.3 & 03/04/2024 & G. Moretto & M. Gobbo & Aggiunto elenco delle tabelle \\
				\hline
				0.0.2 & 02/04/2024 & G. Moretto & M. Gobbo & Aggiunta tabelle tecnologie codifica e testing \\
				\hline
				0.0.1 & 27/03/2024 & G. Moretto & M. Gobbo & Aggiunta struttura documento \\
				\hline
			\end{tabular}
		}
		\label{tab:conference}
	\end{table}
	
	\pagebreak
	\tableofcontents
	\pagebreak
	\textbf{\Large Elenco delle immagini} \\
	\makeatletter
	\@starttoc{lof}% Print List of Figures
	\makeatother
	
	\pagebreak
	\textbf{\Large Elenco delle tabelle} \\
	\makeatletter
	\@starttoc{lot}% Print List of Tables
	\makeatother
	\pagebreak
	
	\section{Introduzione}
	
	\subsection{Scopo del Documento}

    Il documento ha lo scopo di presentare e motivare le scelte architetturali e di design di applicate al prodotto, oltre che a fornire una lista completa delle tecnologie utilizzate. La struttura interna del prodotto è esposta all'interno del documento sotto forma di diagramma delle classi, in maniera da rendere più chiaro il software sviluppato. Vengono inoltre approfonditi e motivati a loro volta i design pattern utilizzati. 
	
	\subsection{Scopo del Prodotto}
    Il capitolato proposto dall'azienda \textit{Zucchetti S.p.A.} ha come obiettivo la realizzazione di un applicativo web al fine di studiare la fattibilità di un prodotto in grado di elaborare una frase in linguaggio naturale. Questa frase, anche se fornita da un utente inesperto, deve generare come output una \textit{query SQL} in grado di interrogare un database (di cui è conosciuta la struttura) in modo efficiente e affidabile.
	
	\subsection{Glossario}
    Al fine di evitare ambiguità o incomprensioni relative alla terminologia usata all'interno del documento, è fornito un \textit{Glossario} in cui vengono riportate definizioni precise per ogni termine potenzialmente ambiguo. La presenza di tali termini all'interno del documento è indicata con la presenza, vicino alla voce, di una \textit{G} in apice ($^G$). 
	\subsection{Riferimenti}
	\subsubsection{Riferimenti normativi}
	\begin{itemize}
		\item Capitolato\textsuperscript{G} C9 - \textit{ChatSQL} \\ \url{https://www.math.unipd.it/~tullio/IS-1/2023/Progetto/C9.pdf}
		\item Norme di progetto\textsuperscript{G} V1.0.2
		\item Glossario V1.0.0 \\
		\url{https://jackpot-coding.github.io/chatSQL/docs/esterni/glossario_v1.0.0.pdf}
	\end{itemize}
	\subsubsection{Riferimenti informativi}
	Slide del corso Ingegneria del Software:
	\begin{itemize}
		\item Progettazione software \\
		\url{https://www.math.unipd.it/~tullio/IS-1/2023/Dispense/T6.pdf}
		\item Diagramma delle classi \\
		\url{https://www.math.unipd.it/~rcardin/swea/2023/Diagrammi%20delle%20Classi.pdf}
		\item Pattern MVC e derivati \\
		\url{https://www.math.unipd.it/~rcardin/sweb/2022/L02.pdf}
		\item SOLID programming \\
		\url{https://www.math.unipd.it/~rcardin/swea/2021/SOLID%20Principles%20of%20Object-Oriented%20Design_4x4.pdf}
		\item Pattern comportamentali \\
		\url{https://www.math.unipd.it/~rcardin/swea/2021/Design%20Pattern%20Comportamentali_4x4.pdf}
	\end{itemize}
	
	\section{Tecnologie}
	
	\subsection{Codifica}

	\begin{longtblr}[
			caption = {Tecnologie di codifica.},
		]
		{
			colspec={|Q[0.15\linewidth]|Q[0.15\linewidth]|Q[0.70\linewidth]|},
			rows={halign=l},
			column{1}={halign=c},
			column{2}={halign=c},
			column{3}={halign=l},
			row{1}={halign=c},
			row{odd} = {gray!20},
			row{1}={teal!50},
			row{2}={teal!50},
			row{6}={teal!50},
			row{10}={teal!50}
		}
	
		\hline
		\textbf{Tecnologia} & \textbf{Versione} & \textbf{Descrizione} \\
		\hline
		\SetCell[c=3]{c} \textbf{Linguaggi} \\
		\hline
		HTML & 5 & Linguaggio di \textit{markup} utilizzato per la definizione della struttura di pagine \textit{web} \\
		\hline
		CSS & 3 & Linguaggio utilizzato per applicare stile a elementi presenti in una pagina \textit{HTML} \\
		\hline
		Python & 3.11.8 & Linguaggio di programmazione ad alto livello, orientato agli oggetti. Viene utilizzato per la creazione del \textit{server}. \\
		\hline
		\SetCell[c=3]{c} \textbf{Framework e Librerie} \\
		\hline
		Django & 5.0.3 & \textit{Framework} per la creazione di applicazioni \textit{web} scritto in linguaggio \textit{Python}. \\
		\hline
		TensorFlow & 2.15.0 & Libreria \textit{Python} per l'apprendimento automatico. \\
		\hline
		Transformers & 4.29.3 & Libreria \textit{Python} per l'utilizzo di modelli del portale \textit{Hugging Face} utilizzando \textit{TensorFlow}\\
		\hline
		\SetCell[c=3]{c} \textbf{Strumenti} \\
		\hline
		Pip & 24.0 & Strumento per la gestione dei pacchetti utilizzati da applicazioni \textit{Python}.\\
		\hline
		Git & 2.44.0 & Strumento per il controllo di versione utilizzato per la gestione della \textit{repository} remota presente su \textit{GitHub}. \\
		\hline
	\end{longtblr}
	
	\subsection{Testing}
	\begin{longtblr}[
		caption = {Tecnologie di testing.},
		]
		{
			colspec={|Q[0.15\linewidth]|Q[0.15\linewidth]|Q[0.70\linewidth]|},
			rows={halign=l},
			column{1}={halign=c},
			column{2}={halign=c},
			column{3}={halign=l},
			row{1}={halign=c},
			row{odd} = {gray!20},
			row{1}={teal!50},
			row{2}={teal!50},
			row{7}={teal!50}
		}
		\hline
		\textbf{Tecnologia} & \textbf{Versione} & \textbf{Descrizione} \\
		\hline
		\SetCell[c=3]{c} \textbf{Framework e Librerie} \\
		\hline
		Unittest & 3.11.8 & \textit{Framework} incluso nel linguaggio \textit{Python} utilizzato per il \textit{testing} di unità, utilizzato dal \textit{framework Django}.\\
		\hline
		Django Test Client & 5.0.3 & \textit{Client} per il \textit{testing} di un applicazione \textit{web} simulando un \textit{browser}, integrato nel \textit{framework Django}.\\
		\hline
		coverage.py & 7.4.4 & \textit{Tool} per misurare il \textit{code coverage} in applicazioni \textit{Python} integrabile nel \textit{framework Django}. \\
		\hline
		Prospector & 1.10.3 & \textit{Tool} per l'analisi statica di codice scritto nel linguaggio \textit{Python}. \\
		\hline
		\SetCell[c=3]{c} \textbf{Strumenti} \\
		\hline
		GitHub Actions & - & Servizio di \textit{Github} per la \textit{Continuous Integration}, automatizza i processi di \textit{build, test e deploy} del prodotto \textit{software}.\\
		\hline
		Coveralls.io & - & Servizio per la visualizzazione dei rapporti di \textit{code coverage} e applicazione di un \textit{badge} alla \textit{repository}.\\
		\hline
	\end{longtblr}
	
	\section{Architettura}
	
	\subsection{Introduzione}
L'architettura di \textit{"Chat SQL"} si basa sul \textit{design pattern} architetturale \textit{MVT(Model View Template)}.\\

In questo \textit{pattern}, il \textit{"Model"} rappresenta i dati e la logica di \textit{business} dell'applicazione, la \textit{"View"} esegue la \textit{business logic}, interagisce con il \textit{Model} e ritorna risposte Http, infine  il \textit{"Template"} definisce la struttura dell'interfaccia utente e come i dati vengono visualizzati al suo interno. Più nello specifico il \textit{"Template"}, nel caso della nostra applicazione, definisce la struttura di documenti \textit{HTML}.\\

Questo \textit{design pattern} è simile a \textit{MVC(Model View Controller)} \\\\
E' stato scelta questa architettura in quanto offre i seguenti vantaggi:
\begin{itemize}
    \item \textbf{Separazione delle responsabilità}: \textit{MVT} separa chiaramente le responsabilità tra \textit{Model}, \textit{View} e \textit{Template}. Questo permette una migliore organizzazione del codice, facilitando la manutenzione e la scalabilità dell'applicazione.

    \item \textbf{Riutilizzo dei template}: I \textit{template} in \textit{MVT} consentono di separare la presentazione dalla logica di \textit{business}. Ciò facilita il riutilizzo dei componenti di interfaccia utente in diverse parti dell'applicazione, riducendo la duplicazione del codice e migliorando l'efficienza dello sviluppo.

    \item \textbf{Aumento delle Performance}: Utilizzare i \textit{template} pre-renderizzati può migliorare le prestazioni dell'applicazione rispetto a un'architettura \textit{MVC} tradizionale, poiché il \textit{rendering} dei \textit{template} può essere più efficiente rispetto ad esempio alla generazione dinamica di \textit{HTML} lato \textit{server}.

\end{itemize}
	
	\subsection{Diagrammi delle classi}
	
	\subsubsection{Modelli}
	
	\begin{figure}[H]
		\includegraphics[scale=0.55]{UML_classes/models.png}
		\caption{Diagramma UML delle classi Model}
		\centering
	\end{figure}	
	
	Il compito del modello è quello di gestire i dati che vengono utilizzati dall'applicazione. Formano la struttura dei dati dell'intera applicazione e sono rappresentati in un database.\\
	
	I modelli definiti derivano dalla classe Model fornita dal framework Django e sono:
	\begin{itemize}
		\item \textbf{StrutturaDatabase}: rappresenta le varie strutture database definite dall'amministratore;
		\item \textbf{Tabella}: rappresenta le tabelle che compongono una Struttura Database;
		\item \textbf{Campo}: rappresenta i campi che compongono una Tabella. Questo utilizza un enumerazione chiamata TipoCampo che indica le tipologie di campo selezionabili;
	\end{itemize}
	
	Viene inoltre utilizzato il modello User fornito dal framework Django del quale si elencano gli attributi e operazioni più rilevanti.\\
	
	
	\subsubsection{View}
	
	\begin{figure}[H]
		\includegraphics[scale=0.55]{UML_classes/views.png}
		\caption{Diagramma UML delle classi View}
		\centering
	\end{figure}

	
	Il compito delle view è quella di ricevere richieste dal browser e di restituire risposte sottoforma di pagine HTML o altri elementi che possono essere visualizzati da un browser.\\
	
	Le view definite derivano dalla classe View fornita dal framework Django e sono divise per area principale e area amministrativa.\\
	
	La view dell'area principare è MainView e si occupa di ricevere la richiesta di generazione del prompt dall'utente e di restituirlo.
	Per fare questo utilizza una classe chiamata PromptCreator che si occupa di generare il prompt per la richiesta ricevuta dalla view.\\
	
	Le view dell'area amministrativa sono:
	\begin{itemize}
		\item \textbf{AdminLoginView}: responsabile per l'autenticazione dell'amministratore;
		\item \textbf{AdminLogoutView}: responsabile per il logout dell'amministratore;
		\item \textbf{AdminHomeView}: resituisce la pagina home dell'area amministrativa dove è presente la lista delle Strutture Database e il link per la creazione delle stesse;
		\item \textbf{AdminStrutturaDatabaseView}: restituisce la pagina di creazione e modifica della StrutturaDatabase e la lista delle tabelle che la compongono;
		\item \textbf{AdminTabellaView}: restituisce la pagina di creazione e modifica delle Tabelle e la lista dei campi che la compongono;
		\item \textbf{AdminCampoView}: restituise la pagina di creazione e modifica dei Campi di una tabella;
		\item \textbf{AdminFileUploadView}: restituisce la pagina per il caricamento di un file di struttura database e utilizza la classe FileUploader per la conversione ed il salvataggio come StrutturaDatabase;
		\item \textbf{AdminEliminaModelView}: responsabile per l'eliminazione di un oggetto dal database dato il suo id ed il suo modello. Restituisce una pagina per la conferma dell'eliminazione e l'eliminazione stessa;
	\end{itemize}
	
	\subsubsection{Form}
	\begin{figure}[H]
			\includegraphics[scale=0.65]{UML_classes/forms.png}
			\caption{Diagramma UML delle classi Form}
			\centering
	\end{figure}

	Il compito delle classi form e quella di definire from HTML sottoforma di classe. Questo per utilizzare le funzioni di validazione e gestione fornite dal framework Django tramite la classe Form dalla quale derivano i form da noi definiti:
	\begin{itemize}
		\item \textbf{NLPromptForm}: raccoglie i dati per la generazione del prompt per la generazione della query SQL;
		\item \textbf{LoginForm}: raccoglie i dati per il login dell'amministratore;
		\item \textbf{StrutturaDatabaseForm}: raccoglie i dati per la creazione e modifica di una Struttura Database;
		\item \textbf{TabellaForm}: raccoglie i dati per la creazione e modifica di una Tabella;
		\item \textbf{CampoForm}: raccoglie i dati per la creazione e modifica di un Campo;
		\item \textbf{FileUploadForm}: raccoglie i dati per il caricamento del file di struttura database;
		\item \textbf{EliminaForm}: raccoglie i dati per l'eliminazione di un oggetto dal database;
	\end{itemize}
	
	\subsection{Design pattern utilizzati}
			\subsubsection{Model-View-Template}
			% eventuale immagine mvt
			Per lo sviluppo del prodotto, il gruppo ha scelto l'utilizzo del \textit{framework Django}. Il \textit{framework} propone un'architettura integrata,
			basata su una generalizzazione della \textit{view}, attraverso il \textit{design pattern MVT}. \\
			L'architettura proposta da \textit{Django} si compone di:
			\begin{itemize}
			\item File per la gestione dei Modelli ;
			\item File per la gestione delle \textit{View};
			\item File per le impostazioni del progetto;
			\item File per la configurazione degli URL;
			\item File di \textit{template} per la definizione dell'interfaccia utente;
			\item File per la gestione dei \textit{form}; 
			\item File per la gestione dei test;
			\item Cartella contenente le migrazioni verso il sistema di Database scelto;
			\item Cartella contenente file statici come fogli di stile e immagini;
			\item Cartella per ogni applicazione che compone il prodotto;
			\end{itemize}

		\subsubsection{Strategy}
			% Immagine Design Pattern Strategy%
			Uno dei \textit{design pattern} comportamentali scelto dal gruppo è lo \textit{Strategy}\textsuperscript{G}. 
			In particolare, viene integrato per l'interrogazione e l'integrazione di diversi \textit{LLM}\textsuperscript{G}. 
			Il prodotto infatti deve lavorare a stretto contatto con questi modelli linguistici di grandi dimensioni, ma non sempre usarne uno solo è vantaggioso.  \\
			
			Spesso i modelli più capaci sono a pagamento e di notevoli dimensioni, vi è quindi la necessità di interrogare modelli di dimensioni ridotte ma specializzati in una singola funzione. \\
			
			Inoltre, l'interrogazione di diversi \textit{LLM} permette un confronto dei diversi risultati, fornendo un'interessante introspezione sulle capacità, sulle possibilità e sulle differenze dei modelli. \\
			
			Nel nostro caso sono due i tipi di interrogazioni agli \textit{LLM} che il gruppo ha individuato.
			\begin{enumerate}
				\item L'interrogazione principale, su cui si basa l'intero progetto: da un \textit{prompt}\textsuperscript{G} restituire una \textit{query} \textit{SQL}\textsuperscript{G}. 
				Questo per valutare le performance di diversi modelli nella generazione di una \textit{query} \textit{SQL} corretta. 
				Risulta quindi vantaggioso poter cambiare l'\textit{LLM} interrogato a seconda della complessità della richiesta e della volontà dell'utente. 
				
				\item Le interrogazioni ausiliarie, che non formano un requisito obbligatorio ma servono per facilitare l'uso del prodotto all'utente. 
					Un esempio è la generazione e l'inserimento dei sinonimi di tabelle e campi, che di base è lasciata all'utente, 
					ma è facilmente integrabile con un \textit{API}\textsuperscript{G} o un \textit{LLM} opportuno per migliorare l'esperienza del prodotto.
			\end{enumerate}
	
\end{document}
