\documentclass[5pt]{article}

\usepackage{sectsty}
\usepackage{graphicx}
\usepackage{lipsum} % for generating dummy text
\usepackage[margin=1in]{geometry}
\usepackage{setspace}
\usepackage{array}
\usepackage{cellspace}
\usepackage{tabularx}


\usepackage{hyperref}
\usepackage{scrextend}
\graphicspath{ {../assets} }


% Margins
\topmargin=-0.45in
\evensidemargin=0in
\oddsidemargin=0in
\textwidth=6.5in
\textheight=9.0in
\headsep=0.25in

\title{Piano di Progetto}
\author{Jackpot Coding}
\renewcommand*\contentsname{Indice}
\date{\today}

%STARTOF THE DOCUMENT
\begin{document}

%-------------------------

% Reduce top margin only on the first page
\newgeometry{top=0.5in}

%UNIPD LOGO
    \vspace{8pt}
    \includegraphics[scale=0.2]{UNIPDFull.png}
%END UNIPD LOGO

\vspace{30pt}

%COURSE INFO
\begin{minipage}[t]{0.48\textwidth}
    %COURSE TITLE
        \begin{flushleft}
            Informatica\\
            \vspace{5pt}
            \textbf{\LARGE Ingegneria del Software}\\
            Anno Accademico: 2023/2024
        \end{flushleft}
    %END COURSE TITLE
\end{minipage}
%END COURSE INFO


\vspace{5px}


%BLACK LINE
\rule{\textwidth}{5pt}

%JACKPOT CODING INFO
\begin{minipage}[t]{0.50\textwidth}
    %LOGO JACKPOT CODING
    \begin{flushleft}
        \hspace{10pt}
        \includegraphics[scale=0.65]{jackpot-logo.png} 
    \end{flushleft}
\end{minipage}
\hspace{-60pt} % This adds horizontal space between the minipages
\begin{flushright}
    \begin{minipage}[t]{0.50\textwidth}
        %INFO JACKPOT CODING
        \begin{flushright}
            Gruppo: {\Large Jackpot Coding}\\
            Email: \href{mailto:jackpotcoding@gmail.com}{jackpotcoding@gmail.com}
        \end{flushright}
    \end{minipage}
\end{flushright}
%END JACKPOT CODING INFO

\vspace{24pt}

%TITLE
\begin{center}
    \textbf{\LARGE PIANO DI PROGETTO}
\end{center}
%END TITLE

\vspace{13pt}

\begin{flushleft}
    \begin{spacing}{1.5}
        REDATTORE: M. Gobbo, M. Favaretto\\%INSERT HERE THE NAMES
        VERIFICATORI: \\
        \vspace{7pt}
        DESTINATARI: Prof. T. Vardanega, Prof. R. Cardin\\%INSERT HERE THE NAMES
    \end{spacing}
\end{flushleft}

\begin{flushright}
    \begin{spacing}{1}
        USO: ESTERNO\\
        VERSIONE: 0.0.1\\
    \end{spacing}
\end{flushright}


% Restore original margins from the second page onwards
\restoregeometry

\pagebreak

\textbf{\Large Registro delle modifiche}
\begin{table}[ht]
\centering
\begin{tabular}{|c|c|c|c|c|}
\hline
\textbf{Versione} & \textbf{Data} & \textbf{Autore} & \textbf{Verificatore} & \textbf{Modifica} \\
V0.0.1 & 03/02/2023 & M. Gobbo - M. Favaretto & - & Creata struttura del documento \\
\hline
v0.0.2 & 05/02/2024 & M. Favaretto & - & Scrittura introduzione \\
\hline
\end{tabular}
\caption{Cronologia delle modifiche}
\label{tab:conference}
\end{table}

\pagebreak
\tableofcontents
\pagebreak

\section{Introduzione}
\subsection{Scopo del documento}
Il presente documento ha come fine la presentazione della pianificazione e delle modalità di sviluppo di questo progetto. 
Nella fattispecie verranno esposte le analisi dei possibili rischi con alcune proposte di mitigazione, il modello adottato dal \textit{team}, 
il preventivo e il consuntivo di periodo.

\subsection{Scopo del capitolato}
Le \textit{I.A.\textsuperscript{G}} stanno vivendo un momento di grande innovazione ed entusiasmo, 
riuscendo a cogliere l'attenzione anche di utenti al di fuori dell'ambito informatico lavorativo e accademico, 
si veda come \textit{ChatGPT\textsuperscript{G}} sia diventato un fenomeno culturale. \\
La versatilità dell'\textit{I.A.\textsuperscript{G}} è oggi al centro dell'attenzione di molte \textit{software house}, 
poiché posso essere usate anche per migliorare e velocizzare la produzione. 
Per farne buon uso, è però necessario essere in grado di fornire i corretti \textit{prompt\textsuperscript{G}} al 
modello di \textit{I.A.\textsuperscript{G}} in uso. \\


L'obiettivo di questo progetto è realizzare un \textit{software\textsuperscript{G}} in grado di generare 
un \textit{prompt\textsuperscript{G}} a partire da una richiesta in linguaggio naturale. La richiesta dovrà riguardare un'interrogazione 
di un \textit{database} caricato sul sistema dall'utente. Tale \textit{prompt\textsuperscript{G}} sarà successivamente da fornire 
ad un \textit{LLM\textsuperscript{G}}, il quale restituirà all'utente una \textit{query\textsuperscript{G}} nel linguaggio 
\textit{SQL\textsuperscript{G}}.

\subsection{Glossario}
Alcuni termini presenti in questo documento potrebbero generare incomprensioni o necessitare di chiarimenti. 
Al fine di evitare queste eventualità, tali termini sono contrassegnati dalla lettera \textit{G} maiuscola posta ad apice della parola, 
per indicare che la loro spiegazione è presente all'interno del documento \textit{Glossario}.

\subsection{Riferimenti}
% --- mettere slide lezione? ---
\subsubsection{Riferimenti Informativi}
\subsubsection{Riferimenti Normativi}

\section{Analisi dei rischi}
\subsection{Rischi tecnologici}
\subsection{Rischi interni}
\subsubsection{Rischi organizzativi}
\subsubsection{Rischi comunicativi}

\section{Modello di Sviluppo}
\subsection{Modello (INSERIRE NOME MODELLO SCELTO)}
\subsection{Motivazione}

\section{Pianificazione}
\subsection{Verso la RTB}
\subsection{Verso la PB}
\subsection{Verso la CA}

\section{Preventivo}
\subsection{Periodo RTB}
\subsection{Periodo PB}
\subsection{Periodo CA}

\section{Consuntivo}
\subsection{Periodo RTB}


\end{document}
