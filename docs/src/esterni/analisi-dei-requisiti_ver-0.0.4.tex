\documentclass[5pt]{article}

\usepackage{sectsty}
\usepackage{graphicx}
\usepackage{lipsum} % for generating dummy text
\usepackage[margin=1in]{geometry}
\usepackage{setspace}
\usepackage{array}
\usepackage{cellspace}
\usepackage{tabularx}


\usepackage{hyperref}
\usepackage{scrextend}
\graphicspath{ {../assets} }


% Margins
\topmargin=-0.45in
\evensidemargin=0in
\oddsidemargin=0in
\textwidth=6.5in
\textheight=9.0in
\headsep=0.25in

\title{Analisi dei Requisiti}
\author{Jackpot Coding}
\renewcommand*\contentsname{Indice}
\date{\today}

%STARTOF THE DOCUMENT
\begin{document}

%-------------------------

% Reduce top margin only on the first page
\newgeometry{top=0.5in}

%UNIPD LOGO
    \vspace{8pt}
    \includegraphics[scale=0.2]{UNIPDFull.png}
%END UNIPD LOGO

\vspace{30pt}

%COURSE INFO
\begin{minipage}[t]{0.48\textwidth}
    %COURSE TITLE
        \begin{flushleft}
            Informatica\\
            \vspace{5pt}
            \textbf{\LARGE Ingegneria del Software}\\
            Anno Accademico: 2023/2024
        \end{flushleft}
    %END COURSE TITLE
\end{minipage}
%END COURSE INFO


\vspace{5px}


%BLACK LINE
\rule{\textwidth}{5pt}

%JACKPOT CODING INFO
\begin{minipage}[t]{0.50\textwidth}
    %LOGO JACKPOT CODING
    \begin{flushleft}
        \hspace{10pt}
        \includegraphics[scale=0.65]{jackpot-logo.png} 
    \end{flushleft}
\end{minipage}
\hspace{-60pt} % This adds horizontal space between the minipages
\begin{flushright}
    \begin{minipage}[t]{0.50\textwidth}
        %INFO JACKPOT CODING
        \begin{flushright}
            Gruppo: {\Large Jackpot Coding}\\
            Email: \href{mailto:jackpotcoding@gmail.com}{jackpotcoding@gmail.com}
        \end{flushright}
    \end{minipage}
\end{flushright}
%END JACKPOT CODING INFO

\vspace{24pt}

%TITLE
\begin{center}
    \textbf{\LARGE ANALISI DEI REQUISITI}
\end{center}
%END TITLE

\vspace{13pt}

\begin{flushleft}
    \begin{spacing}{1.5}
        REDATTORI: R. Simionato, G. Moretto\\%INSERT HERE THE NAMES
        VERIFICATORI: G. Moretto\\
        \vspace{7pt}
        DESTINATARI: Prof. T. Vardanega, Prof. R. Cardin\\%INSERT HERE THE NAMES
    \end{spacing}
\end{flushleft}

\begin{flushright}
    \begin{spacing}{1}
        USO: ESTERNO\\
        VERSIONE: 0.0.2\\
    \end{spacing}
\end{flushright}


% Restore original margins from the second page onwards
\restoregeometry

\pagebreak

\textbf{\Large Registro delle modifiche}
\begin{table}[ht]
\centerline{%
  \begin{tabular}{|c|c|c|c|c|}
    \hline
    \textbf{Versione} & \textbf{Data} & \textbf{Autore} & \textbf{Verificatore} & \textbf{Modifica} \\
    \hline
    % versione & data & autore & verificatore & descrizioneModifica \\
    % \hline
    v.0.0.5 & 7/01/2024 & R. Simionato & - & Aggiunti UC 3,4,7,8,9 e sottocasi  \\
    \hline
    v.0.0.4 & 5/01/2024 & R. Simionato & - & Aggiunta nuova struttura per i Casi d'Uso \\
    \hline
    v.0.0.3 & 29/12/2023 & G. Moretto & - & Aggiunti UC 6,7,8,9,9.1,9.2, commentati altri \\
    \hline
    V0.0.2 & 18/12/2023 & R. Simionato & G. Moretto & Aggiunte Introduzione, Descrizione e Casi d'uso \\
    \hline
    V0.0.1 & 15/11/2023 & R. Simionato & G. Moretto  & Creata struttura del documento \\
    \hline
  \end{tabular}%
  }
  \label{tab:conference}
\end{table}



\pagebreak
\tableofcontents
\pagebreak

% Elenco Figure
% Elenco Tabelle

\section{Introduzione}
\subsection{Scopo del documento}
Questo documento serve a fornire una descrizione dettagliata del funzionamento del prodotto, prestando particolare attenzione a come il prodotto potrà essere usato e a quanto richiesto dai requisiti presentati e discussi con il proponente.
Analizzando questi cerchiamo quindi di individuare ed illustrare i diversi attori e i casi d’uso presenti all’interno del prodotto.\\
Ogni caso d’uso rappresenta uno scenario di utilizzo del programma da parte di un attore, per descriverlo al meglio utilizzeremo la struttura seguente:
\begin{itemize}
    \item Descrizione: breve descrizione del caso d'uso;
    \item Attori: chi esegue l'azione descritta;
    \item Precondizioni: stato del programma prima del caso d'uso;
    \item Postcondizioni: stato del programma dopo il caso d'uso;
    \item Scenario principale: azioni svolte prima, durante e dopo il caso d'uso;
    \item Generalizzazioni: se presenti, scomposizione del caso d'uso in sottocasi;
    \item Estensioni: se presenti, casi d'uso collegati(es. visualizzazione di errori o avvisi).
\end{itemize}
Ogni attore rappresenta una persona o un sistema esterno al programma che si interfaccia con esso.
Nel nostro caso il programma verrà utilizzato da due tipi di attori che avranno accesso a diverse funzionalità del prodotto:
\begin{itemize}
    \item \textbf{Utente}: può scegliere il database a cui fare la richiesta e inserire il messaggio in linguaggio naturale che verrà utilizzato per la creazione del prompt;
    \item \textbf{Amministratore}: può aggiungere, modificare e eliminare i database selezionabili dagli utenti.
\end{itemize} % __DA VERIFICARE__ decidere se lasciare la descrizione degli attori qui o se spostarla

\subsection{Glossario}
\subsection{Riferimenti}

\section{Descrizione}
\subsection{Obiettivi del prodotto}
Il prodotto ha come obiettivo di dare la possibilità di interrogare un database partendo da una richiesta in linguaggio naturale e trasformandola in un prompt da sottoporre ad un sistema di AI per ottenere una query SQL corretta in base alla struttura del database interrogato.

\subsection{Funzioni del prodotto}
Il prodotto dovrà quindi, dati:
\begin{itemize}
    \item un file strutturato contenente le tabelle e le relazioni di un database;
    \item una frase in linguaggio naturale per interrogare suddetto database.
\end{itemize}
Trovare nella frase le parole chiave, capendo quali sono i dati da visualizzare, in quali tabelle sono salvate, tenendo conto di eventuali condizioni imposte.
In seguito riscrivere la frase in modo da generare un prompt che possa essere passato ad un sistema AI che creerà la query richiesta in linguaggio SQL.
Il prompt generato potrà essere semplicemente mostrato all’utente, che sarà tenuto ad inserirlo nel sistema AI da lui scelto, oppure utilizzare le API delle AI per sottoporre direttamente il prompt e mostrare all’utente il codice SQL.
Il prodotto dovrà inoltre dare la possibilità agli utenti amministratori, dopo aver effettuato l’accesso, di:
\begin{itemize}
    \item aggiungere file strutturati selezionabili dagli utenti;
    \item modificare i file strutturati;
    \item eliminare file strutturati non più necessari.
\end{itemize}

\section{Casi d'uso}
% TEMPLATE
%   \subsection{UC - NomeUseCase}
%   \label{sec:UC}
%   \includegraphics[]{diagramma_UML}
%   \begin{itemize}
%       \item \textbf{Descrizione:} 
%       \item \textbf{Attori:} 
%       \item \textbf{Precondizioni:} 
%       \item \textbf{Postcondizioni:} 
%       \item \textbf{Scenario principale:} 
%       \item \textbf{Generalizzazioni:} 
%       \item \textbf{Estensioni:} 
%   \end{itemize}

%   \hyperref[sec:UC]{\textbf{UC}}

% Link ad altre sezioni usando hyperref - utile per linkare generalizzazioni e estensioni
% La label fa da segnalibro a dove dovrà andare il link
%   \label{sec:nomeSezione}
% Link sul quale cliccare per andare alla label
%   \hyperref[sec:nomeSezione]{testo}
% END TEMPLATE


\iffalse

\subsection{UC8xx - Visualizzazione errore: caricamento struttura (formato non accettato)??}
\label{sec:UC8xxx}
%\includegraphics[]{diagramma_UML}
\begin{itemize}
    \item \textbf{Descrizione:} l’amministratore carica un file con un formato errato;
    \item \textbf{Attori:} amministratore;
    \item \textbf{Precondizioni:} 
    \begin{itemize}
        \item L’amministratore ha selezionato un file con formato diverso dal ???(da decidere);
    \end{itemize}
    \item \textbf{Postcondizioni:} 
    \begin{itemize}
        \item Il programma visualizza un messaggio di errore;
        \item Il file non viene elaborato;
    \end{itemize}
    \item \textbf{Scenario principale:} 
    \begin{itemize}
        \item L’amministratore si trova nel pannello amministrativo;
        \item L’amministratore sceglie un file in un formato diverso dal ??? tra i suoi file locali e lo carica nel programma;
        \item Il programma visualizza un messaggio di errore.
    \end{itemize}
\end{itemize}

\subsection{UC9xx - Modifica struttura database}
\label{sec:UC9xxx}
%\includegraphics[]{diagramma_UML}
\begin{itemize}
    \item \textbf{Descrizione:}  l’amministratore modifica una delle strutture database già presenti;
    \item \textbf{Attori:} amministratore;
    \item \textbf{Precondizioni:} 
    \begin{itemize}
        \item L’amministratore si trova nel pannello amministrativo;
        \item ???
    \end{itemize}
    \item \textbf{Postcondizioni:} 
    \begin{itemize}
        \item La struttura è stata modificata e salvata.
    \end{itemize}
    \item \textbf{Scenario principale:} 
    \begin{itemize}
        \item ???
    \end{itemize}
    \item \textbf{Generalizzazioni:} 
    \begin{itemize}
        \item ???
    \end{itemize}
\end{itemize}

\subsection{UC10xx - Eliminazione struttura database}
\label{sec:10xxx}
%\includegraphics[]{diagramma_UML}
\begin{itemize}
    \item \textbf{Descrizione:}  l’amministratore elimina una delle strutture database già presenti;
    \item \textbf{Attori:} amministratore;
    \item \textbf{Precondizioni:} 
    \begin{itemize}
        \item L’amministratore si trova nel pannello amministrativo;
        \item ???
    \end{itemize}
    \item \textbf{Postcondizioni:} 
    \begin{itemize}
        \item La struttura è stata eliminata.
    \end{itemize}
    \item \textbf{Scenario principale:} 
    \begin{itemize}
        \item ???
    \end{itemize}
    \item \textbf{Generalizzazioni:} 
    \begin{itemize}
        \item ???
    \end{itemize}
\end{itemize}

\subsection{UC - }
\label{sec:UC}

\fi

\subsection{UC1 - Login}
\label{sec:UC1}
%\includegraphics[]{diagramma_UML}
\begin{itemize}
	\item \textbf{Descrizione:} L’amministratore accede al pannello amministrativo con le sue credenziali;
	\item \textbf{Attori:} amministratore;
	\item \textbf{Precondizioni:} 
	\begin{itemize}
		\item L’amministratore possiede delle credenziali di accesso valide;
		\item L’amministratore non ha già effettuato l’accesso;
	\end{itemize}
	\item \textbf{Postcondizioni:} 
	\begin{itemize}
		\item L’utente Amministratore viene riconosciuto dal sistema;
	\end{itemize}
	\item \textbf{Scenario principale:} 
	\begin{itemize}
		\item L’ amministratore inserisce il proprio nome utente nel form di accesso (\hyperref[sec:UC1.1]{\textbf{UC1.1}});
		\item L’ amministratore inserisce la propria password nel form di accesso (\hyperref[sec:UC1.2]{\textbf{UC1.2}});
		\item Il sistema verifica che le credenziali ricevute siano corrette. 
	\end{itemize}
	\item \textbf{Generalizzazioni:} 
	\begin{itemize}
		\item \hyperref[sec:UC1.1]{\textbf{UC1.1}} - Inserimento nome utente
		\item \hyperref[sec:UC1.2]{\textbf{UC1.2}} - Inserimento password
	\end{itemize}
	\item \textbf{Estensioni:} Nel caso le credenziali non siano corrette:
	\begin{itemize}
		\item viene mostrato un errore - \hyperref[sec:UC5]{\textbf{UC5}}
	\end{itemize}
\end{itemize}

\subsubsection{UC1.1 - Inserimento nome utente}
\label{sec:UC1.1}
%\includegraphics[]{diagramma_UML}
\begin{itemize}
	\item \textbf{Descrizione:} L’amministratore inserisce il proprio nome utente;
	\item \textbf{Attori:} amministratore;
	\item \textbf{Precondizioni:} 
	\begin{itemize}
		\item L’amministratore possiede le credenziali di accesso;
		\item L’amministratore non ha già effettuato l’accesso;
		\item L’amministratore sta effettuando il login (\hyperref[sec:UC1]{\textbf{UC1}})
	\end{itemize}
	\item \textbf{Postcondizioni:} 
	\begin{itemize}
		\item L’amministratore ha inserito correttamente il proprio nome utente;
	\end{itemize}
	\item \textbf{Scenario principale:} 
	\begin{itemize}
		\item L’amministratore inserisce il proprio nome utente nel form di accesso.
	\end{itemize}
\end{itemize}

\subsubsection{UC1.2 - Inserimento password}
\label{sec:UC1.2}
%\includegraphics[]{diagramma_UML}
\begin{itemize}
	\item \textbf{Descrizione:} L’amministratore inserisce la propria password;
	\item \textbf{Attori:} amministratore;
	\item \textbf{Precondizioni:} 
	\begin{itemize}
		\item L’amministratore possiede le credenziali di accesso;
		\item L’amministratore non ha già effettuato l’accesso;
		\item L’amministratore sta effettuando il login (\hyperref[sec:UC1]{\textbf{UC1}})
	\end{itemize}
	\item \textbf{Postcondizioni:} 
	\begin{itemize}
		\item L’amministratore ha inserito correttamente la propria password;
	\end{itemize}
	\item \textbf{Scenario principale:} 
	\begin{itemize}
		\item L’amministratore inserisce la propria password nel form di accesso.
	\end{itemize}
\end{itemize}

\subsection{UC2 - Credenziali login errate}
\label{sec:UC2}
%\includegraphics[]{diagramma_UML}
\begin{itemize}
	\item \textbf{Descrizione:} L’amministratore visualizza un errore di autenticazione;
	\item \textbf{Attori:} amministratore;
	\item \textbf{Precondizioni:} 
	\begin{itemize}
		\item L’amministratore possiede le credenziali di accesso;
		\item L’amministratore non ha già effettuato l’accesso;
		\item L’amministratore sta effettuando il login (\hyperref[sec:UC1]{\textbf{UC1}});
	\end{itemize}
	\item \textbf{Postcondizioni:}
	\begin{itemize}
		\item L’amministratore non viene riconosciuto dal sistema e deve reinserire le proprie credenziali;
	\end{itemize}
	\item \textbf{Scenario principale:} 
	\begin{itemize}
		\item L’amministratore inserisce il proprio nome utente nel form di accesso (\hyperref[sec:UC1.1]{\textbf{UC1.1}});
		\item L’amministratore inserisce la propria password nel form di accesso (\hyperref[sec:UC1.2]{\textbf{UC1.2}});
		\item Il sistema verifica le credenziali ricevute siano corrette;
		\item Il sistema visualizza un messaggio di errore per le credenziali inserite.
	\end{itemize}
\end{itemize}

\subsection{UC3 - Creazione database}
\label{sec:UC3}
%\includegraphics[]{diagramma_UML}
\begin{itemize}
	\item \textbf{Descrizione:} l’amministratore vuole aggiungere la struttura di un database da poter interrogare;
	\item \textbf{Attori:} amministratore;
	\item \textbf{Precondizioni:} 
	\begin{itemize}
		\item L’amministratore ha effettuato il login (\hyperref[sec:UC1]{\textbf{UC1}});
		\item L’amministratore si trova nel pannello amministrativo;
	\end{itemize}
	\item \textbf{Postcondizioni:} 
	\begin{itemize}
		\item La struttura del database viene salvata nel programma;
	\end{itemize}
	\item \textbf{Scenario principale:} 
	\begin{itemize}
		\item L’amministratore inserisce il nome e la descrizione del database;
	\end{itemize}
	\item \textbf{Generalizzazioni:} 
	\begin{itemize}
		\item \hyperref[sec:UC3.1]{\textbf{UC3.1}} - Inserimento nome DB
		\item \hyperref[sec:UC3.2]{\textbf{UC3.2}} - Inserimento descrizione DB
	\end{itemize}
	\item \textbf{Estensioni:} nel caso in cui venga inserito un nome già esistente:
	\begin{itemize}
		\item \hyperref[sec:UC4]{\textbf{UC4}} - Errore: nome DB già presente
	\end{itemize}
\end{itemize}

\subsubsection{UC3.1 - Inserimento nome DB}
\label{sec:UC3.1}
%\includegraphics[]{diagramma_UML}
\begin{itemize}
	\item \textbf{Descrizione:} l’amministratore deve inserire il nome del nuovo database da aggiungere;
	\item \textbf{Attori:} amministratore;
	\item \textbf{Precondizioni:} 
	\begin{itemize}
		\item L’amministratore ha effettuato il login (\hyperref[sec:UC1]{\textbf{UC1}});
		\item L’amministratore si trova nel pannello amministrativo;
		\item L’amministratore sta creando un nuovo database (\hyperref[sec:UC3]{\textbf{UC3}});
	\end{itemize}
	\item \textbf{Postcondizioni:} 
	\begin{itemize}
		\item L'amministratore ha inserito correttamente il nome del nuovo database;
	\end{itemize}
	\item \textbf{Scenario principale:} 
	\begin{itemize}
		\item L’amministratore inserisce il nome del database nel form di creazione;
	\end{itemize}
\end{itemize}

\subsubsection{UC3.2 - Inserimento descrizione DB}
\label{sec:UC3.2}
%\includegraphics[]{diagramma_UML}
\begin{itemize}
	\item \textbf{Descrizione:} l’amministratore deve inserire la descrizione del nuovo database da aggiungere;
	\item \textbf{Attori:} amministratore;
	\item \textbf{Precondizioni:} 
	\begin{itemize}
		\item L’amministratore ha effettuato il login (\hyperref[sec:UC1]{\textbf{UC1}});
		\item L’amministratore si trova nel pannello amministrativo;
		\item L’amministratore sta creando un nuovo database (\hyperref[sec:UC3]{\textbf{UC3}});
	\end{itemize}
	\item \textbf{Postcondizioni:} 
	\begin{itemize}
		\item L'amministratore ha inserito correttamente la descrizione del nuovo database;
	\end{itemize}
	\item \textbf{Scenario principale:} 
	\begin{itemize}
		\item L’amministratore inserisce la descrizione del database nel form di creazione;
	\end{itemize}
\end{itemize}

\subsection{UC4 - Errore: nome DB già presente}
\label{sec:UC4}
%\includegraphics[]{diagramma_UML}
\begin{itemize}
	\item \textbf{Descrizione:} L’amministratore visualizza un errore di creazione del database;
	\item \textbf{Attori:} amministratore;
	\item \textbf{Precondizioni:} 
	\begin{itemize}
		\item L’amministratore ha effettuato il login (\hyperref[sec:UC1]{\textbf{UC1}});
		\item L’amministratore si trova nel pannello amministrativo;
		\item L’amministratore sta creando un nuovo database (\hyperref[sec:UC3]{\textbf{UC3}});
	\end{itemize}
	\item \textbf{Postcondizioni:} 
	\begin{itemize}
		\item La struttura del database non viene salvata nel programma e visualizza un messaggio di errore;
	\end{itemize}
	\item \textbf{Scenario principale:} 
	\begin{itemize}
		\item L’amministratore inserisce il nome del database nel form di creazione (\hyperref[sec:UC3.1]{\textbf{UC3.1}});
		\item L’amministratore inserisce la descrizione del database nel form di creazione (\hyperref[sec:UC3.2]{\textbf{UC3.2}});
		\item Il sistema verifica che non esista già un database con lo stesso nome;
		\item Il sistema visualizza un messaggio di errore per il nome inserito.
	\end{itemize}
\end{itemize}

\subsection{UC5 - Visualizzazione DB}
\label{sec:UC5}
%\includegraphics[]{diagramma_UML}
\begin{itemize}
	\item \textbf{Descrizione:} l’amministratore visualizza tutte le strutture database disponibili;
	\item \textbf{Attori:} amministratore;
	\item \textbf{Precondizioni:} 
	\begin{itemize}
		\item L’amministratore ha effettuato il login (\hyperref[sec:UC1]{\textbf{UC1}});
		\item L’amministratore si trova nel pannello amministrativo;
	\end{itemize}
	\item \textbf{Postcondizioni:} 
	\begin{itemize}
		\item L'amministrazione naviga il pannello amministrativo e può vedere nome e descrizione dei database presenti;
	\end{itemize}
	\item \textbf{Scenario principale:} 
	\begin{itemize}
		\item Il programma visualizza la lista dei database presenti, con la possibilità di modificarli, visualizzarli o eliminarli; ???
	\end{itemize}
\end{itemize}

\subsection{UC6 - Modifica DB}
\label{sec:UC6}

\subsection{UC7 - Elimina DB}
\label{sec:UC7}
%\includegraphics[]{diagramma_UML}
\begin{itemize}
	\item \textbf{Descrizione:} l’amministratore elimina la struttura database selezionata;
	\item \textbf{Attori:} amministratore;
	\item \textbf{Precondizioni:} 
	\begin{itemize}
		\item L’amministratore ha effettuato il login (\hyperref[sec:UC1]{\textbf{UC1}});
		\item L’amministratore si trova nel pannello amministrativo;
		\item L’amministratore sta visualizzando la lista dei database; ???
	\end{itemize}
	\item \textbf{Postcondizioni:} 
	\begin{itemize}
		\item La struttura database selezionata viene eliminata dal sistema;
	\end{itemize}
	\item \textbf{Scenario principale:} 
	\begin{itemize}
		\item L'amministratore sta visualizzando la lista dei database presenti nel sistema (\hyperref[sec:UC5]{\textbf{UC5}});
		\item L'amministratore seleziona il database da eliminare usando il pulsante di eliminazione apposito;
		\item Il sistema visualizza un messaggio di conferma dell'eliminazione;
		\item Se l'amministrazione conferma l'eliminazione, il database e le tabelle collegate verranno rimossi dal sistema e verrà visualizzato un messaggio di avvenuta eliminazione.
	\end{itemize}
\end{itemize}

\subsection{UC8 - Creazione tabella DB}
\label{sec:UC8}
%\includegraphics[]{diagramma_UML}
\begin{itemize}
	\item \textbf{Descrizione:} l’amministratore vuole aggiungere una tabella alla struttura del database da interrogare;
	\item \textbf{Attori:} amministratore;
	\item \textbf{Precondizioni:} 
	\begin{itemize}
		\item L’amministratore ha effettuato il login (\hyperref[sec:UC1]{\textbf{UC1}});
		\item L’amministratore si trova nel pannello amministrativo;
		\item L’amministratore si trova nella sezione di creazione di una nuova tabella;
	\end{itemize}
	\item \textbf{Postcondizioni:} 
	\begin{itemize}
		\item La tabella viene aggiunta alla struttura del database;
	\end{itemize}
	\item \textbf{Scenario principale:} 
	\begin{itemize}
		\item L’amministratore inserisce il nome, i sinonimi del nome e la descrizione della tabella;
	\end{itemize}
	\item \textbf{Generalizzazioni:} 
	\begin{itemize}
		\item \hyperref[sec:UC8.1]{\textbf{UC8.1}} - Inserimento nome tabella
		\item \hyperref[sec:UC8.2]{\textbf{UC8.2}} - Inserimento sinonimi tabella
		\item \hyperref[sec:UC8.3]{\textbf{UC8.3}} - Inserimento descrizione tabella
	\end{itemize}
	\item \textbf{Estensioni:} nel caso in cui non vengano inseriti i sinonimi del nome della tabella, o il nome esisti già:
	\begin{itemize}
		\item \hyperref[sec:UC9]{\textbf{UC9}} - Errore nella creazione della tabella
	\end{itemize}
\end{itemize}

\subsubsection{UC8.1 - Inserimento nome tabella}
\label{sec:UC8.1}
%\includegraphics[]{diagramma_UML}
\begin{itemize}
	\item \textbf{Descrizione:} l’amministratore inserisce il nome della tabella da creare;
	\item \textbf{Attori:} amministratore;
	\item \textbf{Precondizioni:} 
	\begin{itemize}
		\item L’amministratore ha effettuato il login (\hyperref[sec:UC1]{\textbf{UC1}});
		\item L’amministratore sta creando una nuova tabella (\hyperref[sec:UC1]{\textbf{UC8}});
	\end{itemize}
	\item \textbf{Postcondizioni:} 
	\begin{itemize}
		\item Il nome della tabella viene inserito nel form;
	\end{itemize}
	\item \textbf{Scenario principale:} 
	\begin{itemize}
		\item L’amministratore inserisce il nome della tabella nell'apposito form di creazione;
	\end{itemize}
\end{itemize}

\subsubsection{UC8.2 - Inserimento sinonimi tabella}
\label{sec:UC8.2}
%\includegraphics[]{diagramma_UML}
\begin{itemize}
	\item \textbf{Descrizione:} l’amministratore inserisce i sinonimi associati al nome della tabella da creare;
	\item \textbf{Attori:} amministratore;
	\item \textbf{Precondizioni:} 
	\begin{itemize}
		\item L’amministratore ha effettuato il login (\hyperref[sec:UC1]{\textbf{UC1}});
		\item L’amministratore sta creando una nuova tabella (\hyperref[sec:UC1]{\textbf{UC8}});
	\end{itemize}
	\item \textbf{Postcondizioni:} 
	\begin{itemize}
		\item I sinonimi del nome della tabella vengono inseriti nel form;
	\end{itemize}
	\item \textbf{Scenario principale:} 
	\begin{itemize}
		\item L’amministratore inserisce i sinonimi del nome della tabella nell'apposito form di creazione;
	\end{itemize}
\end{itemize}

\subsubsection{UC8.3 - Inserimento descrizione tabella}
\label{sec:UC8.3}
%\includegraphics[]{diagramma_UML}
\begin{itemize}
	\item \textbf{Descrizione:} l’amministratore inserisce la descrizione della tabella da creare;
	\item \textbf{Attori:} amministratore;
	\item \textbf{Precondizioni:} 
	\begin{itemize}
		\item L’amministratore ha effettuato il login (\hyperref[sec:UC1]{\textbf{UC1}});
		\item L’amministratore sta creando una nuova tabella (\hyperref[sec:UC1]{\textbf{UC8}});
	\end{itemize}
	\item \textbf{Postcondizioni:} 
	\begin{itemize}
		\item La descrizione della tabella viene inserita nel form;
	\end{itemize}
	\item \textbf{Scenario principale:} 
	\begin{itemize}
		\item L’amministratore inserisce la descrizione della tabella nell'apposito form di creazione;
	\end{itemize}
\end{itemize}

\subsection{UC9 - Errore nella creazione della tabella}
\label{sec:UC9}
%\includegraphics[]{diagramma_UML}
\begin{itemize}
	\item \textbf{Descrizione:} L’amministratore visualizza un errore di creazione della tabella;
	\item \textbf{Attori:} amministratore;
	\item \textbf{Precondizioni:} 
	\begin{itemize}
		\item L’amministratore ha effettuato il login (\hyperref[sec:UC1]{\textbf{UC1}});
		\item L’amministratore sta creando una nuova tabella (\hyperref[sec:UC1]{\textbf{UC8}});
	\end{itemize}
	\item \textbf{Postcondizioni:} 
	\begin{itemize}
		\item La tabella non viene creata e il programma visualizza un messaggio di errore;
	\end{itemize}
	\item \textbf{Scenario principale:} 
	\begin{itemize}
		\item L’amministratore inserisce il nome della tabella nel form di creazione (\hyperref[sec:UC8.1]{\textbf{UC8.1}});
		\item L’amministratore inserisce i sinonimi del nome della tabella nel form di creazione (\hyperref[sec:UC8.2]{\textbf{UC8.2}});
		\item L’amministratore inserisce la descrizione della tabella nel form di creazione (\hyperref[sec:UC8.3]{\textbf{UC8.3}});
		\item Il sistema verifica che non esista già una tabella con lo stesso nome e che vengano inseriti sinonimi e descrizione della tabella;
		\item Il sistema visualizza il messaggio di errore opportuno.
	\end{itemize}
\end{itemize}

\subsubsection{UC9.1 - Errore nome tabella già presente}
\label{sec:UC9.1}
%\includegraphics[]{diagramma_UML}
\begin{itemize}
	\item \textbf{Descrizione:} L’amministratore visualizza un errore relativo al nome della tabella;
	\item \textbf{Attori:} amministratore;
	\item \textbf{Precondizioni:} 
	\begin{itemize}
		\item L’amministratore ha effettuato il login (\hyperref[sec:UC1]{\textbf{UC1}});
		\item L’amministratore sta creando una nuova tabella (\hyperref[sec:UC1]{\textbf{UC8}});
	\end{itemize}
	\item \textbf{Postcondizioni:} 
	\begin{itemize}
		\item La tabella non viene creata e il programma visualizza un messaggio di errore;
	\end{itemize}
	\item \textbf{Scenario principale:} 
	\begin{itemize}
		\item Il sistema verifica che non esista già una tabella con lo stesso nome;
		\item Il sistema visualizza il messaggio di errore per il nome inserito.
	\end{itemize}
\end{itemize}

\subsubsection{UC9.2 - Errore sinonimi non inseriti}
\label{sec:UC9.2}
%\includegraphics[]{diagramma_UML}
\begin{itemize}
	\item \textbf{Descrizione:} L’amministratore visualizza un errore relativo ai sinonimi del nome della tabella;
	\item \textbf{Attori:} amministratore;
	\item \textbf{Precondizioni:} 
	\begin{itemize}
		\item L’amministratore ha effettuato il login (\hyperref[sec:UC1]{\textbf{UC1}});
		\item L’amministratore sta creando una nuova tabella (\hyperref[sec:UC1]{\textbf{UC8}});
	\end{itemize}
	\item \textbf{Postcondizioni:} 
	\begin{itemize}
		\item La tabella non viene creata e il programma visualizza un messaggio di errore;
	\end{itemize}
	\item \textbf{Scenario principale:} 
	\begin{itemize}
		\item Il sistema verifica che il campo relativo ai sinonimi del nome della tabella non sia vuoto;
		\item Il sistema visualizza il messaggio di errore per il campo sinonimi vuoto.
	\end{itemize}
\end{itemize}

\subsubsection{UC9.3 - Errore descrizione non inserita}
\label{sec:UC9.3}
%\includegraphics[]{diagramma_UML}
\begin{itemize}
	\item \textbf{Descrizione:} L’amministratore visualizza un errore relativo alla descrizione della tabella;
	\item \textbf{Attori:} amministratore;
	\item \textbf{Precondizioni:} 
	\begin{itemize}
		\item L’amministratore ha effettuato il login (\hyperref[sec:UC1]{\textbf{UC1}});
		\item L’amministratore sta creando una nuova tabella (\hyperref[sec:UC1]{\textbf{UC8}});
	\end{itemize}
	\item \textbf{Postcondizioni:} 
	\begin{itemize}
		\item La tabella non viene creata e il programma visualizza un messaggio di errore;
	\end{itemize}
	\item \textbf{Scenario principale:} 
	\begin{itemize}
		\item Il sistema verifica che il campo relativo ai sinonimi del nome della tabella non sia vuoto;
		\item Il sistema visualizza il messaggio di errore per il campo descrizione vuoto.
	\end{itemize}
\end{itemize}

\subsection{UC10 - Modifica della tabella}
\label{sec:UC10}

\subsection{UC11 - Visualizzazione della tabella}
\label{sec:UC11}

\subsection{UC12 - Eliminazione della tabella}
\label{sec:UC12}

\subsection{UC13 - Creazione campo tabella}
\label{sec:UC13}
%\includegraphics[]{diagramma_UML}
\begin{itemize}
	\item \textbf{Descrizione:} l’amministratore vuole aggiungere i campi che compongono le tabelle alla struttura del database;
	\item \textbf{Attori:} amministratore;
	\item \textbf{Precondizioni:} 
	\begin{itemize}
		\item L’amministratore ha effettuato il login (\hyperref[sec:UC1]{\textbf{UC1}});
		\item L’amministratore si trova nel pannello amministrativo;
		\item L’amministratore si trova nella sezione di inserimento della struttura database;
		\item L’amministratore sta inserendo i campi che compongono la tabella;
	\end{itemize}
	\item \textbf{Postcondizioni:} 
	\begin{itemize}
		\item I campi vengono aggiunti alla tabella;
	\end{itemize}
	\item \textbf{Scenario principale:} 
	\begin{itemize}
		\item L’amministratore inserisce il nome del campo, ne seleziona il tipo e inserisce i sinonimi;
	\end{itemize}
	\item \textbf{Generalizzazioni:} 
	\begin{itemize}
		\item \hyperref[sec:UC13.1]{\textbf{UC13.1}} - Inserimento nome campo
		\item \hyperref[sec:UC13.2]{\textbf{UC13.2}} - Inserimento tipo campo
		\item \hyperref[sec:UC13.3]{\textbf{UC13.3}} - Inserimento sinonimi campo
	\end{itemize}
	\item \textbf{Estensioni:} nel caso in cui il nome inserito sia già esistente o non sia stato selezionato il tipo o inseriti i sinonimi:
	\begin{itemize}
		\item \hyperref[sec:UC14]{\textbf{UC14}} - Errore creazione campo
	\end{itemize}
\end{itemize}

\subsubsection{UC13.1 - Inserimento nome campo}
\label{sec:UC13.1}
%\includegraphics[]{diagramma_UML}
\begin{itemize}
	\item \textbf{Descrizione:} l’amministratore vuole inserire il nome del campo da inserire nella tabella;
	\item \textbf{Attori:} amministratore;
	\item \textbf{Precondizioni:} 
	\begin{itemize}
		\item L’amministratore ha effettuato il login (\hyperref[sec:UC1]{\textbf{UC1}});
		\item L’amministratore si trova nel pannello amministrativo;
		\item L’amministratore si trova nella sezione di inserimento della struttura database;
		\item L’amministratore sta inserendo i campi che compongono la tabella;
	\end{itemize}
	\item \textbf{Postcondizioni:} 
	\begin{itemize}
		\item Il nome del campo viene inserito;
	\end{itemize}
	\item \textbf{Scenario principale:} 
	\begin{itemize}
		\item L’amministratoer inserisce il nome del campo nella casella di testo dedicata;
	\end{itemize}
	\item \textbf{Estensioni:} nel caso in cui il nome inserito sia già esistente:
	\begin{itemize}
		\item \hyperref[sec:UC14.1]{\textbf{UC14.1}} - Errore nome campo già esistente
	\end{itemize}
\end{itemize}

\subsubsection{UC13.2 - Inserimento tipo campo}
\label{sec:UC13.2}
%\includegraphics[]{diagramma_UML}
\begin{itemize}
	\item \textbf{Descrizione:} l’amministratore vuole selezionare il tipo del campo da inserire nella tabella;
	\item \textbf{Attori:} amministratore;
	\item \textbf{Precondizioni:} 
	\begin{itemize}
		\item L’amministratore ha effettuato il login (\hyperref[sec:UC1]{\textbf{UC1}});
		\item L’amministratore si trova nel pannello amministrativo;
		\item L’amministratore si trova nella sezione di inserimento della struttura database;
		\item L’amministratore sta inserendo i campi che compongono la tabella;
	\end{itemize}
	\item \textbf{Postcondizioni:} 
	\begin{itemize}
		\item Il tipo del campo viene selezionato;
	\end{itemize}
	\item \textbf{Scenario principale:} 
	\begin{itemize}
		\item L’amministratore sceglie il tipo di campo, selezionandolo dalle scelte possibili;
	\end{itemize}
	\item \textbf{Estensioni:} nel caso in cui il tipo non venga selezionato:
	\begin{itemize}
		\item \hyperref[sec:UC14.1]{\textbf{UC14.1}} - Errore tipo campo non selezionato
	\end{itemize}
\end{itemize}

\subsubsection{UC13.3 - Inserimento sinonimi campo}
\label{sec:UC13.3}

\subsection{UC14 - Errore creazione campo}
\label{sec:UC14}

\subsubsection{UC14.1 - Errore nome campo già esistente}
\label{sec:UC14.1}

\subsubsection{UC14.2 - Errore tipo campo non selezionato}
\label{sec:UC14.2}

\subsubsection{UC14.3 - Errore mancato inserimento sinonimi campo}
\label{sec:UC14.3}

\subsection{UC15 - Visualizzazione campo tabella}
\label{sec:UC15}

\subsection{UC16 - Modifica campo tabella}
\label{sec:UC16}

\subsection{UC17 - Eliminazione campo tabella}
\label{sec:UC17}

\subsection{UC18 - Logout}
\label{sec:UC18}
%\includegraphics[]{diagramma_UML}
\begin{itemize}
	\item \textbf{Descrizione: l'amministratore vuole fare il logout dall'area amministrativa} 
	\item \textbf{Attori:} amministratore;
	\item \textbf{Precondizioni:} 
	\begin{itemize}
		\item L’amministratore ha effettuato il login (\hyperref[sec:UC1]{\textbf{UC1}});
	\end{itemize}
	\item \textbf{Postcondizioni:} 
	\begin{itemize}
		\item Viene visualizzata la pagina iniziale;
	\end{itemize}
	\item \textbf{Scenario principale:} 
	\begin{itemize}
		\item L’amministratore clicca il pulsante di logout;
		\item L’amministratore viene reindirizzato alla pagina iniziale.
	\end{itemize}
	\item \textbf{Estensioni:} nel caso in cui l'utente non fosse loggato:
	\begin{itemize}
		\item Viene visualizzato un errore - \hyperref[sec:UC19]{\textbf{UC19}}.
	\end{itemize}
\end{itemize}

\subsection{UC19 - Errore logout (not logged in)}
\label{sec:UC19}

\subsection{UC20 - Selezione database da interrogare}
\label{sec:UC20}
%\includegraphics[]{diagramma_UML}
\begin{itemize}
	\item \textbf{Descrizione:} l’utente seleziona il database che vuole interrogare da una lista;
	\item \textbf{Attori:} utente;
	\item \textbf{Precondizioni:}
	\begin{itemize}
		\item Sono presenti una o più strutture database da poter selezionare;
	\end{itemize}
	\item \textbf{Postcondizioni:}
	\begin{itemize}
		\item L’utente ha selezionato una struttura database e può procedere con \hyperref[sec:UC2]{\textbf{UC2}};
	\end{itemize}
	\item \textbf{Scenario principale:}
	\begin{itemize}
		\item L’utente ha il programma aperto;
		\item L’utente seleziona uno dei database presenti nella lista.
	\end{itemize}
\end{itemize}

\subsection{UC21 - Errore DB non selezionato}
\label{sec:UC21}

\subsection{UC22 - Inserimento frase in linguaggio naturale}
\label{sec:UC22}
%\includegraphics[]{diagramma_UML}
\begin{itemize}
	\item \textbf{Descrizione:} L’utente scrive una frase in linguaggio naturale nella casella di testo disponibile nell’interfaccia e ne conferma l’inserimento;
	\item \textbf{Attori:} utente;
	\item \textbf{Precondizioni:} 
	\begin{itemize}
		\item L’utente ha selezionato un file di struttura Database (\hyperref[sec:UC20]{\textbf{UC20}});
	\end{itemize}
	\item \textbf{Postcondizioni:} 
	\begin{itemize}
		\item L’utente riceve un prompt per la creazione della query richiesta (\hyperref[sec:UC24]{\textbf{UC24}});
	\end{itemize}
	\item \textbf{Scenario principale:} 
	\begin{itemize}
		\item L’utente ha il programma aperto;
		\item L’utente seleziona la casella di testo;
		\item L’utente scrive la frase per interrogare il database;
		\item L’utente clicca il pulsante apposito per ottenere il prompt.
	\end{itemize}
	\item \textbf{Estensioni:} nel caso in cui venga inserita una frase non inerente al database, o non comprensibile:
	\begin{itemize}
		\item Viene visualizzato un errore - \hyperref[sec:UC23]{\textbf{UC23}}.
	\end{itemize}
\end{itemize}

\subsection{UC23 - Errore frase naturale}
\label{sec:UC23}

\subsubsection{UC23.1 - Errore frase non inserita}
\label{sec:UC23.1}

\subsubsection{UC23.2 - Errore frase non compresa}
\label{sec:UC23.2}

\subsubsection{UC23.3 - Errore frase non inerente}
\label{sec:UC23.3}
%\includegraphics[]{diagramma_UML}
\begin{itemize}
	\item \textbf{Descrizione:} l’utente inserisce una frase in linguaggio naturale non interpretabile dal sistema come inerente al database;
	\item \textbf{Attori:} utente;
	\item \textbf{Precondizioni:} 
	\begin{itemize}
		\item L’utente ha selezionato un file di struttura Database (\hyperref[sec:UC20]{\textbf{UC20}});
		\item L’utente ha scritto una frase in linguaggio naturale nella casella di testo apposita e ne ha confermato l’inserimento (\hyperref[sec:UC22]{\textbf{UC22}});
	\end{itemize}
	\item \textbf{Postcondizioni:} 
	\begin{itemize}
		\item Il programma visualizza un messaggio di errore;
		\item Il prompt non viene generato;
	\end{itemize}
	\item \textbf{Scenario principale:} 
	\begin{itemize}
		\item L’utente ha il programma aperto;
		\item L’utente ha selezionato il database(\hyperref[sec:UC20]{\textbf{UC20}}) e inserito la frase(\hyperref[sec:UC22]{\textbf{UC22}});
		\item Il programma visualizza un messaggio di errore.
	\end{itemize}
\end{itemize}

\subsection{UC24 - Visualizzazione prompt generato}
\label{sec:UC24}
%\includegraphics[]{diagramma_UML}
\begin{itemize}
	\item \textbf{Descrizione:} L’utente riceve il prompt per la generazione della query;
	\item \textbf{Attori:} utente;
	\item \textbf{Precondizioni:} 
	\begin{itemize}
		\item L’utente ha selezionato un file di struttura Database (\hyperref[sec:UC20]{\textbf{UC20}});
		\item L’utente ha scritto una frase in linguaggio naturale nella casella di testo apposita e ne ha confermato l’inserimento (\hyperref[sec:UC22]{\textbf{UC22}});
	\end{itemize}
	\item \textbf{Postcondizioni:} 
	\begin{itemize}
		\item Il programma visualizza il prompt per la creazione della query richiesta;
	\end{itemize}
	\item \textbf{Scenario principale:} 
	\begin{itemize}
		\item L’utente ha il programma aperto;
		\item L’utente ha selezionato il database(\hyperref[sec:UC20]{\textbf{UC20}}) e inserito la frase(\hyperref[sec:UC22]{\textbf{UC22}});
		\item L’utente preme il bottone per ottenere il prompt;
		\item Il programma visualizza il prompt elaborato.
	\end{itemize}
\end{itemize}

\subsection{UC25 - Errore generazione prompt}
\label{sec:UC25}

\subsubsection{UC25.1 - Errore comunicazione con LLM}
\label{sec:UC25.1}

\subsubsection{UC25.2 - Errore dati mancanti per creazione prompt}
\label{sec:UC25.2}

\subsection{UC26 - Richiesta generazione query SQL}
\label{sec:UC26}

\subsection{UC27 - Errore generazione query SQL}
\label{sec:UC27}

\subsubsection{UC27.1 - Errore comunicazione con API}
\label{sec:UC27.1}

\subsubsection{UC27.2 - Errore formattazione prompt}
\label{sec:UC27.2}

\subsection{UC28 - Visualizzazione query SQL}
\label{sec:UC28}

%\subsection{UC - }
%\label{sec:UC}

\section{Requisiti}
\subsection{Requisiti funzionali}
\subsection{Requisiti di qualità}
\subsection{Requisiti di vincolo}
\subsection{Tracciamento}


\end{document}