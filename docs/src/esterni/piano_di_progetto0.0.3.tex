\documentclass[5pt]{article}

\usepackage{sectsty}
\usepackage{graphicx}
\usepackage{lipsum} % for generating dummy text
\usepackage[margin=1in]{geometry}
\usepackage{setspace}
\usepackage{array}
\usepackage{cellspace}
\usepackage{tabularx}


\usepackage{hyperref}
\usepackage{scrextend}
\graphicspath{ {../assets} }


% Margins
\topmargin=-0.45in
\evensidemargin=0in
\oddsidemargin=0in
\textwidth=6.5in
\textheight=9.0in
\headsep=0.25in

\title{Piano di Progetto}
\author{Jackpot Coding}
\renewcommand*\contentsname{Indice}
\date{\today}

%STARTOF THE DOCUMENT
\begin{document}

%-------------------------

% Reduce top margin only on the first page
\newgeometry{top=0.5in}

%UNIPD LOGO
    \vspace{8pt}
    \includegraphics[scale=0.2]{UNIPDFull.png}
%END UNIPD LOGO

\vspace{30pt}

%COURSE INFO
\begin{minipage}[t]{0.48\textwidth}
    %COURSE TITLE
        \begin{flushleft}
            Informatica\\
            \vspace{5pt}
            \textbf{\LARGE Ingegneria del Software}\\
            Anno Accademico: 2023/2024
        \end{flushleft}
    %END COURSE TITLE
\end{minipage}
%END COURSE INFO


\vspace{5px}


%BLACK LINE
\rule{\textwidth}{5pt}

%JACKPOT CODING INFO
\begin{minipage}[t]{0.50\textwidth}
    %LOGO JACKPOT CODING
    \begin{flushleft}
        \hspace{10pt}
        \includegraphics[scale=0.65]{jackpot-logo.png} 
    \end{flushleft}
\end{minipage}
\hspace{-60pt} % This adds horizontal space between the minipages
\begin{flushright}
    \begin{minipage}[t]{0.50\textwidth}
        %INFO JACKPOT CODING
        \begin{flushright}
            Gruppo: {\Large Jackpot Coding}\\
            Email: \href{mailto:jackpotcoding@gmail.com}{jackpotcoding@gmail.com}
        \end{flushright}
    \end{minipage}
\end{flushright}
%END JACKPOT CODING INFO

\vspace{24pt}

%TITLE
\begin{center}
    \textbf{\LARGE PIANO DI PROGETTO}
\end{center}
%END TITLE

\vspace{13pt}

\begin{flushleft}
    \begin{spacing}{1.5}
        REDATTORE: M. Gobbo, M. Favaretto\\%INSERT HERE THE NAMES
        VERIFICATORI: \\
        \vspace{7pt}
        DESTINATARI: Prof. T. Vardanega, Prof. R. Cardin\\%INSERT HERE THE NAMES
    \end{spacing}
\end{flushleft}

\begin{flushright}
    \begin{spacing}{1}
        USO: ESTERNO\\
        VERSIONE: 0.0.1\\
    \end{spacing}
\end{flushright}


% Restore original margins from the second page onwards
\restoregeometry

\pagebreak

\textbf{\Large Registro delle modifiche}
\begin{table}[ht]
\centering
\begin{tabular}{|c|c|c|c|c|}
\hline
\textbf{Versione} & \textbf{Data} & \textbf{Autore} & \textbf{Verificatore} & \textbf{Modifica} \\
\hline
v0.0.3 & 08/02/2024 & M. Gobbo - M. Favaretto & - & Scrittura sezione 2 \\
\hline
v0.0.2 & 05/02/2024 & M. Gobbo - M. Favaretto & - & Scrittura sezione 1 \\
\hline
V0.0.1 & 03/02/2023 & M. Gobbo - M. Favaretto & - & Creata struttura del documento \\
\hline
\end{tabular}
\caption{Cronologia delle modifiche}
\label{tab:conference}
\end{table}

\pagebreak
\tableofcontents
\pagebreak

\section{Introduzione}
\subsection{Scopo del documento}
Il presente documento ha come fine la presentazione della pianificazione e delle modalità di sviluppo di questo progetto. 
Nella fattispecie verranno esposte le analisi dei possibili rischi con alcune proposte di mitigazione, il modello adottato dal \textit{team}, 
il preventivo e il consuntivo di periodo.

\subsection{Scopo del capitolato}
Le \textit{I.A.\textsuperscript{G}} stanno vivendo un momento di grande innovazione ed entusiasmo, 
riuscendo a cogliere l'attenzione anche di utenti al di fuori dell'ambito informatico lavorativo e accademico, 
si veda come \textit{ChatGPT\textsuperscript{G}} sia diventato un fenomeno culturale. \\
La versatilità dell'\textit{I.A.\textsuperscript{G}} è oggi al centro dell'attenzione di molte \textit{software house}, 
poiché posso essere usate anche per migliorare e velocizzare la produzione. 
Per farne buon uso, è però necessario essere in grado di fornire i corretti \textit{prompt\textsuperscript{G}} al 
modello di \textit{I.A.\textsuperscript{G}} in uso. \\


L'obiettivo di questo progetto è realizzare un \textit{software\textsuperscript{G}} in grado di generare 
un \textit{prompt\textsuperscript{G}} a partire da una richiesta in linguaggio naturale. La richiesta dovrà riguardare un'interrogazione 
di un \textit{database} caricato sul sistema dall'utente. Tale \textit{prompt\textsuperscript{G}} sarà successivamente da fornire 
ad un \textit{LLM\textsuperscript{G}}, il quale restituirà all'utente una \textit{query\textsuperscript{G}} nel linguaggio 
\textit{SQL\textsuperscript{G}}.

\subsection{Glossario}
Alcuni termini presenti in questo documento potrebbero generare incomprensioni o necessitare di chiarimenti. 
Al fine di evitare queste eventualità, tali termini sono contrassegnati dalla lettera \textit{G} maiuscola posta ad apice della parola, 
per indicare che la loro spiegazione è presente all'interno del documento \textit{Glossario}.

\subsection{Riferimenti}
\subsubsection{Riferimenti Informativi}
Slide del corso "Inegegneria Del Software"
\begin{itemize}
      \item \href{https://www.math.unipd.it/~tullio/IS-1/2023/Dispense/T3.pdf}{Modelli di sviluppo software}
      \item \href{https://www.math.unipd.it/~tullio/IS-1/2023/Dispense/T4.pdf}{Gestione del progetto}
      \item \href{https://www.math.unipd.it/~tullio/IS-1/2023/Dispense/T5.pdf}{Analisi dei requisiti} 
\end{itemize}
\subsubsection{Riferimenti Normativi}
\begin{itemize}
      \item Documento "Norme di progetto"
      \item \href{https://www.math.unipd.it/~tullio/IS-1/2023/Dispense/PD2.pdf}{Regolamento del progetto}
      \item \href{https://www.math.unipd.it/~tullio/IS-1/2023/Progetto/C9.pdf}{Capitolato C9 "ChatSQL"} 
\end{itemize}

\section{Analisi dei rischi}
La realizzazione di un progetto \textit{software} comporta una serie di rischi che devono essere valutati e gestiti attentamente al fine di garantire il successo del prodotto finale. Nel corso di questo paragrafo, esamineremo i rischi che siamo riusciti ad individuare e le strategie per mitigarli efficacemente. \\[1\baselineskip]
Per facilitare l'analisi, ogni rischio avrà la seguente struttura:
\begin{itemize}
  \item \textbf{Codice}: Identifica univocamente il rischio, che può essere di 3 tipi:
      \begin{itemize}
      \item \textbf{RT}: Rischi tecnologici
      \item \textbf{RO}: Rischi organizzativi
      \item \textbf{RC}: Rischi comunicativi
      \end{itemize}
  \item \textbf{Titolo}: Breve titolo descrittivo del rischio.
  \item \textbf{Descrizione}: Spiegazione dettagliata del rischio.
  \item \textbf{Occorrenza}: Frequenza con cui il rischio si verifica.
  \item \textbf{Pericolosità}: Livello di pericolo associato al rischio.
  \item \textbf{Mitigazione}: Azioni per ridurre l'impatto o la probabilità del rischio.
\end{itemize}

\subsection{Rischi tecnologici}
\textbf{RT-1} \\
\textbf{Inesperienza nello sviluppo tecnologico}
\begin{itemize}
  \item \textbf{Descrizione}: La mancanza di esperienza nel gruppo su una determinata tecnologia o piattaforma potrebbe rallentare lo sviluppo del \textit{software} e portare a errori nella progettazione e nell'implementazione.
  \item \textbf{Occorrenza}: Media
  \item \textbf{Pericolosità}: Media
  \item \textbf{Mitigazione}: 
    \begin{itemize}
      \item Assicurarsi che i membri del gruppo ricevano formazione e supporto adeguati sulla tecnologia o piattaforma specifica.
      \item Affidare compiti critici a membri del gruppo con esperienza nel settore.
      \item Effettuare una pianificazione accurata per consentire più tempo per l'apprendimento e l'adattamento.
      \end{itemize}
\end{itemize}
\textbf{RT-2} \\
\textbf{Inesperienza strumenti \textit{software}}
\begin{itemize}
  \item \textbf{Descrizione}: Il gruppo non ha familiarità con l'utilizzo di\textit{software} specifici per la gestione di un progetto. Questa mancanza di esperienza potrebbe rallentare la produttività del gruppo e compromettere la qualità del progetto.
  \item \textbf{Occorrenza}: Bassa
  \item \textbf{Pericolosità}: Media
  \item \textbf{Mitigazione}: 
    \begin{itemize}
      \item  Attivarsi per avere una formazione adeguata sull'uso del software gestionali attraverso tutorial e documentazione.
      \item Controllare se gli strumenti scelti dispongono di una solida guida utente e documentazione varia
    \end{itemize}
\end{itemize}


\subsection{Rischi interni}
\subsubsection{Rischi organizzativi}
\textbf{RO-1}\\
\textbf{Organizzazione carente}
\begin{itemize}
  \item \textbf{Descrizione}: Una struttura organizzativa inefficiente o una mancanza di chiarezza nei ruoli e nelle responsabilità possono portare a ritardi nelle decisioni, duplicazioni di compiti e confusione all'interno del gruppo.
  \item \textbf{Occorrenza}: Alta
  \item \textbf{Pericolosità}: Media
  \item \textbf{Mitigazione}: 
    \begin{itemize}
      \item Definire chiaramente le responsabilità di ciascun membro del gruppo e comunicarle in modo trasparente a tutti.
      \item Condurre regolarmente riunioni di coordinamento per monitorare lo stato del progetto e risolvere eventuali problemi organizzativi in modo tempestivo.
    \end{itemize}
\end{itemize}
\textbf{RO-2}\\
\textbf{Distribuzione disomogenea dei compiti}
\begin{itemize}
  \item \textbf{Descrizione}: Una distribuzione disomogenea dei compiti all'interno del gruppo potrebbe portare a una mancanza di lavoro per alcuni componenti e sovraccarico di lavoro per altri, compromettendo l'efficienza complessiva del progetto.
  \item \textbf{Occorrenza}: Media
  \item \textbf{Pericolosità}: Alta
  \item \textbf{Mitigazione}: 
    \begin{itemize}
      \item Pianificare attentamente l'allocazione dei compiti, tenendo conto delle competenze, delle disponibilità di ciascun membro del gruppo.
      \item Promuovere una cultura di collaborazione e flessibilità, incoraggiando il supporto reciproco tra i membri del gruppo e la condivisione dei carichi di lavoro.
      \item Monitorare regolarmente lo stato dei compiti assegnati e intervenire prontamente per ridistribuire le risorse in caso di necessità.
    \end{itemize}
\end{itemize}
\subsubsection{Rischi comunicativi}
\textbf{RC-1}\\
\textbf{Mancanza di comunicazione interna efficace}
\begin{itemize}
  \item \textbf{Descrizione}: La mancanza di una comunicazione interna efficace tra i membri del gruppo può portare a fraintendimenti, duplicazione del lavoro e rallentamenti nel processo decisionale.
  \item \textbf{Occorrenza}: Bassa
  \item \textbf{Pericolosità}: Alta
  \item \textbf{Mitigazione}: 
    \begin{itemize}
      \item Stabilire procedure chiare e canali di comunicazione ben definiti all'interno del gruppo.
      \item Organizzare riunioni regolari per discutere lo stato del progetto, assegnare compiti e condividere aggiornamenti.
      \item Utilizzare strumenti di gestione del progetto e di collaborazione per facilitare la comunicazione e la condivisione delle informazioni come \textit{Trello} (utilizzato come \textit{dashboard\textsuperscript{G}}).
    \end{itemize}
\end{itemize}
\textbf{RC-2}\\
\textbf{Problemi di comunicazione con l'azienda}
\begin{itemize}
  \item \textbf{Descrizione}: La comunicazione inefficace con l'azienda può portare a malintesi, ritardi nelle consegne e insoddisfazione del cliente, rallentando inoltre l'attività del gruppo.
  \item \textbf{Occorrenza}: Bassa
  \item \textbf{Pericolosità}: Alta
  \item \textbf{Mitigazione}: 
    \begin{itemize}
      \item Stabilire canali di comunicazione chiari e definire ruoli e responsabilità per l'interazione con l'azienda.
      \item Fornire aggiornamenti regolari e trasparenti sullo stato del progetto.
      \item Ascoltare attivamente i \textit{feedback} dell'azienda e risolvere eventuali problemi di comunicazione tempestivamente.
    \end{itemize}
\end{itemize}
\textbf{RC-3}\\
\textbf{Tensioni e conflitti interni}
\begin{itemize}
  \item \textbf{Descrizione}: La presenza di tensioni, conflitti o mancanza di collaborazione tra i membri del gruppo può compromettere l'efficienza e la coesione del gruppo, influenzando negativamente la qualità del lavoro e il raggiungimento degli obiettivi del progetto.
  \item \textbf{Occorrenza}: Media
  \item \textbf{Pericolosità}: Media
  \item \textbf{Mitigazione}: 
    \begin{itemize}
      \item Utilizzare la votazione per prendere decisioni oggettive basate sulla maggioranza delle preferenze. Assicurarsi che tutte le parti coinvolte comprendano e accettino il risultato della votazione per favorire la collaborazione e il senso di appartenenza al gruppo.
      \item Stabilire procedure per la gestione dei conflitti e incoraggiare il coinvolgimento di un mediatore neutrale (arbitro) se necessario.
      \item Utilizzare il metodo del dibattito per esaminare i punti di vista contrastanti in modo strutturato e razionale
    \end{itemize}
\end{itemize}

\section{Modello di Sviluppo}
\subsection{Modello (INSERIRE NOME MODELLO SCELTO)}
\subsection{Motivazione}

\section{Pianificazione}
\subsection{Verso la RTB}
\subsection{Verso la PB}
\subsection{Verso la CA}

\section{Preventivo}
\subsection{Periodo RTB}
\subsection{Periodo PB}
\subsection{Periodo CA}

\section{Consuntivo}
\subsection{Periodo RTB}


\end{document}
