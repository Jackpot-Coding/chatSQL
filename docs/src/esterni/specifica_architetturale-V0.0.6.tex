

\documentclass[5pt]{article}

\usepackage{sectsty}
\usepackage{graphicx}
\usepackage{lipsum} % for generating dummy text
\usepackage[margin=1in]{geometry}
\usepackage{setspace}
\usepackage{array}
\usepackage{cellspace}
\usepackage{tabularx}
\usepackage[table]{xcolor}
\usepackage{tabularray}
\usepackage{pgfplots}


\usepackage{hyperref}
\usepackage{scrextend}
\graphicspath{ {../assets} }

\usepackage[italian]{babel}


% package and setup for tables
\usepackage{float}
\usepackage[table]{xcolor}
\renewcommand{\arraystretch}{1.5}
\arrayrulecolor{black}


% Margins
\topmargin=-0.45in
\evensidemargin=0in
\oddsidemargin=0in
\textwidth=6.5in
\textheight=9.0in
\headsep=0.25in

\setlength{\parindent}{0pt}

\title{Piano di Qualifica}
\author{Jackpot Coding}
\renewcommand*\contentsname{Indice}
\date{\today}
%STARTOF THE DOCUMENT
\begin{document}
	
	%-------------------------
	
	% Reduce top margin only on the first page
	\newgeometry{top=0.5in}
	
	%UNIPD LOGO
	\vspace{8pt}
		\includegraphics[scale=0.2]{UNIPDFull.png}
	%END UNIPD LOGO
	
	\vspace{30pt}
	
	%COURSE INFO
	\begin{minipage}[t]{0.48\textwidth}
		%COURSE TITLE
		\begin{flushleft}
			Informatica\\
			\vspace{5pt}
			\textbf{\LARGE Ingegneria del Software}\\
			Anno Accademico: 2023/2024
		\end{flushleft}
		%END COURSE TITLE
	\end{minipage}
	%END COURSE INFO
	
	
	\vspace{5px}
	
	
	%BLACK LINE
	\rule{\textwidth}{5pt}
	
	%JACKPOT CODING INFO
	\begin{minipage}[t]{0.50\textwidth}
		%LOGO JACKPOT CODING
		\begin{flushleft}
			\hspace{10pt}
				\includegraphics[scale=0.65]{jackpot-logo.png} 
		\end{flushleft}
	\end{minipage}
	\hspace{-60pt} % This adds horizontal space between the minipages
	\begin{flushright}
		\begin{minipage}[t]{0.50\textwidth}
			%INFO JACKPOT CODING
			\begin{flushright}
				Gruppo: {\Large Jackpot Coding}\\
				Email: \href{mailto:jackpotcoding@gmail.com}{jackpotcoding@gmail.com}
			\end{flushright}
		\end{minipage}
	\end{flushright}
	%END JACKPOT CODING INFO
	
	\vspace{24pt}
	
	%TITLE
	\begin{center}
		\textbf{\LARGE SPECIFICA ARCHITETTURALE}
	\end{center}
	%END TITLE
	
	\vspace{13pt}
	
	\begin{flushleft}
		\begin{spacing}{1.5}
			DESTINATARI: Prof. T. Vardanega, Prof. R. Cardin\\%INSERT HERE THE NAMES
		\end{spacing}
	\end{flushleft}
	
	\begin{flushright}
		\begin{spacing}{1}
			USO: ESTERNO\\
			VERSIONE: 0.0.6\\
		\end{spacing}
	\end{flushright}
	
	
	% Restore original margins from the second page onwards
	\restoregeometry
	
	\pagebreak
	
	\textbf{\Large Registro delle modifiche}
	\begin{table}[H]
		\centering
		\rowcolors{2}{black!15}{}
		\resizebox{\linewidth}{!}{
			\begin{tabular}{|c|c|c|c|c|}
				\rowcolor{teal!50}
				\hline
				\textbf{Versione} & \textbf{Data} & \textbf{Autore} & \textbf{Verificatore} & \textbf{Modifica} \\
				\hline
				0.0.7 & 06/04/2024 & M. Favaretto & - & Aggiunta sezione \textit{Design pattern} utilizzati - \textit{M.V.T.} \\ 
    			\hline
    			0.0.6 & 05/04/2024 & E. Gallo & M. Favaretto & Aggiunta sezione Design pattern utilizzati - Strategy \\
    			\hline
				0.0.5 & 04/04/2024 & M. Camillo & E. Gallo & Introduzione Documento \\
                \hline
				0.0.4 & 04/04/2024 & M. Gobbo & E. Gallo & Introduzione all'architettura \\
				\hline
				0.0.3 & 03/04/2024 & G. Moretto & M. Gobbo & Aggiunto elenco delle tabelle \\
				\hline
				0.0.2 & 02/04/2024 & G. Moretto & M. Gobbo & Aggiunta tabelle tecnologie codifica e testing \\
				\hline
				0.0.1 & 27/03/2024 & G. Moretto & M. Gobbo & Aggiunta struttura documento \\
				\hline
			\end{tabular}
		}
		\label{tab:conference}
	\end{table}
	
	\pagebreak
	\tableofcontents
	\pagebreak
	\textbf{\Large Elenco delle immagini} \\
	
	\pagebreak
	\textbf{\Large Elenco delle tabelle} \\
	\makeatletter
	\@starttoc{lot}% Print List of Tables
	\makeatother
	\pagebreak
	
	\section{Introduzione}
	
	\subsection{Scopo del Documento}

    Il documento ha lo scopo di presentare e motivare le scelte architetturali e di design di applicate al prodotto, oltre che a fornire una lista completa delle tecnologie utilizzate. La struttura interna del prodotto è esposta all'interno del documento sotto forma di diagramma delle classi, in maniera da rendere più chiaro il software sviluppato. Vengono inoltre approfonditi e motivati a loro volta i design pattern utilizzati. 
	
	\subsection{Scopo del Prodotto}
    Il capitolato proposto dall'azienda \textit{Zucchetti S.p.A.} ha come obiettivo la realizzazione di un applicativo web al fine di studiare la fattibilità di un prodotto in grado di elaborare una frase in linguaggio naturale. Questa frase, anche se fornita da un utente inesperto, deve generare come output una \textit{query SQL} in grado di interrogare un database (di cui è conosciuta la struttura) in modo efficiente e affidabile.
	
	\subsection{Glossario}
    Al fine di evitare ambiguità o incomprensioni relative alla terminologia usata all'interno del documento, è fornito un \textit{Glossario} in cui vengono riportate definizioni precise per ogni termine potenzialmente ambiguo. La presenza di tali termini all'interno del documento è indicata con la presenza, vicino alla voce, di una \textit{G} in apice ($^G$). 
	\subsection{Riferimenti}
	
	\section{Tecnologie}
	
	\subsection{Codifica}

	\begin{longtblr}[
			caption = {Tecnologie di codifica.},
		]
		{
			colspec={|Q[0.15\linewidth]|Q[0.15\linewidth]|Q[0.70\linewidth]|},
			rows={halign=l},
			column{1}={halign=c},
			column{2}={halign=c},
			column{3}={halign=l},
			row{1}={halign=c},
			row{odd} = {gray!20},
			row{1}={teal!50},
			row{2}={teal!50},
			row{6}={teal!50},
			row{10}={teal!50}
		}
	
		\hline
		\textbf{Tecnologia} & \textbf{Versione} & \textbf{Descrizione} \\
		\hline
		\SetCell[c=3]{c} \textbf{Linguaggi} \\
		\hline
		HTML & 5 & Linguaggio di \textit{markup} utilizzato per la definizione della struttura di pagine \textit{web} \\
		\hline
		CSS & 3 & Linguaggio utilizzato per applicare stile a elementi presenti in una pagina \textit{HTML} \\
		\hline
		Python & 3.11.8 & Linguaggio di programmazione ad alto livello, orientato agli oggetti. Viene utilizzato per la creazione del \textit{server}. \\
		\hline
		\SetCell[c=3]{c} \textbf{Framework e Librerie} \\
		\hline
		Django & 5.0.3 & \textit{Framework} per la creazione di applicazioni \textit{web} scritto in linguaggio \textit{Python}. \\
		\hline
		TensorFlow & 2.15.0 & Libreria \textit{Python} per l'apprendimento automatico. \\
		\hline
		Transformers & 4.29.3 & Libreria \textit{Python} per l'utilizzo di modelli del portale \textit{Hugging Face} utilizzando \textit{TensorFlow}\\
		\hline
		\SetCell[c=3]{c} \textbf{Strumenti} \\
		\hline
		Pip & 24.0 & Strumento per la gestione dei pacchetti utilizzati da applicazioni \textit{Python}.\\
		\hline
		Git & 2.44.0 & Strumento per il controllo di versione utilizzato per la gestione della \textit{repository} remota presente su \textit{GitHub}. \\
		\hline
	\end{longtblr}
	
	\subsection{Testing}
	\begin{longtblr}[
		caption = {Tecnologie di testing.},
		]
		{
			colspec={|Q[0.15\linewidth]|Q[0.15\linewidth]|Q[0.70\linewidth]|},
			rows={halign=l},
			column{1}={halign=c},
			column{2}={halign=c},
			column{3}={halign=l},
			row{1}={halign=c},
			row{odd} = {gray!20},
			row{1}={teal!50},
			row{2}={teal!50},
			row{7}={teal!50}
		}
		\hline
		\textbf{Tecnologia} & \textbf{Versione} & \textbf{Descrizione} \\
		\hline
		\SetCell[c=3]{c} \textbf{Framework e Librerie} \\
		\hline
		Unittest & 3.11.8 & \textit{Framework} incluso nel linguaggio \textit{Python} utilizzato per il \textit{testing} di unità, utilizzato dal \textit{framework Django}.\\
		\hline
		Django Test Client & 5.0.3 & \textit{Client} per il \textit{testing} di un applicazione \textit{web} simulando un \textit{browser}, integrato nel \textit{framework Django}.\\
		\hline
		coverage.py & 7.4.4 & \textit{Tool} per misurare il \textit{code coverage} in applicazioni \textit{Python} integrabile nel \textit{framework Django}. \\
		\hline
		Prospector & 1.10.3 & \textit{Tool} per l'analisi statica di codice scritto nel linguaggio \textit{Python}. \\
		\hline
		\SetCell[c=3]{c} \textbf{Strumenti} \\
		\hline
		GitHub Actions & - & Servizio di \textit{Github} per la \textit{Continuous Integration}, automatizza i processi di \textit{build, test e deploy} del prodotto \textit{software}.\\
		\hline
	\end{longtblr}
	
	\section{Architettura}
	
	\subsection{Introduzione}
L'architettura di \textit{"Chat SQL"} si basa sul \textit{design pattern} architetturale \textit{MVT(Model View Template)}.\\
In questo modello, il \textit{"Model"} rappresenta i dati e la logica di \textit{business} dell'applicazione, la \textit{"View"} è responsabile della presentazione dei dati all'utente e il \textit{"Template"} definisce la struttura della \textit{View} e come i dati vengono visualizzati al suo interno. Più nello specifico il \textit{"Template"}, nel caso della nostra applicazione, definisce la struttura \textit{HTML}.\\
Questo \textit{design pattern} è simile al classico \textit{MVC(Model View Controller)} soltanto che al posto della parte di \textit{Controller} c'è il \textit{Template}.\\\\
E' stato scelta questa architettura in quanto offre i seguenti vantaggi:
\begin{itemize}
    \item \textbf{Separazione delle responsabilità}: \textit{MVT} separa chiaramente le responsabilità tra \textit{Model}, \textit{View} e \textit{Template}. Questo permette una migliore organizzazione del codice, facilitando la manutenzione e la scalabilità dell'applicazione.

    \item \textbf{Riutilizzo dei template}: I \textit{template} in \textit{MVT} consentono di separare la presentazione dalla logica di \textit{business}. Ciò facilita il riutilizzo dei componenti di interfaccia utente in diverse parti dell'applicazione, riducendo la duplicazione del codice e migliorando l'efficienza dello sviluppo.

    \item \textbf{Aumento delle Performance}: Utilizzare i \textit{template} pre-renderizzati può migliorare le prestazioni dell'applicazione rispetto a un'architettura \textit{MVC} tradizionale, poiché il \textit{rendering} dei \textit{template} può essere più efficiente rispetto ad esempio alla generazione dinamica di \textit{HTML} lato \textit{server}.

\end{itemize}
	
	\subsection{Diagrammi delle classi}
	
	\subsection{Design pattern utilizzati}
		\subsubsection{Model-View-Template}
			% eventuale immagine mvt
			Per lo sviluppo del prodotto, il gruppo ha scelto l'utilizzo del \textit{framework Django}. Il \textit{framework} propone un'architettura integrata,
			basata su una generalizzazione della \textit{view}, attraverso il \textit{design pattern MVT}. \\
			L'architettura proposta da \textit{Django} si compone di:
			\begin{itemize}
				\item diversi \textit{file} di impostazioni che si occupano di mettere in comunicazione le varie parti dell'architettura e instradare correttamente i vari percorsi della \textit{Web-App}
				\item una cartella contenente i \textit{Template}
				\item una cartella per ogni applicazione che compone il prodotto
			\end{itemize}
			La cartella \textit{templates} conterrà le viste del prodotto. Tali viste - composte principalmente da \textit{file HTML e CSS} - 
			saranno viste generiche, che andranno popolate e arricchite dall'applicazione in uso. \\
			Ogni cartella di applicazione conterrà:
			\begin{itemize}
				\item il rispettivo modello che realizzerà la \textit{business logic} e la \textit{persistence logic}
				\item un file chiamato \textit{view.py} che gestirà la vista della specifica applicazione, basandosi sul \textit{template} richiesto e sui dati forniti ed elaborati dal modello
			\end{itemize}

	\subsubsection{Strategy}
	% Immagine Design Pattern Strategy%
	Uno dei design pattern comportamentali scelto dal gruppo è lo \textit{Strategy}\textsuperscript{G}. In particolare, viene implementato per l'interrogazione e l'integrazione coi modelli LLM\textsuperscript{G}. Il prodotto infatti deve lavorare a stretto contatto con questi modelli linguistici di grandi dimensioni ma non sempre usarne uno solo è vantaggioso. Soprattutto se non si vogliono usare quelli più capaci, che oltre alle notevoli dimensioni sono molto spesso a pagamento, c'è la necessità di interrogare modelli di dimensioni ridotte ma specializzati in una singola funzione. \\
	Nel nostro caso sono due i tipi di interrogazioni agli \textit{LLM} che il gruppo ha individuato.
	\begin{enumerate}
		\item L'interrogazione principale, su cui si basa l'intero progetto: da un \textit{prompt}\textsuperscript{G} restituisce la frase in \textit{SQL}\textsuperscript{G}. \\
		In questo caso ci sono pochi modelli che sono in grado di generare frasi \textit{SQL} perfette e molto spesso tutti i modelli restituiscono un risultato diverso. Risulta quindi necessario poter cambiare \textit{l'LLM} interrogato a seconda della complessità della richiesta e della volontà dell'utente. \\
		Data l'esponenziale crescita dei linguaggi \textit{AI}\textsuperscript{G} negli ultimi anni, è importante che la lista dei diversi modelli utilizzabili sia facilmente integrabile con quelli più capaci e all'avanguardia.
		\item Le interrogazioni ausiliarie, che non formano un requisito obbligatorio ma servono per facilitare l'uso del prodotto all'utente. Un esempio è la generazione e l'inserimento dei sinonimi di tabelle e campi, che di base è lasciata all'utente ma è facilmente integrabile con un \textit{API}\textsuperscript{G} o un \textit{LLM} opportuno per migliorare l'esperienza del prodotto.
	\end{enumerate}
	
\end{document}
