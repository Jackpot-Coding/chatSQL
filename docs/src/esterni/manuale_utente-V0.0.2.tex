

\documentclass[5pt]{article}

\usepackage{sectsty}
\usepackage{graphicx}
\usepackage{lipsum} 
\usepackage[margin=1in]{geometry}
\usepackage{setspace}
\usepackage{array}
\usepackage{cellspace}
\usepackage{tabularx}
\usepackage[table]{xcolor}
\usepackage{tabularray}
\usepackage{pgfplots}


\usepackage{hyperref}
\usepackage{scrextend}
\graphicspath{ {../assets} }


% Margins
\topmargin=-0.45in
\evensidemargin=0in
\oddsidemargin=0in
\textwidth=6.5in
\textheight=9.0in
\headsep=0.25in

\setlength{\parindent}{0pt}

\title{Piano di Qualifica}
\author{Jackpot Coding}
\renewcommand*\contentsname{Indice}
\date{\today}

%STARTOF THE DOCUMENT
\begin{document}
	
	%-------------------------
	
	% Reduce top margin only on the first page
	\newgeometry{top=0.5in}
	
	%UNIPD LOGO
	\vspace{8pt}
	\includegraphics[scale=0.2]{UNIPDFull.png}
	%END UNIPD LOGO
	
	\vspace{30pt}
	
	%COURSE INFO
	\begin{minipage}[t]{0.48\textwidth}
		%COURSE TITLE
		\begin{flushleft}
			Informatica\\
			\vspace{5pt}
			\textbf{\LARGE Ingegneria del Software}\\
			Anno Accademico: 2023/2024
		\end{flushleft}
		%END COURSE TITLE
	\end{minipage}
	%END COURSE INFO
	
	
	\vspace{5px}
	
	
	%BLACK LINE
	\rule{\textwidth}{5pt}
	
	%JACKPOT CODING INFO
	\begin{minipage}[t]{0.50\textwidth}
		%LOGO JACKPOT CODING
		\begin{flushleft}
			\hspace{10pt}
			\includegraphics[scale=0.65]{jackpot-logo.png} 
		\end{flushleft}
	\end{minipage}
	\hspace{-60pt} % This adds horizontal space between the minipages
	\begin{flushright}
		\begin{minipage}[t]{0.50\textwidth}
			%INFO JACKPOT CODING
			\begin{flushright}
				Gruppo: {\Large Jackpot Coding}\\
				Email: \href{mailto:jackpotcoding@gmail.com}{jackpotcoding@gmail.com}
			\end{flushright}
		\end{minipage}
	\end{flushright}
	%END JACKPOT CODING INFO
	
	\vspace{24pt}
	
	%TITLE
	\begin{center}
		\textbf{\LARGE MANUALE UTENTE}
	\end{center}
	%END TITLE
	
	\vspace{13pt}
	
	\begin{flushleft}
		\begin{spacing}{1.5}
			REDATTORI: M. Gobbo\\%INSERT HERE THE NAMES
			VERIFICATORI: \\
			\vspace{7pt}
			DESTINATARI: Prof. T. Vardanega, Prof. R. Cardin\\%INSERT HERE THE NAMES
		\end{spacing}
	\end{flushleft}
	
	\begin{flushright}
		\begin{spacing}{1}
			USO: ESTERNO\\
			VERSIONE: 0.0.2\\
		\end{spacing}
	\end{flushright}
	
	
	% Restore original margins from the second page onwards
	\restoregeometry
	
	\pagebreak
	
	\textbf{\Large Registro delle modifiche}
	\begin{table}[ht]
		\centerline{%
			\begin{tabular}{|c|c|c|c|c|}
                \hline
				v.0.0.2 & 29/03/2024 & M. Gobbo & - & Creazione sezione 5 \\
			\hline
				v.0.0.1 & 28/03/2024 & M. Gobbo & - & Creata struttura del documento \\
				\hline
			\end{tabular}%
		}
		\label{tab:conference}
	\end{table}
	
	
	
	\pagebreak
	\tableofcontents
	\pagebreak
	
	
	
	\section{Introduzione}
	\subsection{Scopo del documento}
	\subsection{Scopo del prodotto}
        \subsection{Glossario}

	\section{Requisiti minimi di sistema}
        \subsection{Requisiti minimi}
        \subsection{Requisiti hardware}
        \subsection{Requisiti browser}
	
	\section{Installazione}
        \subsection{Clonazione Repo}
 
        \section{Istruzioni all'uso}
        
        \section{Supporto Tecnico}
        Nel caso di malfunzionamenti di qualsiasi genere il \textit{team} "\textit{Jackpot Coding}" si rende disponibile presso la seguente \textit{email}:
        \begin{center}
        \textbf{\url{jackpotcoding@gmail.com}}
        \end{center}

        Per la segnalazione di problemi si consiglia di adottare il seguente formato:
\begin{itemize}
    \item \textbf{OGGETTO}: "nome dell'evento da segnalare"
    \item \textbf{CORPO}:
    \begin{itemize}
        \item "Data del Problema"
        \item "Descrizione problema"
        \item "Browser e sistema operativo nei quali si è verificato il problema"
    \end{itemize}
    \item \textbf{ALLEGATO}: "immagine illustrativa del problema" se possibile
\end{itemize}
        
        
        
			
\end{document}
