

\documentclass[5pt]{article}

\usepackage{sectsty}
\usepackage{graphicx}
\usepackage{lipsum} 
\usepackage[margin=1in]{geometry}
\usepackage{setspace}
\usepackage{array}
\usepackage{cellspace}
\usepackage{tabularx}
\usepackage[table]{xcolor}
\usepackage{tabularray}
\usepackage{pgfplots}


\usepackage{hyperref}
\usepackage{scrextend}
\graphicspath{ {../assets} }

% package and setup for tables
\usepackage{float}
\usepackage[table]{xcolor}
\renewcommand{\arraystretch}{1.5}
\arrayrulecolor{black}

% Margins
\topmargin=-0.45in
\evensidemargin=0in
\oddsidemargin=0in
\textwidth=6.5in
\textheight=9.0in
\headsep=0.25in

\setlength{\parindent}{0pt}

\title{Piano di Qualifica}
\author{Jackpot Coding}
\renewcommand*\contentsname{Indice}
\date{\today}

%STARTOF THE DOCUMENT
\begin{document}
	
	%-------------------------
	
	% Reduce top margin only on the first page
	\newgeometry{top=0.5in}
	
	%UNIPD LOGO
	\vspace{8pt}
	\includegraphics[scale=0.2]{UNIPDFull.png}
	%END UNIPD LOGO
	
	\vspace{30pt}
	
	%COURSE INFO
	\begin{minipage}[t]{0.48\textwidth}
		%COURSE TITLE
		\begin{flushleft}
			Informatica\\
			\vspace{5pt}
			\textbf{\LARGE Ingegneria del Software}\\
			Anno Accademico: 2023/2024
		\end{flushleft}
		%END COURSE TITLE
	\end{minipage}
	%END COURSE INFO
	
	
	\vspace{5px}
	
	
	%BLACK LINE
	\rule{\textwidth}{5pt}
	
	%JACKPOT CODING INFO
	\begin{minipage}[t]{0.50\textwidth}
		%LOGO JACKPOT CODING
		\begin{flushleft}
			\hspace{10pt}
			\includegraphics[scale=0.65]{jackpot-logo.png} 
		\end{flushleft}
	\end{minipage}
	\hspace{-60pt} % This adds horizontal space between the minipages
	\begin{flushright}
		\begin{minipage}[t]{0.50\textwidth}
			%INFO JACKPOT CODING
			\begin{flushright}
				Gruppo: {\Large Jackpot Coding}\\
				Email: \href{mailto:jackpotcoding@gmail.com}{jackpotcoding@gmail.com}
			\end{flushright}
		\end{minipage}
	\end{flushright}
	%END JACKPOT CODING INFO
	
	\vspace{24pt}
	
	%TITLE
	\begin{center}
		\textbf{\LARGE MANUALE UTENTE}
	\end{center}
	%END TITLE
	
	\vspace{13pt}

	
	\begin{flushright}
		\begin{spacing}{1}
			USO: ESTERNO\\
			VERSIONE: 0.0.3\\
		\end{spacing}
	\end{flushright}
	
	
	% Restore original margins from the second page onwards
	\restoregeometry
	
	\pagebreak
	
	\textbf{\Large Registro delle modifiche}
	\begin{table}[H]
		\centering
		\rowcolors{2}{black!15}{}
		\resizebox{\linewidth}{!}{
		  \begin{tabular}{|c|c|c|c|c|c|c|c|}
			\hline
			\rowcolor{teal!50}
			\textbf{versione} & \textbf{data} & \textbf{autore} & \textbf{verifica} & \textbf{modifica} \\ \hline
			v.0.0.4 & 25/04/2024 & M. Gobbo & M. Favaretto & Fine Stesura sezione 1, aggiunti riferimenti \\ \hline
			v.0.0.3 & 29/03/2024 & M. Gobbo & M. Favaretto & Prima Stesura sezione 1 \\ \hline
			v.0.0.2 & 29/03/2024 & M. Gobbo & M. Favaretto & Creazione sezione 5 \\ \hline
			v.0.0.1 & 28/03/2024 & M. Gobbo & M. Favaretto & Creata struttura del documento \\ \hline
		  \end{tabular}
		}
		\label{tab:conference}
	\end{table}
	
	
	
	\pagebreak
	\tableofcontents
	\pagebreak
	
	
	
	\section{Introduzione}
	\subsection{Scopo del documento}
		Lo scopo del documento "Manuale Utente" è quello di mostrare le istruzioni per l'uso e le funzionalità inerenti al prodotto. Grazie a questo l'utente sarà a conoscenza dei requisiti minimi necessari per il corretto funzionamento di "Chat SQL".
 
	\subsection{Scopo del prodotto}
     	Il capitolato proposto dall'azienda \textit{Zucchetti S.p.A.} ha come obiettivo la realizzazione di un applicativo web al fine di ottenere un riscontro riguardo la fattibilità di un prodotto in grado di elaborare una frase in linguaggio naturale, fornita da un utente inesperto, e genere come output una \textit{query SQL} in grado di interrogare un database, di cui è conosciuta la struttura, in modo efficiente e che fornisca le informazioni richieste dall'utente.

        \subsection{Glossario}
     		Al fine di evitare ambiguità o incomprensioni relative alla terminologia usata all'interno del documento, è fornito un \textit{Glossario} in cui vengono riportate definizioni precise per ogni termine potenzialmente ambiguo. La presenza di tali termini all'interno del documento è indicata con la presenza, vicino alla voce, di una \textit{G} in apice ($^G$). 

        \subsection{Riferimenti}
        
        \subsubsection{Riferimenti Normativi}
        \begin{itemize}
			\item Capitolato\textsuperscript{G} C9 - \textit{ChatSQL} \\ \url{https://www.math.unipd.it/~tullio/IS-1/2023/Progetto/C9.pdf} 
			\item Norme di progetto\textsuperscript{G} V1.0.0
        \end{itemize}
        
        \subsubsection{Riferimenti informativi}
        \begin{itemize}
         	\item Diagramma delle classi \\ \url{https://www.math.unipd.it/~rcardin/swea/2023/Diagrammi%20delle%20Classi.pdf}
         	\item Analisi dei requisiti: \\
         	\url{https://www.math.unipd.it/~tullio/IS-1/2023/Dispense/T3.pdf}\\
        \end{itemize}
        


	\section{Requisiti minimi di sistema}
        \subsection{Requisiti minimi}
        \subsection{Requisiti hardware}
        \subsection{Requisiti browser}
	
	\section{Installazione}
        \subsection{Clonazione Repo}
 
        \section{Istruzioni all'uso}
        
        \section{Supporto Tecnico}
        	Nel caso di malfunzionamenti di qualsiasi genere il \textit{team} "\textit{Jackpot Coding}" si rende disponibile presso la seguente \textit{email}:
        	\begin{center}
        		\textbf{\url{jackpotcoding@gmail.com}}
        	\end{center}

			Per la segnalazione di problemi si consiglia di adottare il seguente formato:
			\begin{itemize}
				\item \textbf{OGGETTO}: "nome dell'evento da segnalare"
				\item \textbf{CORPO}:
				\begin{itemize}
					\item "Data del Problema"
					\item "Descrizione problema"
					\item "Browser e sistema operativo nei quali si è verificato il problema"
				\end{itemize}
				\item \textbf{ALLEGATO}: "immagine illustrativa del problema" se possibile
			\end{itemize}
        
        
        
			
\end{document}
