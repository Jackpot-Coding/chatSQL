\documentclass[5pt]{article}

\usepackage{sectsty}
\usepackage{graphicx}
\usepackage{lipsum} % for generating dummy text
\usepackage[margin=1in]{geometry}
\usepackage{setspace}
\usepackage{array}
\usepackage{cellspace}
\usepackage{tabularx}
\usepackage[table]{xcolor}
\usepackage{tabularray}


\usepackage{hyperref}
\usepackage{scrextend}
\graphicspath{ {../assets} }


% Margins
\topmargin=-0.45in
\evensidemargin=0in
\oddsidemargin=0in
\textwidth=6.5in
\textheight=9.0in
\headsep=0.25in

\title{Analisi dei Requisiti}
\author{Jackpot Coding}
\renewcommand*\contentsname{Indice}
\date{\today}

%STARTOF THE DOCUMENT
\begin{document}
	
	%-------------------------
	
	% Reduce top margin only on the first page
	\newgeometry{top=0.5in}
	
	%UNIPD LOGO
	\vspace{8pt}
	\includegraphics[scale=0.2]{UNIPDFull.png}
	%END UNIPD LOGO
	
	\vspace{30pt}
	
	%COURSE INFO
	\begin{minipage}[t]{0.48\textwidth}
		%COURSE TITLE
		\begin{flushleft}
			Informatica\\
			\vspace{5pt}
			\textbf{\LARGE Ingegneria del Software}\\
			Anno Accademico: 2023/2024
		\end{flushleft}
		%END COURSE TITLE
	\end{minipage}
	%END COURSE INFO
	
	
	\vspace{5px}
	
	
	%BLACK LINE
	\rule{\textwidth}{5pt}
	
	%JACKPOT CODING INFO
	\begin{minipage}[t]{0.50\textwidth}
		%LOGO JACKPOT CODING
		\begin{flushleft}
			\hspace{10pt}
			\includegraphics[scale=0.65]{jackpot-logo.png} 
		\end{flushleft}
	\end{minipage}
	\hspace{-60pt} % This adds horizontal space between the minipages
	\begin{flushright}
		\begin{minipage}[t]{0.50\textwidth}
			%INFO JACKPOT CODING
			\begin{flushright}
				Gruppo: {\Large Jackpot Coding}\\
				Email: \href{mailto:jackpotcoding@gmail.com}{jackpotcoding@gmail.com}
			\end{flushright}
		\end{minipage}
	\end{flushright}
	%END JACKPOT CODING INFO
	
	\vspace{24pt}
	
	%TITLE
	\begin{center}
		\textbf{\LARGE PIANO DI QUALIFICA}
	\end{center}
	%END TITLE
	
	\vspace{13pt}
	
	\begin{flushleft}
		\begin{spacing}{1.5}
			REDATTORI: G. Moretto\\%INSERT HERE THE NAMES
			VERIFICATORI: \\
			\vspace{7pt}
			DESTINATARI: Prof. T. Vardanega, Prof. R. Cardin\\%INSERT HERE THE NAMES
		\end{spacing}
	\end{flushleft}
	
	\begin{flushright}
		\begin{spacing}{1}
			USO: ESTERNO\\
			VERSIONE: 0.0.3\\
		\end{spacing}
	\end{flushright}
	
	
	% Restore original margins from the second page onwards
	\restoregeometry
	
	\pagebreak
	
	\textbf{\Large Registro delle modifiche}
	\begin{table}[ht]
		\centerline{%
			\begin{tabular}{|c|c|c|c|c|}
				\hline
				v.0.0.3 & 14/03/2024 & G. Moretto & - & Aggiunta sezione Valutazione attività di verifica \\
				\hline
				v.0.0.2 & 13/02/2024 &G. Moretto & -  &  Aggiunte descrizioni test\\
				\hline
				v.0.0.1 & 03/02/2024 &G. Moretto & -  & Creata struttura del documento \\
				\hline
			\end{tabular}%
		}
		\label{tab:conference}
	\end{table}
	
	
	
	\pagebreak
	\tableofcontents
	\pagebreak
	
	% Elenco Figure
	% Elenco Tabelle
	
	\section{Introduzione}

	\subsection{Premessa}
	Questo documento viene modificato durante la durata del progetto ed i suoi contenuti verranno aggiornati in base alle pratiche adottate dal gruppo.
	
	\subsection{Scopo del documento}
	Lo scopo di questo documento è quello di raccogliere:
	\begin{itemize}
		\item Obiettivi della qualità di Prodotto;
		\item Obiettivi della qualità di Processo;
		\item Metodi per la misurazione di questi tramite metriche;
		\item Definizione dei test da effettuare;
		\item Cruscotto\textsuperscript{G} per la visione dello stato del raggiungimento degli obiettivi;
	\end{itemize}
	
	\subsection{Scopo del prodotto}
	Il capitolato proposto dall'azienda Zucchetti manifesta l'esigenza di avere un prodotto \textit{Software}\textsuperscript{G} per la creazione di \textit{prompt}\textsuperscript{G} da fornire ad un modello\textsuperscript{G} \textit{LLM}\textsuperscript{G} per la creazione di \textit{query} SQL\textsuperscript{G} per l'interrogazione di \textit{database}\textsuperscript{G} con struttura nota.
	
	
	\section{Qualità di Prodotto}
	
	\subsection{Obiettivi}
	
	
	\subsection{Metriche}



	\section{Qualità di Processo}
	
	\subsection{Processi primari}
	
	\subsection{Processi di supporto}
	
	\subsection{Processi organizzativi}
	
	
	
	\section{Specifica dei Test}
	
	\subsection{Test di Unità}
	Vengono impiegati per la verifica di unità del \textit{software}\textsuperscript{G}. Come unità si intende una piccola parte del programma che funzioni in maniera autonoma.
	
	\subsection{Test di Integrazione}
	Vengono impiegati per la verifica delle interazioni fra le unità sopra descritte.
	
	\subsection{Test di Sistema}
	Vengono impiegati per verificare che l'esecuzione del sistema soddisfi i requisiti\textsuperscript{G} prestabiliti nel documento di Analisi dei Requisiti.
	
	\subsection{Test di Regressione}
	Vengono impiegati per accertarsi che correzioni o aggiunte effettuate sul specifiche unità non interferiscano sul funzionamento del programma.
	
	\subsection{Test di Accettazione}
	Detto anche collaudo, viene effettuato per verificare che i requisiti\textsubscript{G} utente siano soddisfatti. Questo test viene effettuato assieme al committente\textsubscript{G}.
	
	
	\section{Resoconto attività di verifica}
	
	\section{Valutazioni delle attività di verifica}
	In questa sezione si riportano le valutazioni sulle criticità incontrate nel corso dello svolgimento del progetto e le correzioni applicate ad esse.
		
	\subsection{Organizzazione}
	
	\title{Tabella 1}
	\begin{longtblr}
		{
			colspec={|Q[0.12\linewidth]|Q[0.40\linewidth]|Q[0.08\linewidth]|Q[0.40\linewidth]|},
			rows={halign=l},
			column{1}={halign=c},
			column{3}={halign=c},
			column{4}={halign=c},
			row{1}={halign=c},
			row{odd} = {gray!20},
			row{1}={teal!50},
			caption=Tabella 1
		}
		\hline
		\textbf{Criticità} & \textbf{Descrizione} & \textbf{Gravità} & \textbf{Soluzione} \\
		\hline
		Suddivisione dei compiti & Utilizzando Trello come sistema di \textit{Issue Tracking} non era chiaro come suddividere i compiti  & Media & Si è passati a Jira che permette una netta divisione dei compiti per ruolo e arco temporale \\
		\hline
		Verifica & Nei periodi iniziali del progetto non era chiaro quando effettuare l'attività di verifica & Media & Si è aggiunto uno \textit{step} di verifica da completare prima della terminazione del'attività \\
		\hline
	\end{longtblr}
	
	\subsection{Strumenti utilizzati}
	
	\title{Tabella 2}
	\begin{longtblr}
		{
			colspec={|Q[0.12\linewidth]|Q[0.40\linewidth]|Q[0.08\linewidth]|Q[0.40\linewidth]|},
			rows={halign=l},
			column{1}={halign=c},
			column{3}={halign=c},
			column{4}={halign=c},
			row{1}={halign=c},
			row{odd} = {gray!20},
			row{1}={teal!50},
			caption=Tabella 1
		}
		\hline
		\textbf{Criticità} & \textbf{Descrizione} & \textbf{Gravità} & \textbf{Soluzione} \\
		\hline
		Trello & Utile all'inizio del progetto ma non permette la gestione delle attività con \textit{Scrum} & Media & Si è passati a Jira come \textit{Issue Tracking System} \\
		\hline
		Jira & Molti componenti del gruppo non conoscevano il programma & Bassa & Sono state consultate le guide a disposizione ed il progetto è stato impostato da chi aveva più familiarità con esso\\
		\hline
		
		
	\end{longtblr}
	
	\subsection{Ruoli}
	
	Per ora non sono stati rilevate criticità per quanto riguarda l'organizzazione dei ruoli. 

	
	
\end{document}