\documentclass[5pt]{article}

\usepackage{sectsty}
\usepackage{graphicx}
\usepackage{lipsum} % for generating dummy text
\usepackage[margin=1in]{geometry}
\usepackage{setspace}
\usepackage{array}
\usepackage{cellspace}
\usepackage{tabularx}

% package and setup for tables
\usepackage[table]{xcolor}
\renewcommand{\arraystretch}{1.5}
\arrayrulecolor{black}

\usepackage{hyperref}
\usepackage{scrextend}
\graphicspath{ {../assets} }


% Margins
\topmargin=-0.45in
\evensidemargin=0in
\oddsidemargin=0in
\textwidth=6.5in
\textheight=9.0in
\headsep=0.25in

\title{Piano di Progetto}
\author{Jackpot Coding}
\renewcommand*\contentsname{Indice}
\date{\today}

%STARTOF THE DOCUMENT
\begin{document}

%-------------------------

% Reduce top margin only on the first page
\newgeometry{top=0.5in}

%UNIPD LOGO
  \vspace{8pt}
  \includegraphics[scale=0.2]{UNIPDFull.png}
%END UNIPD LOGO

\vspace{30pt}

%COURSE INFO
\begin{minipage}[t]{0.48\textwidth}
    %COURSE TITLE
        \begin{flushleft}
            Informatica\\
            \vspace{5pt}
            \textbf{\LARGE Ingegneria del Software}\\
            Anno Accademico: 2023/2024
        \end{flushleft}
    %END COURSE TITLE
\end{minipage}
%END COURSE INFO


\vspace{5px}


%BLACK LINE
\rule{\textwidth}{5pt}

%JACKPOT CODING INFO
\begin{minipage}[t]{0.50\textwidth}
    %LOGO JACKPOT CODING
    \begin{flushleft}
        \hspace{10pt}
        \includegraphics[scale=0.65]{jackpot-logo.png} 
    \end{flushleft}
\end{minipage}
\hspace{-60pt} % This adds horizontal space between the minipages
\begin{flushright}
    \begin{minipage}[t]{0.50\textwidth}
        %INFO JACKPOT CODING
        \begin{flushright}
            Gruppo: {\Large Jackpot Coding}\\
            Email: \href{mailto:jackpotcoding@gmail.com}{jackpotcoding@gmail.com}
        \end{flushright}
    \end{minipage}
\end{flushright}
%END JACKPOT CODING INFO

\vspace{24pt}

%TITLE
\begin{center}
    \textbf{\LARGE PIANO DI PROGETTO}
\end{center}
%END TITLE

\vspace{13pt}

\begin{flushleft}
    \begin{spacing}{1.5}
        REDATTORE: M. Gobbo, M. Favaretto\\%INSERT HERE THE NAMES
        VERIFICATORI: \\
        \vspace{7pt}
        DESTINATARI: Prof. T. Vardanega, Prof. R. Cardin\\%INSERT HERE THE NAMES
    \end{spacing}
\end{flushleft}

\begin{flushright}
    \begin{spacing}{1}
        USO: ESTERNO\\
        VERSIONE: 0.0.5\\
    \end{spacing}
\end{flushright}


% Restore original margins from the second page onwards
\restoregeometry

\pagebreak

\textbf{\Large Registro delle modifiche}
\begin{table}[ht]
\centering
\begin{tabular}{|c|c|c|c|c|}
\hline
\textbf{Versione} & \textbf{Data} & \textbf{Autore} & \textbf{Verificatore} & \textbf{Modifica} \\
\hline
v0.0.5 & 12/02/2024 & M. Gobbo - M. Favaretto & - & Scrittura sezione 4 \\
\hline
v0.0.4 & 10/02/2024 & M. Gobbo - M. Favaretto & - & Scrittura sezione 3 \\
\hline
v0.0.3 & 08/02/2024 & M. Gobbo - M. Favaretto & - & Scrittura sezione 2 \\
\hline
v0.0.2 & 05/02/2024 & M. Gobbo - M. Favaretto & - & Scrittura sezione 1 \\
\hline
V0.0.1 & 03/02/2023 & M. Gobbo - M. Favaretto & - & Creata struttura del documento \\
\hline
\end{tabular}
\caption{Cronologia delle modifiche}
\label{tab:conference}
\end{table}

\pagebreak
\tableofcontents
\pagebreak

\section{Introduzione}
\subsection{Scopo del documento}
Il presente documento ha come fine la presentazione della pianificazione e delle modalità di sviluppo di questo progetto. 
Nella fattispecie verranno esposte le analisi dei possibili rischi con alcune proposte di mitigazione, il modello adottato dal \textit{team}, 
il preventivo e il consuntivo di periodo.

\subsection{Scopo del capitolato}
Le \textit{I.A.\textsuperscript{G}} stanno vivendo un momento di grande innovazione ed entusiasmo, 
riuscendo a cogliere l'attenzione anche di utenti al di fuori dell'ambito informatico lavorativo e accademico, 
si veda come \textit{ChatGPT\textsuperscript{G}} sia diventato un fenomeno culturale. \\
La versatilità dell'\textit{I.A.\textsuperscript{G}} è oggi al centro dell'attenzione di molte \textit{software house\textsuperscript{G}}, 
poiché posso essere usate anche per migliorare e velocizzare la produzione. 
Per farne buon uso, è però necessario essere in grado di fornire i corretti \textit{prompt\textsuperscript{G}} al 
modello di \textit{I.A.\textsuperscript{G}} in uso. \\


L'obiettivo di questo progetto è realizzare un \textit{software\textsuperscript{G}} in grado di generare 
un \textit{prompt\textsuperscript{G}} a partire da una richiesta in linguaggio naturale\textsuperscript{G}. La richiesta dovrà riguardare un'interrogazione 
di un \textit{database\textsuperscript{G}} caricato sul sistema dall'utente. Tale \textit{prompt\textsuperscript{G}} sarà successivamente da fornire 
ad un \textit{LLM\textsuperscript{G}}, il quale restituirà all'utente una \textit{query\textsuperscript{G}} nel linguaggio 
\textit{SQL\textsuperscript{G}}.

\subsection{Glossario}
Alcuni termini presenti in questo documento potrebbero generare incomprensioni o necessitare di chiarimenti. 
Al fine di evitare queste eventualità, tali termini sono contrassegnati dalla lettera \textit{G} maiuscola posta ad apice della parola, 
per indicare che la loro spiegazione è presente all'interno del documento \textit{Glossario}.

\subsection{Riferimenti}
\subsubsection{Riferimenti Informativi}
Slide del corso "Inegegneria Del Software"
\begin{itemize}
      \item \href{https://www.math.unipd.it/~tullio/IS-1/2023/Dispense/T3.pdf}{Modelli di sviluppo software}
      \item \href{https://www.math.unipd.it/~tullio/IS-1/2023/Dispense/T4.pdf}{Gestione del progetto}
      \item \href{https://www.math.unipd.it/~tullio/IS-1/2023/Dispense/T5.pdf}{Analisi dei requisiti} 
\end{itemize}
\subsubsection{Riferimenti Normativi}
\begin{itemize}
      \item Documento "Norme di progetto"
      \item \href{https://www.math.unipd.it/~tullio/IS-1/2023/Dispense/PD2.pdf}{Regolamento del progetto}
      \item \href{https://www.math.unipd.it/~tullio/IS-1/2023/Progetto/C9.pdf}{Capitolato C9 "ChatSQL"} 
\end{itemize}

\section{Analisi dei rischi}
La realizzazione di un progetto \textit{software\textsuperscript{G}} comporta una serie di rischi che devono essere valutati e gestiti attentamente al fine di garantire il successo del prodotto finale. Nel corso di questo paragrafo, esamineremo i rischi che siamo riusciti ad individuare e le strategie per mitigarli efficacemente. \\[1\baselineskip]
Per facilitare l'analisi, ogni rischio avrà la seguente struttura:
\begin{itemize}
  \item \textbf{Codice}: Identifica univocamente il rischio, che può essere di 3 tipi:
      \begin{itemize}
      \item \textbf{RT}: Rischi tecnologici
      \item \textbf{RO}: Rischi organizzativi
      \item \textbf{RC}: Rischi comunicativi
      \end{itemize}
  \item \textbf{Titolo}: Breve titolo descrittivo del rischio.
  \item \textbf{Descrizione}: Spiegazione dettagliata del rischio.
  \item \textbf{Occorrenza}: Frequenza con cui il rischio si verifica.
  \item \textbf{Pericolosità}: Livello di pericolo associato al rischio.
  \item \textbf{Mitigazione}: Azioni per ridurre l'impatto o la probabilità del rischio.
\end{itemize}

\subsection{Rischi tecnologici}
\textbf{RT-1} \\
\textbf{Inesperienza nello sviluppo tecnologico}
\begin{itemize}
  \item \textbf{Descrizione}: La mancanza di esperienza nel gruppo su una determinata tecnologia o piattaforma potrebbe rallentare lo sviluppo del \textit{software\textsuperscript{G}} e portare a errori nella progettazione e nell'implementazione\textsuperscript{G}.
  \item \textbf{Occorrenza}: Media
  \item \textbf{Pericolosità}: Media
  \item \textbf{Mitigazione}: 
    \begin{itemize}
      \item Assicurarsi che i membri del gruppo ricevano formazione e supporto adeguati sulla tecnologia o piattaforma specifica.
      \item Affidare compiti critici a membri del gruppo con esperienza nel settore.
      \item Effettuare una pianificazione accurata per consentire più tempo per l'apprendimento e l'adattamento.
      \end{itemize}
\end{itemize}
\textbf{RT-2} \\
\textbf{Inesperienza strumenti \textit{software}}
\begin{itemize}
  \item \textbf{Descrizione}: Il gruppo non ha familiarità con l'utilizzo di \textit{software\textsuperscript{G}} specifici per la gestione di un progetto. Questa mancanza di esperienza potrebbe rallentare la produttività del gruppo e compromettere la qualità del progetto.
  \item \textbf{Occorrenza}: Bassa
  \item \textbf{Pericolosità}: Media
  \item \textbf{Mitigazione}: 
    \begin{itemize}
      \item  Attivarsi per avere una formazione adeguata sull'uso del \textit{software\textsuperscript{G}} gestionali attraverso \textit{tutorial\textsuperscript{G}} e documentazione.
      \item Controllare se gli strumenti scelti dispongono di una solida guida utente e documentazione varia.
    \end{itemize}
\end{itemize}


\subsection{Rischi interni}
\subsubsection{Rischi organizzativi}
\textbf{RO-1}\\
\textbf{Organizzazione carente}
\begin{itemize}
  \item \textbf{Descrizione}: Una struttura organizzativa inefficiente o una mancanza di chiarezza nei ruoli e nelle responsabilità possono portare a ritardi nelle decisioni, duplicazioni di compiti e confusione all'interno del gruppo.
  \item \textbf{Occorrenza}: Alta
  \item \textbf{Pericolosità}: Media
  \item \textbf{Mitigazione}: 
    \begin{itemize}
      \item Definire chiaramente le responsabilità di ciascun membro del gruppo e comunicarle in modo trasparente a tutti.
      \item Condurre regolarmente riunioni di coordinamento per monitorare lo stato del progetto e risolvere eventuali problemi organizzativi in modo tempestivo.
    \end{itemize}
\end{itemize}
\textbf{RO-2}\\
\textbf{Distribuzione disomogenea dei compiti}
\begin{itemize}
  \item \textbf{Descrizione}: Una distribuzione disomogenea dei compiti all'interno del gruppo potrebbe portare a una mancanza di lavoro per alcuni componenti e sovraccarico di lavoro per altri, compromettendo l'efficienza\textsuperscript{G} complessiva del progetto.
  \item \textbf{Occorrenza}: Media
  \item \textbf{Pericolosità}: Alta
  \item \textbf{Mitigazione}: 
    \begin{itemize}
      \item Pianificare attentamente l'allocazione dei compiti, tenendo conto delle competenze, delle disponibilità di ciascun membro del gruppo.
      \item Promuovere una cultura di collaborazione e flessibilità, incoraggiando il supporto reciproco tra i membri del gruppo e la condivisione dei carichi di lavoro.
      \item Monitorare regolarmente lo stato dei compiti assegnati e intervenire prontamente per ridistribuire le risorse in caso di necessità.
    \end{itemize}
\end{itemize}
\subsubsection{Rischi comunicativi}
\textbf{RC-1}\\
\textbf{Mancanza di comunicazione interna efficace}
\begin{itemize}
  \item \textbf{Descrizione}: La mancanza di una comunicazione interna efficace tra i membri del gruppo può portare a fraintendimenti, duplicazione del lavoro e rallentamenti nel processo decisionale.
  \item \textbf{Occorrenza}: Bassa
  \item \textbf{Pericolosità}: Alta
  \item \textbf{Mitigazione}: 
    \begin{itemize}
      \item Stabilire procedure chiare e canali di comunicazione ben definiti all'interno del gruppo.
      \item Organizzare riunioni regolari per discutere lo stato del progetto, assegnare compiti e condividere aggiornamenti.
      \item Utilizzare strumenti di gestione del progetto e di collaborazione per facilitare la comunicazione e 
      la condivisione delle informazioni come \textit{Trello} (utilizzato come \textit{dashboard\textsuperscript{G}}).
    \end{itemize}
\end{itemize}
\textbf{RC-2}\\
\textbf{Problemi di comunicazione con l'azienda}
\begin{itemize}
  \item \textbf{Descrizione}: La comunicazione inefficace con l'azienda può portare a malintesi, ritardi nelle consegne e insoddisfazione del cliente, rallentando inoltre l'attività del gruppo.
  \item \textbf{Occorrenza}: Bassa
  \item \textbf{Pericolosità}: Alta
  \item \textbf{Mitigazione}: 
    \begin{itemize}
      \item Stabilire canali di comunicazione chiari e definire ruoli e responsabilità per l'interazione con l'azienda.
      \item Fornire aggiornamenti regolari e trasparenti sullo stato del progetto.
      \item Ascoltare attivamente i \textit{feedback\textsuperscript{G}} dell'azienda e risolvere eventuali problemi di comunicazione tempestivamente.
    \end{itemize}
\end{itemize}
\textbf{RC-3}\\
\textbf{Tensioni e conflitti interni}
\begin{itemize}
  \item \textbf{Descrizione}: La presenza di tensioni, conflitti o mancanza di collaborazione tra i membri del gruppo può compromettere l'efficienza e la coesione del gruppo, influenzando negativamente la qualità del lavoro e il raggiungimento degli obiettivi del progetto.
  \item \textbf{Occorrenza}: Media
  \item \textbf{Pericolosità}: Media
  \item \textbf{Mitigazione}: 
    \begin{itemize}
      \item Utilizzare la votazione per prendere decisioni oggettive basate sulla maggioranza delle preferenze. Assicurarsi che tutte le parti coinvolte comprendano e accettino il risultato della votazione per favorire la collaborazione e il senso di appartenenza al gruppo.
      \item Stabilire procedure per la gestione dei conflitti e incoraggiare il coinvolgimento di un mediatore neutrale (arbitro) se necessario.
      \item Utilizzare il metodo del dibattito per esaminare i punti di vista contrastanti in modo strutturato e razionale
    \end{itemize}
\end{itemize}

\section{Modello di Sviluppo}
\subsection{Modello \textit{Agile}}
Il gruppo \textit{"Jackpot Coding"} ha deciso di ispirarsi ai Modelli Agili per lo sviluppo del progetto.\\
Il modello\textsuperscript{G} è caratterizzato da rilasci continui e un progressivo arricchimento delle funzionalità del prodotto, questo approccio consente di identificare agevolmente ciascun requisito\textsuperscript{G}, assegnandogli una priorità e organizzando lo sviluppo in modo da mantenere il prodotto funzionante in ogni fase, attivamente perseguendo i requisiti e adattandosi lungo il percorso verso il completamento del prodotto finale.\\
Anche la documentazione creata durante le varie fasi di sviluppo sarà costantemente aggiornata per integrare e/o modificare le informazioni rilevanti.

\subsection{Motivazione}
I motivi che hanno spinto il gruppo ad adottare un modello\textsuperscript{G} di questo tipo sono molteplici e sono i seguenti:
\begin{itemize}
    \item Attraverso la suddivisione temporale delle attività, i \textit{test} diventano più agevoli, semplificando la fase di modifica e \textit{test}.
    \item Gli errori nel prodotto sono più facilmente rilevabili grazie ai continui incrementi\textsuperscript{G}.
    \item La capacità di adattamento del \textit{team} alle variazioni è migliorata, con una documentazione che si evolve in modo collaborativo e flessibile insieme al prodotto, adattandosi alle esigenze del progetto.
    \item Le sfide nelle varie fasi vengono individuate e possono essere affrontate tramite la modifica degli obiettivi, se necessario, per risolvere eventuali problemi di natura temporale o organizzativa.
\end{itemize}

%-----------------------------

\section{Pianificazione}
\textbf{Scadenze}:
\begin{itemize}
    \item \textit{Requirments and Technology Baseline\textsuperscript{G}}: 01/03/2024
    \item \textit{Product Baseline\textsuperscript{G}}: 17/05/2024
\end{itemize}
% --- Aggiunta possibile linea del tempo

\subsection{Verso la \textit{RTB}}
\textbf{Periodo:} dal 24/10/2023 al 01/03/2024
\subsubsection{Preparazione preliminare}
\textbf{Periodo:} dal 24/10/2023 al 15/11/2023 
\vspace{0.3cm} \\
In questa fase il \textit{team} appena formato deve concentrarsi nella scelta del progetto dopo le necessarie consulenze con i proponenti presi in considerazione. \\
Contemporaneamente verranno scelti alcuni dei principali sistemi di gestione del progetto e della comunicazione. \\
Nel nostro caso, tali scelte sono ricadute nelle seguenti tecnologie: 
\begin{itemize}
  \item \textit{Trello} per la gestione di \textit{issue\textsuperscript{G}} e \textit{ticketing\textsuperscript{G}} delle attività
  \item \textit{Telegram\textsuperscript{G}} e \textit{Discord\textsuperscript{G}} per la comunicazione asincrona\textsuperscript{G} e sincrona\textsuperscript{G}
  \item \textit{\LaTeX\textsuperscript{G}} come linguaggio per la stesura dei documenti
\end{itemize}
È necessario iniziare una parziale identificazione di alcuni rischi, riportati nell'apposita sezione, che potrebbero compromettere il corretto svolgimento del progetto, 
ma utili anche a presentare un preventivo maggiormente accurato. 
In ultimo, è importante fornire almeno un'iniziale stesura del documento di \textit{Way of Working\textsuperscript{G}}, necessario per definire i principali metodi di lavoro dei membri del \textit{team}.
% 31/10 -> scelta proposta 
% 06/11 -> aggiudicazione appalti
% 15/11 -> verbale ultimato wow
\subsubsection{Analisi Requisiti}
\textbf{Periodo:} dal 15/11/2023 al 25/01/2024 
\vspace{0.3cm} \\
Una volta che l'appalto è stato aggiudicato al \textit{team}, è di principale importanza l'analisi di ogni possibile caso d'uso\textsuperscript{G} e requisito\textsuperscript{G} del capitolato\textsuperscript{G} preso a carico.
Per evitare dubbi ed avere un \textit{feedback\textsuperscript{G}} sul lavoro svolto, è utile organizzare un colloquio con il proponente relativamente all'analisi svoltasi fino a quel momento. \\
Dopo una fase di \textit{brainstorming\textsuperscript{G}} del gruppo e \textit{feedback\textsuperscript{G}} dal proponente riguardo i casi d'uso identificati, si procederà alla stesura del documento di Analisi dei Requisiti\textsuperscript{G},
dove verranno riportati tutti i casi d'uso identificati, i requisiti obbligatori, desiderabili e opzionali. \\
Questa fase è dunque di fondamentale importanza per la successiva.
% 15/11 -> inizio discussione adr, contattato proponente
% ??/01 -> fine adr?
\subsubsection{Progettazione per RTB}
\textbf{Periodo} dal 18/01/2024 al 12/02/2024 
\vspace{0.3cm} \\
% periodo 18-25 ultimazione adr e relativa verifica -> quindi inizio poc
In questo periodo, basandosi sulla precedente analisi dei requisiti\textsuperscript{G}, inizierà la realizzazione del \textit{PoC\textsuperscript{G}}. \\
Questa fase si concluderà con un colloquio con il proponente, per un \textit{feedback\textsuperscript{G}} circa il \textit{PoC\textsuperscript{G}} prodotto.
% 18/01 -> inizio POC
% 11/02 -> fine POC
\subsubsection{Rifiniture}
\textbf{Periodo:} dal 12/02/2024 al 01/03/2024 
\vspace{0.3cm} \\
Nell'ultima fase della prima parte del progetto, verranno ultimate le modifiche e relative verifiche ai vari documenti 
prodotti fino a questo momento. \\
Questa parte si concluderà con un colloquio e una presentazione del lavoro fino ad ora svolto 
ai professori del corso Ingegneria del \textit{Software}.
% aggiunta immagini / schemi esempio: linee del tempo
% 25/02 -> fine rtb (non definitiva)
% rifinitura ultimi documenti: pdp, pdq, ndp

%-------------
\subsection{Verso la PB}
\textbf{Periodo Pianificato:} dal 01/03/2024 al 17/05/2024
\subsubsection{Preparazione preliminare}
\textbf{Periodo:} dal 25/02/2024 al 10/03/2024 
\vspace{0.3cm} \\
In questo periodo l'obiettivo sarà l'integrazione dei \textit{feedback\textsuperscript{G}} e la sistemazione degli errori emersi durante la \textit{RTB\textsuperscript{G}}; 
una volta passata questa fase preliminare è il momento di procedere verso la \textit{PB\textsuperscript{G}}.
Verranno dunque pianificate le attività future e la relativa suddivisione dei compiti.

\subsubsection{Progettazione primaria e codifica dei requisiti obbligatori}
\textbf{Periodo:} dal 10/03/2024 al 10/04/2024 
\vspace{0.3cm} \\
Terminata la retrospettiva dell'\textit{RTB\textsuperscript{G}}, Verrà utilizzato il \textit{PoC\textsuperscript{G}}, realizzato nelle fasi precedenti, 
come riferimento per la costruzione del progetto effettivo.
Lo scopo di questa fase è la realizzazione di una prima versione del prodotto finale, che dovrà soddisfare tutti i requisiti\textsuperscript{G} obbligatori. \\
Per raggiungere questo obiettivo, andranno svolte le seguenti attività:
\begin{itemize}
  \item \textbf{Studio tecnologie}: al fine di produrre un risultato migliore, andranno approfondite maggiormente le tecnologie approcciate 
  durante la codifica del \textit{PoC\textsuperscript{G}}
  \item \textbf{Progettazione}: scelta dei \textit{design pattern\textsuperscript{G}} e delle principali unità architetturali\textsuperscript{G}
  \item \textbf{Realizzazione del prodotto}: fase che si dividerà nelle seguente sotto-attività che si svolgeranno parallelamente
  \begin{itemize}
    \item \textbf{Codifica}: scrittura del codice che soddisferà i requisiti\textsuperscript{G} obbligatori, rispettando le scelte architetturali
    \item \textbf{Documentazione}: ampliamento della documentazione già prodotta e scrittura della documentazione relativa al programma
    \item \textbf{Verifica e validazione}: fase di svolgimento di test automatici e non al fine di verificare che il codice non produca errori e soddisfi i requisiti\textsuperscript{G} obbligatori.
  \end{itemize}  
\end{itemize}

\subsubsection{Progettazione secondaria e codifica dei requisiti opzionali}
\textbf{Periodo:} dal 10/04/2024 al 30/04/2024 
\vspace{0.3cm} \\
Questo periodo sarà simile al precedente, con la differenza che sarà dedicato alla realizzazione e codifica dei requisiti desiderabili e opzionali 
che si è deciso di implementare. \\
Anche la suddivisione di attività risulterà molto simile:
\begin{itemize}
  \item \textbf{Scelta dei requisiti}: scelta dei requisiti non obbligatori da implementare. Da concordare con i membri del gruppo in base alle risorse attualmente disponibili.
  \item \textbf{Realizzazione del prodotto}: fase che verrà divisa nelle seguente sotto-attività e che sarà incrementale al periodo precedente
  \begin{itemize}
    \item \textbf{Codifica}: scrittura del codice che soddisferà i requisiti non obbligatori, rispettando le scelte architetturali
    \item \textbf{Documentazione}: ampliamento della documentazione già prodotta
    \item \textbf{Verifica e validazione}: fase di svolgimento di test automatici e non al fine di verificare che il codice non produca errori e soddisfi i requisiti non obbligatori integrati.
  \end{itemize}  
\end{itemize}

\subsubsection{Validazione finale, collaudo e \textit{PB}}
\textbf{Periodo:} dal 30/04/2024 al 17/05/2024 
\vspace{0.3cm} \\
In quest'ultima fase, il \textit{team} ultimerà le attività di verifica, validazione e collaudo del codice prodotto e della documentazione scritta fino a questo momento.
In caso di necessità, verranno aggiunte le necessarie correzioni risultate dai \textit{feedback\textsuperscript{G}} di queste attività. \\
Il fine di questo periodo è la seconda revisione del progetto, ossia la \textit{Product Baseline\textsuperscript{G}}.

\section{Preventivo}
\subsection{Periodo \textit{RTB}}

\begin{table}[ht]
  \centering
  \rowcolors{2}{black!15}{}
  \resizebox{\linewidth}{!}{
    \begin{tabular}{|c|c|c|c|c|c|c||c|}
      \hline
      \rowcolor{yellow!50}
      MEMBRI & Responsabile & Amministratore & Analista & Progettista &	Programmatore	& Verificatore & Totale \\ \hline \hline
      Giulio Moretto & 3 & 3 & 12	& 15 & 7 & 1 & 41 \\ \hline
      Riccardo Simionato & 3 & * & 12	& 5	& * & 21 & 41 \\ \hline
      Marco Favaretto & 10 & 2 & 12 & * & * & 17 & 41 \\ \hline
      Marco Gobbo & 8 & 6 & 12 & * & * & 15 & 41 \\ \hline
      Matteo Camillo & * & 7 & 12 & 10 & 7 & 5 & 41 \\ \hline
      Edoardo	Gallo & * & 6 & 12 & 10 & 7 & 6 & 41 \\ \hline
    \end{tabular}
  }
\end{table}

\subsection{Periodo \textit{PB}}
\begin{table}[ht]
  \centering
  \rowcolors{2}{black!15}{}
  \resizebox{\linewidth}{!}{
    \begin{tabular}{|c|c|c|c|c|c|c||c|}
      \rowcolor{yellow!50}
      \hline
      MEMBRI & Responsabile & Amministratore & Analista & Progettista &	Programmatore	& Verificatore & Totale \\ \hline \hline
      Giulio Moretto & 4 & 5 & * & 10 & 19 & 13 & 51 \\ \hline
      Riccardo Simionato & 4 & 8 & * & 9 & 19 & 11 & 51 \\ \hline
      Marco Favaretto & * & 2 & * & 18 & 19 & 12 & 51 \\ \hline
      Marco Gobbo & * & * & * & 18 & 18 & 15 & 51 \\ \hline
      Matteo Camillo & 8 & 4 & * & 8 & 18 & 13 & 51 \\ \hline
      Edoardo	Gallo & 8 & 5 & * & 5 & 18 & 15 & 51 \\ \hline
    \end{tabular}
  }
\end{table}

\section{Consuntivo}
Di seguito verranno mostrate le spese effettivamente avvenute nelle fasi citate
\subsection{Periodo \textit{RTB}}


\end{document}
