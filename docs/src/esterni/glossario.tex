\documentclass[5pt]{article}

\usepackage{sectsty}
\usepackage{graphicx}
\usepackage{lipsum} % for generating dummy text
\usepackage[margin=1in]{geometry}
\usepackage{setspace}
\usepackage{array}
\usepackage{cellspace}
\usepackage{tabularx}


\usepackage{hyperref}
\usepackage{scrextend}
\graphicspath{ {../assets} }


% Margins
\topmargin=-0.45in
\evensidemargin=0in
\oddsidemargin=0in
\textwidth=6.5in
\textheight=9.0in
\headsep=0.25in

\title{Glossario}
\author{Jackpot Coding}
\renewcommand*\contentsname{Indice}
\date{\today}

%STARTOF THE DOCUMENT
\begin{document}

%-------------------------

% Reduce top margin only on the first page
\newgeometry{top=0.5in}

%UNIPD LOGO
    \vspace{8pt}
    \includegraphics[scale=0.2]{UNIPDFull.png}
%END UNIPD LOGO

\vspace{30pt}

%COURSE INFO
\begin{minipage}[t]{0.48\textwidth}
    %COURSE TITLE
        \begin{flushleft}
            Informatica\\
            \vspace{5pt}
            \textbf{\LARGE Ingegneria del Software}\\
            Anno Accademico: 2023/2024
        \end{flushleft}
    %END COURSE TITLE
\end{minipage}
%END COURSE INFO


\vspace{5px}


%BLACK LINE
\rule{\textwidth}{5pt}

%JACKPOT CODING INFO
\begin{minipage}[t]{0.50\textwidth}
    %LOGO JACKPOT CODING
    \begin{flushleft}
        \hspace{10pt}
        \includegraphics[scale=0.65]{jackpot-logo.png} 
    \end{flushleft}
\end{minipage}
\hspace{-60pt} % This adds horizontal space between the minipages
\begin{flushright}
    \begin{minipage}[t]{0.50\textwidth}
        %INFO JACKPOT CODING
        \begin{flushright}
            Gruppo: {\Large Jackpot Coding}\\
            Email: \href{mailto:jackpotcoding@gmail.com}{jackpotcoding@gmail.com}
        \end{flushright}
    \end{minipage}
\end{flushright}
%END JACKPOT CODING INFO

\vspace{24pt}

%TITLE
\begin{center}
    \textbf{\LARGE GLOSSARIO}
\end{center}
%END TITLE

\vspace{13pt}

\begin{flushleft}
    \begin{spacing}{1.5}
        REDATTORE: E. Gallo, G. Moretto\\%INSERT HERE THE NAMES
        VERIFICATORI: \\
        \vspace{7pt}
        DESTINATARI: Prof. T. Vardanega, Prof. R. Cardin, Zucchetti S.p.A.\\%INSERT HERE THE NAMES
    \end{spacing}
\end{flushleft}

\begin{flushright}
    \begin{spacing}{1}
        USO: ESTERNO\\
        VERSIONE: 0.0.3\\
    \end{spacing}
\end{flushright}


% Restore original margins from the second page onwards
\restoregeometry

\pagebreak

\textbf{\Large Registro delle modifiche}
\begin{table}[ht]
\centerline{%
  \begin{tabular}{|c|c|c|c|c|}
    \hline
    \textbf{Versione} & \textbf{Data} & \textbf{Autore} & \textbf{Verificatore} & \textbf{Modifica} \\
    \hline
    % versione & data & autore & verificatore & descrizioneModifica \\
    % \hline
    V0.0.3 & 16/02/2024 & G.Moretto & - & Inserito definizioni per i termini nelle sezioni B-U \\
    \hline
    V0.0.2 & 13/02/2024 & G.Moretto & - & Aggiunto termini, inserito definizioni per i termini nella sezione A \\
    \hline
    V0.0.1 & 15/11/2023 & E. Gallo & G. Moretto & Creata struttura del documento \\
    \hline
  \end{tabular}%
  }
  \label{tab:conference}
\end{table}



\pagebreak
\tableofcontents
\pagebreak

\section*{Introduzione e struttura}
\addcontentsline{toc}{section}{Introduzione e struttura}
\subsection*{Riferimenti informativi}
\addcontentsline{toc}{subsection}{Riferimenti informativi}
\pagebreak



\section*{A}
\addcontentsline{toc}{section}{A}

\begin{flushleft}

\addcontentsline{toc}{subsection}{AI/IA}
\textbf{AI/IA}: Sigla per \textit{Artificial Intelligence} (o Intelligenza Artificiale) identifica una disciplina che studia come simulare il pensiero umano con l'utilizzo di sistemi informatici\newline

\addcontentsline{toc}{subsection}{Amministratore}
\textbf{Amministratore}: Utente autenticato con la possibilità di poter effettuare tutte le operazioni possibili nell'applicazione . \newline

\addcontentsline{toc}{subsection}{Analisi dei Requisiti}
\textbf{Analisi dei Requisiti}: Attività nella quale vengono raccolte le necessità di un proponente per quanto riguarda un programma software. Queste necessità vengono poi trasformate in requisiti descritti in un documento dedicato. \newline

\addcontentsline{toc}{subsection}{API}
\textbf{API}: Sigla per \textit{Application Program Interface}, descrive un insieme di procedure utilizzate per lo scambio di informazioni tra diversi sistemi, computer o software. \newline

\addcontentsline{toc}{subsection}{Attore}
\textbf{Attore}: Viene identificato come attore chi interagisce con il sistema dall'esterno. Questo può essere una tipologia di utente o un sistema esterno. 

\end{flushleft}

\section*{B}
\addcontentsline{toc}{section}{B}
\begin{flushleft}
\addcontentsline{toc}{subsection}{Browser}
\textbf{\textit{Browser}}: Applicazione utilizzata per la navigazione di risorse nel web. 	
	
\end{flushleft}

\section*{C}
\addcontentsline{toc}{section}{C}

\begin{flushleft}

\addcontentsline{toc}{subsection}{Capitolato}
\textbf{Capitolato}: Documento che specifica i bisogni del proponente. In questo documento sono espresse caratteristiche e requisiti che il prodotto consegnato deve o può avere.\newline

\addcontentsline{toc}{subsection}{Caso d'uso}
\textbf{Caso d'uso}: Uno scenario che descrive il comportamento che uno o più attori hanno con il sistema rendendo esplicite le interazioni tra loro.\newline

\addcontentsline{toc}{subsection}{ChatGPT}
\textbf{ChatGTP}: Un software creato dall'azienda \textit{Open AI} basato su intelligenza artificiale. Il suo obiettivo è quello di poter effettuare conversazioni con un utente umano e rispondere a richieste espresse in linguaggio naturale.\newline

\addcontentsline{toc}{subsection}{Credenziali}
\textbf{Credenziali}: Combinazione di dati che permettono ad un utente di identificarsi. Di norma si tratta di nome utente e password.\newline

\addcontentsline{toc}{subsection}{CSS}
\textbf{CSS}: Sigla per \textit{Cascading Style Sheets}, (fogli di stile a cascata), si tratta di un linguaggio utilizzato per applicare formattazione a pagine HTML utilizzate in siti internet e applicazioni web.\\

\end{flushleft}

\section*{D}
\addcontentsline{toc}{section}{D}

\begin{flushleft}

\addcontentsline{toc}{subsection}{Database}
\textbf{Database}: Detta anche base di dati, si tratta di una collezione di dati organizzata e accessibile in maniera elettronica.\newline

\addcontentsline{toc}{subsection}{Dashboard}
\textbf{\textit{Dashboard}}: Detto anche cruscotto, si tratta di una rappresentazione grafica dei valori importanti riguardo l'andamento di un progetto. Questa permette una visione rapida dei valori scelti.\newline

\end{flushleft}

\section*{E}
\addcontentsline{toc}{section}{E}

\section*{F}
\addcontentsline{toc}{section}{F}
\begin{flushleft}

\addcontentsline{toc}{subsection}{File}
\textbf{\textit{File}}: Traducibile come archivio, indica un contenitore di dati che risiede in un supporto di archiviazione digitale.\newline

\addcontentsline{toc}{subsection}{Form}
\textbf{\textit{Form}}: Traducibile come modulo, indica un interfaccia utente che permette l'inserimento e l'invio di dati ad un sistema tramite elementi che compongono l'interfaccia stessa (es. scelta multipla, campo di testo, pulsante, ecc...).\newline

\addcontentsline{toc}{subsection}{Formato}
\textbf{Formato}: Si tratta di una convenzione adottata per la lettura e scrittura dei contenuti di un \textit{file}.

\end{flushleft}

\section*{G}
\addcontentsline{toc}{section}{G}
\begin{flushleft}

\addcontentsline{toc}{subsection}{git}
\textbf{git}: \textit{Software} creato per la gestione del codice sorgente ed il tracciamento delle modifiche apportate ad esso.\newline

\addcontentsline{toc}{subsection}{GitHub}
\textbf{\textit{GitHub}}: Si tratta di un servizio creato per poter ospitare progetti che utilizzano git.

\end{flushleft}

\section*{H}
\addcontentsline{toc}{section}{H}

\addcontentsline{toc}{subsection}{HTML}
\textbf{HTML}: Sigla per \textit{HyperText Markup Language} (Linguaggio di marcatura d'ipertesto). Si tratta di un linguaggio utilizzato per definire la struttura di un documento ipertestuale visualizzato da un browser.

\section*{I}
\addcontentsline{toc}{section}{I}

\addcontentsline{toc}{section}{Input}
\textbf{\textit{Input}}: Termine inglese che ha come significato l'atto di immettere dati o informazioni, tramite una periferica adibita a tale scopo, verso un sistema che li elabori.

\section*{J}
\addcontentsline{toc}{section}{J}

\section*{K}
\addcontentsline{toc}{section}{K}

\section*{L}
\addcontentsline{toc}{section}{L}
\begin{flushleft}
\addcontentsline{toc}{subsection}{Linguaggio naturale}
\textbf{Linguaggio naturale}: Modalità di espressione utilizzata dagli essere umani.\newline

\addcontentsline{toc}{subsection}{LLM}
\textbf{\textit{LLM}}: Acronimo per \textit{Large Language Model} (modello linguistico di grandi dimensioni), si tratta di un modello utilizzato per la comprensione e generazione del linguaggio naturale.\newline

\addcontentsline{toc}{subsection}{Login}
\textbf{\textit{Login}}: In Italiano definito come accesso, o meglio il processo di identificazione di un utente nel momento in cui entra in un sistema informatico. \newline

\addcontentsline{toc}{subsection}{Logout}
\textbf{\textit{Logout}}: indentifica il processo di uscita di un utente da un sistema informatico.
\end{flushleft}
\section*{M}
\addcontentsline{toc}{section}{M}

\addcontentsline{toc}{subsection}{Modello}
\textbf{Modello}: Un programma utilizzato nell'ambito dell'intelligenza artificiale, addestrato su un gruppo di dati, utilizzato per riconoscere alcuni pattern e effettuare decisioni senza intervento umano.

\section*{N}
\addcontentsline{toc}{section}{N}

\addcontentsline{toc}{subsection}{Norme di Progetto}
\textbf{Norme di Progetto}: Documento che raccoglie le norme e procedure che il gruppo di sviluppo deve utilizzare durante il ciclo di vita del \textit{software}.

\section*{O}
\addcontentsline{toc}{section}{O}

\section*{P}
\addcontentsline{toc}{section}{P}

\addcontentsline{toc}{subsection}{Password}
\textbf{Password}: Traducibile come "codice di accesso", consiste nella sequenza di caratteri alfanumerici e simboli utilizzata da un utente per l'accesso esclusivo ad un risorsa informatica. \newline

\addcontentsline{toc}{subsection}{PoC}
\textbf{\textit{PoC}}: Sigla per \textit{Proof of Concept}, si tratta di un \textit{software} che il gruppo di sviluppo utilizza per verificare l'applicazione delle tecnologie selezionate alle necessità espresse dal proponente.\newline

\addcontentsline{toc}{subsection}{Prompt}
\textbf{\textit{Prompt}}: Un testo in lijnguaggio naturale interpretabile dal un modello di intelligenza artificiale.\newline

\addcontentsline{toc}{subsection}{Proponente}
\textbf{Proponente}: Figura che propone un lavoro o un progetto per la sua realizzazione. \newline

\addcontentsline{toc}{subsection}{Python}
\textbf{\textit{Python}}: Linguaggio di programmazione ad alto livello, orientato agli oggetti.

\section*{Q}
\addcontentsline{toc}{section}{Q}

\addcontentsline{toc}{subsection}{Query SQL}
\textbf{\textit{Query SQL}}: Traducibile come "interrogazione". Si tratta di un comando inserito da un utente per la richiesta di dati o modifiche su dati presenti in un \textit{database} (in questo caso di tipo SQL).

\section*{R}
\addcontentsline{toc}{section}{R}

\addcontentsline{toc}{subsection}{Repository}
\textbf{\textit{Repository}}: Si tratta di un ambiente dove vengono "depositati" in modo digitale dati ed informazioni che compongono il progetto.\newline

\addcontentsline{toc}{subsection}{Requisito}
\textbf{Requisito}: Funzionalità e/o caratteristica di un sistema e/o le sue componenti. Questo viene descritto specificando il suo comportamento, i suoi ingressi, le sue uscite ed il suo obiettivo.

\section*{S}
\addcontentsline{toc}{section}{S}

\addcontentsline{toc}{subsection}{Sinonimi}
\textbf{Sinonimi}: Relazione tra due parole con lo stesso significato. Utilizzato in questo progetto per individuare il soggetto della richiesta da parte dell'utente.\newline

\addcontentsline{toc}{subsection}{SQL}
\textbf{\textit{SQL}}: Sigla per \textit{Structured Query Language}, si tratta di un linguaggio standardizzato per interrogare e amministrare un \textit{database} basato su modello relazionale.\newline

\addcontentsline{toc}{subsection}{Database}
\textbf{Struttura \textit{Database}}: Definizione della composizione di un \textit{database} definendo le sue tabelle e la loro composizione.

\section*{T}
\addcontentsline{toc}{section}{T}

\addcontentsline{toc}{subsection}{Tabella}
\textbf{Tabella}: Struttura utilizzata da un \textit{database} basato su modello relazionale per la gestione dei dati. Questa è composta da dei campi che indicano il tipo di dato che esso contiene e altri dati utili all'utilizzo e l'amministrazione della tabella.

\section*{U}
\addcontentsline{toc}{section}{U}

\addcontentsline{toc}{subsection}{Utente}
\textbf{Utente}: Colui che interagisce con un prodotto \textit{software}.

\section*{V}
\addcontentsline{toc}{section}{V}

\section*{W}
\addcontentsline{toc}{section}{W}

\section*{X}
\addcontentsline{toc}{section}{X}

\section*{Y}
\addcontentsline{toc}{section}{Y}

\section*{Z}
\addcontentsline{toc}{section}{Z}



\end{document}