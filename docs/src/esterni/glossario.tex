\documentclass[5pt]{article}

\usepackage{sectsty}
\usepackage{graphicx}
\usepackage{lipsum} % for generating dummy text
\usepackage[margin=1in]{geometry}
\usepackage{setspace}
\usepackage{array}
\usepackage{cellspace}
\usepackage{tabularx}


\usepackage{hyperref}
\usepackage{scrextend}
\graphicspath{ {../assets} }


% Margins
\topmargin=-0.45in
\evensidemargin=0in
\oddsidemargin=0in
\textwidth=6.5in
\textheight=9.0in
\headsep=0.25in

\title{Glossario}
\author{Jackpot Coding}
\renewcommand*\contentsname{Indice}
\date{\today}

%STARTOF THE DOCUMENT
\begin{document}

%-------------------------

% Reduce top margin only on the first page
\newgeometry{top=0.5in}

%UNIPD LOGO
    \vspace{8pt}
    \includegraphics[scale=0.2]{UNIPDFull.png}
%END UNIPD LOGO

\vspace{30pt}

%COURSE INFO
\begin{minipage}[t]{0.48\textwidth}
    %COURSE TITLE
        \begin{flushleft}
            Informatica\\
            \vspace{5pt}
            \textbf{\LARGE Ingegneria del Software}\\
            Anno Accademico: 2023/2024
        \end{flushleft}
    %END COURSE TITLE
\end{minipage}
%END COURSE INFO


\vspace{5px}


%BLACK LINE
\rule{\textwidth}{5pt}

%JACKPOT CODING INFO
\begin{minipage}[t]{0.50\textwidth}
    %LOGO JACKPOT CODING
    \begin{flushleft}
        \hspace{10pt}
        \includegraphics[scale=0.65]{jackpot-logo.png} 
    \end{flushleft}
\end{minipage}
\hspace{-60pt} % This adds horizontal space between the minipages
\begin{flushright}
    \begin{minipage}[t]{0.50\textwidth}
        %INFO JACKPOT CODING
        \begin{flushright}
            Gruppo: {\Large Jackpot Coding}\\
            Email: \href{mailto:jackpotcoding@gmail.com}{jackpotcoding@gmail.com}
        \end{flushright}
    \end{minipage}
\end{flushright}
%END JACKPOT CODING INFO

\vspace{24pt}

%TITLE
\begin{center}
    \textbf{\LARGE GLOSSARIO}
\end{center}
%END TITLE

\vspace{13pt}

\begin{flushleft}
    \begin{spacing}{1.5}
        REDATTORE: E. Gallo\\%INSERT HERE THE NAMES
        VERIFICATORI: \\
        \vspace{7pt}
        DESTINATARI: Prof. T. Vardanega, Prof. R. Cardin, Zucchetti S.p.A.\\%INSERT HERE THE NAMES
    \end{spacing}
\end{flushleft}

\begin{flushright}
    \begin{spacing}{1}
        USO: ESTERNO\\
        VERSIONE: 0.0.1\\
    \end{spacing}
\end{flushright}


% Restore original margins from the second page onwards
\restoregeometry

\pagebreak

\textbf{\Large Registro delle modifiche}
\begin{table}[ht]
\centerline{%
  \begin{tabular}{|c|c|c|c|c|}
    \hline
    \textbf{Versione} & \textbf{Data} & \textbf{Autore} & \textbf{Verificatore} & \textbf{Modifica} \\
    \hline
    % versione & data & autore & verificatore & descrizioneModifica \\
    % \hline
    V0.0.1 & 15/11/2023 & E. Gallo & - & Creata struttura del documento \\
    \hline
  \end{tabular}%
  }
  \label{tab:conference}
\end{table}



\pagebreak
\tableofcontents
\pagebreak

\section*{Introduzione e struttura}
\addcontentsline{toc}{section}{Introduzione e struttura}
\subsection*{Riferimenti informativi}
\addcontentsline{toc}{subsection}{Riferimenti informativi}
\pagebreak

\section*{A}
\addcontentsline{toc}{section}{A}

\textbf{AI/IA}: Sigla per \textit{Artificial Intelligence} (o Intelligenza Artificiale) \\

\textbf{Amministratore}: Utente autenticato con privilegi elevati, in grado di compiere operazioni non permesse a tutti i tipi di utente. \\

\textbf{Analisi dei Requisiti}: Attività nella quale vengono raccolte le necessità di un proponente per quanto riguarda un programma software. Queste necessità vengono poi trasformate in requisiti descritti secondo una modalità definita a priori. \\

\textbf{API}: Sigla per \textit{Application Program Interface}, descrive un insieme di procedure utilizzate per lo scambio di informazioni tra diversi sistemi, computer o software. \\

\textbf{Attore}: Viene identificato come attore chi interagisce con il sistema dall'esterno. Questo può essere una tipologia di utente o un sistema esterno. \\


\section*{B}
\addcontentsline{toc}{section}{B}

\section*{C}
\addcontentsline{toc}{section}{C}

\textbf{Capitolato}: \\

\textbf{Caso d'uso}: \\

\textbf{ChatGTP}: \\

\textbf{Credenziali}: \\

\textbf{CSS}: \\

\section*{D}
\addcontentsline{toc}{section}{D}

\textbf{Database}: \\

\textbf{\textit{Dashboard}}: \\

\section*{E}
\addcontentsline{toc}{section}{E}

\section*{F}
\addcontentsline{toc}{section}{F}

\textbf{File}: \\

\textbf{\textit{Form}}: \\

\textbf{Formato}: \\

\section*{G}
\addcontentsline{toc}{section}{G}

\textbf{GitHub}: \\

\section*{H}
\addcontentsline{toc}{section}{H}

\textbf{HTML}: \\

\section*{I}
\addcontentsline{toc}{section}{I}

\textbf{\textit{Input}}: \\

\section*{J}
\addcontentsline{toc}{section}{J}

\section*{K}
\addcontentsline{toc}{section}{K}

\section*{L}
\addcontentsline{toc}{section}{L}

\textbf{Linguaggio naturale}: \\

\textbf{\textit{LLM}}: \\

\textbf{\textit{Login}}: \\

\textbf{\textit{Logout}}: \\

\section*{M}
\addcontentsline{toc}{section}{M}

\textbf{Modello}: \\

\section*{N}
\addcontentsline{toc}{section}{N}

\textbf{Norme di Progetto}: \\

\section*{O}
\addcontentsline{toc}{section}{O}

\section*{P}
\addcontentsline{toc}{section}{P}

\textbf{Password}: \\

\textbf{\textit{PoC}}: \\

\textbf{\textit{Prompt}}: \\

\textbf{Proponente}: \\

\textbf{\textit{Python}}: \\

\section*{Q}
\addcontentsline{toc}{section}{Q}

\textbf{\textit{Query SQL}}: \\

\section*{R}
\addcontentsline{toc}{section}{R}

\textbf{\textit{Repository}}: \\

\textbf{Requisiti}: \\

\section*{S}
\addcontentsline{toc}{section}{S}

\textbf{Sinonimi}: \\

\textbf{\textit{Software}}: \\

\textbf{\textit{SQL}}: \\

\textbf{Struttura \textit{Database}}: \\

\section*{T}
\addcontentsline{toc}{section}{T}

\textbf{Tabella}: \\

\section*{U}
\addcontentsline{toc}{section}{U}

\textbf{Utente}: \\

\section*{V}
\addcontentsline{toc}{section}{V}

\section*{W}
\addcontentsline{toc}{section}{W}

\section*{X}
\addcontentsline{toc}{section}{X}

\section*{Y}
\addcontentsline{toc}{section}{Y}

\section*{Z}
\addcontentsline{toc}{section}{Z}

\end{document}