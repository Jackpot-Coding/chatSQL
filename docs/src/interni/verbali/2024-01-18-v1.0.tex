\documentclass[5pt]{article}

\usepackage{sectsty}
\usepackage{graphicx}
\usepackage{lipsum} % for generating dummy text
\usepackage[margin=1in]{geometry}
\usepackage{setspace}
\usepackage{array}
\usepackage{cellspace}


\usepackage{hyperref}
\usepackage{scrextend}
\graphicspath{ {../../assets} }


% Margins
\topmargin=-0.45in
\evensidemargin=0in
\oddsidemargin=0in
\textwidth=6.5in
\textheight=9.0in
\headsep=0.25in

\title{ Verbale - 18/01/2024 }
\date{\today}

%STARTOF THE DOCUMENT
\begin{document}
	
	%-------------------------
	
	% Reduce top margin only on the first page
	\newgeometry{top=0.5in}
	
	%UNIPD LOGO
	\vspace{8pt}
	\includegraphics[scale=0.2]{UNIPDFull.png}
	%END UNIPD LOGO
	
	\vspace{10pt}
	
	%COURSE INFO
	\begin{minipage}[t]{0.48\textwidth}
		%COURSE TITLE
		\begin{flushleft}
			Informatica\\
			\vspace{5pt}
			\textbf{\LARGE Ingegneria del Software}\\
			Anno Accademico: 2023/2024
		\end{flushleft}
		%END COURSE TITLE
	\end{minipage}
	%END COURSE INFO
	
	
	\vspace{5px}
	
	
	%BLACK LINE
	\rule{\textwidth}{5pt}
	
	%JACKPOT CODING INFO
	\begin{minipage}[t]{0.50\textwidth}
		%LOGO JACKPOT CODING
		\begin{flushleft}
			\hspace{10pt}
			\includegraphics[scale=0.65]{jackpot-logo.png} 
		\end{flushleft}
	\end{minipage}
	\hspace{-60pt} % This adds horizontal space between the minipages
	\begin{flushright}
		\begin{minipage}[t]{0.50\textwidth}
			%INFO JACKPOT CODING
			\begin{flushright}
				Gruppo: {\Large Jackpot Coding}\\
				Email: \href{mailto:jackpotcoding@gmail.com}{jackpotcoding@gmail.com}
			\end{flushright}
		\end{minipage}
	\end{flushright}
	%END JACKPOT CODING INFO
	
	\vspace{20pt}
	
	%TITLE
	\begin{center}
		\textbf{\large VERBALE }
		\textbf{\large 18/01/2024} \\
		\textbf{\LARGE DISCUSSIONE STRUTTURA E REALIZZAZIONE DEL POC}
	\end{center}
	%END TITLE
	
	\vspace{13pt}
	
	\begin{flushleft}
		\begin{spacing}{1.5}
			REDATTORE: R. Simionato\\%INSERT HERE THE NAMES
			VERIFICATORE: G. Moretto\\%INSERT HERE THE NAMES
			RESPONSABILE: G. Moretto\\%INSERT HERE THE NAMES
			\vspace{7pt}
			SCRIBA: R. Simionato\\%INSERT HERE THE NAMES
			\vspace{7pt}
			DESTINATARI:   Prof. T. Vardanega, Prof. R. Cardin\\%INSERT HERE THE NAMES
		\end{spacing}
	\end{flushleft}
	
	\begin{flushright}
		\begin{spacing}{1}
			USO: INTERNO\\
			VERSIONE: 1.0\\
		\end{spacing}
	\end{flushright}
	
	
	% Restore original margins from the second page onwards
	\restoregeometry
	
	\pagebreak
	
	% Optional TOC
	% \tableofcontents
	% \pagebreak
	
	\section{ORARIO}
	\begin{spacing}{1.5}
		{\large Inizio incontro: 20:20}\\
		{\large Fine incontro: 21:00}
	\end{spacing}
	
	\section{PARTECIPANTI}
	% Define minimal spacing at the top and bottom of cells
	\setlength\cellspacetoplimit{6pt}
	\setlength\cellspacebottomlimit{6pt}
	
	\begin{table}[ht]
		\begin{tabular}{|Sc|Sc|}
			\hline
			\textbf{Nome Componente} & \textbf{Durata della presenza} \\
			\hline
			Camillo Matteo & - \\
			Favaretto Marco & 40 minuti \\
			Gallo Edoardo & 40 minuti \\
			Gobbo Marco & 30 minuti \\
			Moretto Giulio & 40 minuti \\
			Simionato Riccardo & 40 minuti \\
			\hline
		\end{tabular}
		\label{tab:conference}
	\end{table}
	
	\section{SINTESI DELL'INCONTRO}
	Durante l'incontro è stata fatta una rapida verifica dei verbali interni mancanti non ancora caricati nella repository.\\
	Sono stati discussi vari metodi di realizzazione del POC, in particolare come approcciarsi alla creazione del prompt da sottoporre all'LLM (tecniche di prompt engineering come ad esempio la tecnica INSTRUCT) e come estrarre dal file strutturato che descrive il database solamente le tabelle relative alla richiesta dell'utente (tecniche di text analysis) per non dover fornire l'intera struttura database all'LLM.\\
	È stata infine definita una struttura minima per il POC, decidendo di tralasciare l'interfaccia grafica web per concentrarsi sullo sviluppo dell'applicativo e testare diversi metodi di creazione del prompt. Il POC dovrà fornire un file JSON con la descrizione di un database, e un file eseguibile che permetta di inserire la frase in linguaggio naturale da terminale e la inserisca all'interno di un prompt con la descrizione del database.
	
	\section{DECISIONI}
	\begin{itemize}
		\item Creare il file JSON con la descrizione di un database da poter usare per il POC;
		\item Approfondire individualmente le tecniche di prompt engineering e text analysis e appuntare le risorse utili sull'apposito file condiviso.
	\end{itemize}
	
	\section{RISORSE}
	Prompt engineering
	\begin{itemize}
		\item Concetti sul prompt engineering\\ \url{https://thenewstack.io/prompt-engineering-get-llms-to-generate-the-content-you-want/}
		\item Tecnica INSTRUCT\\ \url{https://medium.com/@ickman/instruct-making-llms-do-anything-you-want-ff4259d4b91}
	\end{itemize}
	
	Text analysis
	\begin{itemize}
		\item Estrarre dati da un testo non strutturato\\ \url{https://xebia.com/blog/archetype-llm-batch-use-case/}
		\item Usare gli LLM per l'analisi dei testi\\ \url{https://github.com/cssmodels/howtousellms}
		\item Esempi HuggingFace\\ \url{https://huggingface.co/dslim/bert-base-NER}\\ \url{https://huggingface.co/knowledgator/UTC-DeBERTa-large}
	\end{itemize}
	
\end{document}