\documentclass[5pt]{article}

\usepackage{sectsty}
\usepackage{graphicx}
\usepackage{lipsum} % for generating dummy text
\usepackage[margin=1in]{geometry}
\usepackage{setspace}
\usepackage{array}
\usepackage{cellspace}
\usepackage{tabularx}


\usepackage{hyperref}
\usepackage{scrextend}
\graphicspath{ {../../assets} }


% Margins
\topmargin=-0.45in
\evensidemargin=0in
\oddsidemargin=0in
\textwidth=6.5in
\textheight=9.0in
\headsep=0.25in

\title{ Verbale - data }
\date{\today}

%STARTOF THE DOCUMENT
\begin{document}

%-------------------------

% Reduce top margin only on the first page
\newgeometry{top=0.5in}

%UNIPD LOGO
    \vspace{8pt}
    \includegraphics[scale=0.2]{UNIPDFull.png}
%END UNIPD LOGO

\vspace{30pt}

%COURSE INFO
\begin{minipage}[t]{0.48\textwidth}
    %COURSE TITLE
        \begin{flushleft}
            Informatica\\
            \vspace{5pt}
            \textbf{\LARGE Ingegneria del Software}\\
            Anno Accademico: 2023/2024
        \end{flushleft}
    %END COURSE TITLE
\end{minipage}
%END COURSE INFO


\vspace{5px}


%BLACK LINE
\textcolor{}{\rule{\textwidth}{5pt}}

%JACKPOT CODING INFO
\begin{minipage}[t]{0.50\textwidth}
    %LOGO JACKPOT CODING
    \begin{flushleft}
        \hspace{10pt}
        \includegraphics[scale=0.65]{jackpot-logo.png} 
    \end{flushleft}
\end{minipage}
\hspace{-60pt} % This adds horizontal space between the minipages
\begin{flushright}
    \begin{minipage}[t]{0.50\textwidth}
        %INFO JACKPOT CODING
        \begin{flushright}
            Gruppo: {\Large Jackpot Coding}\\
            Email: \href{mailto:jackpotcoding@gmail.com}{jackpotcoding@gmail.com}
        \end{flushright}
    \end{minipage}
\end{flushright}
%END JACKPOT CODING INFO

\vspace{24pt}

%TITLE
\begin{center}
    \textbf{\large VERBALE }
    \textbf{\large 24/10/2023} \\
    \textbf{\Large IMPOSTAZIONE ORGANIZZAZIONE}
\end{center}
%END TITLE

\vspace{13pt}

\begin{flushleft}
    \begin{spacing}{1.5}
        REDATTORE: R. Simionato\\%INSERT HERE THE NAMES
        VERIFICATORI: E. Gallo, M. Camillo \\
        RESPONSABILE: G. Moretto\\%INSERT HERE THE NAMES
        \vspace{7pt}
        SCRIBA: R. Simionato\\%INSERT HERE THE NAMES
    \end{spacing}
\end{flushleft}

\begin{flushright}
    \begin{spacing}{1}
        USO: INTERNO\\
        VERSIONE: 1.1\\
    \end{spacing}
\end{flushright}


% Restore original margins from the second page onwards
\restoregeometry

\pagebreak




% Optional TOC
% \tableofcontents
% \pagebreak

%--Paper--

\section{\Large REGISTRO DELLE MODIFICHE}
\begin{table}[ht]
  \centering
  \renewcommand{\arraystretch}{1.5} % Adjust the value as needed
  \begin{tabular}{|>{\centering}p{40pt}|>{\centering}p{60pt}|>{\centering}p{85pt}|>{\centering}p{85pt}|>{\centering}p{150pt}|}
    \hline \textbf{Versione} & \textbf{Data} &
    \textbf{Autore} & \textbf{Verificatore} & \textbf{Modifica} \tabularnewline
    \hline v 1.1 & 14/11/2023 & E. Gallo & G.Moretto & Aggiunta sezione "Decisioni" \tabularnewline
    \hline v 1.0 & 25/10/2023 & R. Simionato & M. Camillo & Redatto documento \tabularnewline
    \hline
  \end{tabular}
  \label{tab:conference}
\end{table}

\section{\Large ORARIO}
\begin{spacing}{1.5}
    {\large Inizio incontro: 20:00}\\
    {\large Fine incontro: 21:30}
\end{spacing}

\section{PARTECIPANTI}
% Define minimal spacing at the top and bottom of cells
\setlength\cellspacetoplimit{6pt}
\setlength\cellspacebottomlimit{6pt}

\begin{table}[ht]
  \begin{tabular}{|Sc|Sc|}
    \hline
    \textbf{Nome Componente} & \textbf{Durata della presenza} \\
    \hline
    Camillo Matteo & 1 ora 30 minuti \\
    Favaretto Marco & 1 ora 30 minuti \\
    Gallo Edoardo & 1 ora 30 minuti \\
    Gobbo Marco & 1 ora 30 minuti \\
    Moretto Giulio & 1 ora 30 minuti \\
    Simionato Riccardo & 1 ora 30 minuti \\
    \hline
  \end{tabular}
  \label{tab:conference}
\end{table}

\section{SINTESI DELL'INCONTRO}

È stato creato l’indirizzo email del gruppo \href{mailto:jackpotcoding@gmail.com}{jackpotcoding@gmail.com} e creata l'\href{https://github.com/Jackpot-Coding}{Organizzazione su Github}.\\
Sono state esposte ed elaborate le preferenze di ciascuno sulla scelta del capitolato. \\

\medskip
\noindent Il gruppo si è trovato in accordo nel portare come prima scelta il capitolato C9 - ChatSQL presentato dall’azienda Zucchetti, seguito da C3 - EasyMeal e C8 - JMAP.\\

\medskip
\noindent Sono state scritte ed inviate mail per fissare gli incontri conoscitivi con i referenti di Zucchetti e Imola Informatica.\\

\medskip
\noindent È stato finalizzato il logo.

\section{DECISIONI}
\begin{itemize}
    \item Trovare un accordo con i proponenti riguardo data e ora dell’incontro e il mezzo tramite il quale comunicare.
    \item Preparare una lista di domande e dubbi riguardanti i vari capitolati che verranno approfonditi. 
\end{itemize}
%--/Paper--

\end{document}