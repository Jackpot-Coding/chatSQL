\documentclass[5pt]{article}

\usepackage{sectsty}
\usepackage{graphicx}
\usepackage{lipsum} % for generating dummy text
\usepackage[margin=1in]{geometry}
\usepackage{setspace}
\usepackage{array}
\usepackage{cellspace}


\usepackage{hyperref}
\usepackage{scrextend}
\graphicspath{ {../../assets} }


% Margins
\topmargin=-0.45in
\evensidemargin=0in
\oddsidemargin=0in
\textwidth=6.5in
\textheight=9.0in
\headsep=0.25in

\title{ Verbale - data }
\date{\today}

%STARTOF THE DOCUMENT
\begin{document}

%-------------------------

% Reduce top margin only on the first page
\newgeometry{top=0.5in}

%UNIPD LOGO
    \vspace{8pt}
    \includegraphics[scale=0.2]{UNIPDFull.png}
%END UNIPD LOGO

\vspace{30pt}

%COURSE INFO
\begin{minipage}[t]{0.48\textwidth}
    %COURSE TITLE
        \begin{flushleft}
            Informatica\\
            \vspace{5pt}
            \textbf{\LARGE Ingegneria del Software}\\
            Anno Accademico: 2023/2024
        \end{flushleft}
    %END COURSE TITLE
\end{minipage}
%END COURSE INFO


\vspace{5px}


%BLACK LINE
\rule{\textwidth}{5pt}

%JACKPOT CODING INFO
\begin{minipage}[t]{0.50\textwidth}
    %LOGO JACKPOT CODING
    \begin{flushleft}
        \hspace{10pt}
        \includegraphics[scale=0.65]{jackpot-logo.png} 
    \end{flushleft}
\end{minipage}
\hspace{-60pt} % This adds horizontal space between the minipages
\begin{flushright}
    \begin{minipage}[t]{0.50\textwidth}
        %INFO JACKPOT CODING
        \begin{flushright}
            Gruppo: {\Large Jackpot Coding}\\
            Email: \href{mailto:jackpotcoding@gmail.com}{jackpotcoding@gmail.com}
        \end{flushright}
    \end{minipage}
\end{flushright}
%END JACKPOT CODING INFO

\vspace{24pt}

%TITLE
\begin{center}
    \textbf{\large VERBALE }
    \textbf{\large 15/04/2024} \\
\end{center}
%END TITLE

\vspace{13pt}

\begin{flushleft}
    \begin{spacing}{1.5}
        REDATTORE:  R. Simionato\\%INSERT HERE THE NAMES
        VERIFICATORE:   G. Moretto\\%INSERT HERE THE NAMES
        RESPONSABILE:  M. Camillo\\%INSERT HERE THE NAMES
        \vspace{7pt}
        SCRIBA: R. Simionato\\%INSERT HERE THE NAMES
        \vspace{7pt}
    \end{spacing}
\end{flushleft}

\begin{flushright}
    \begin{spacing}{1}
        USO: INTERNO\\
        VERSIONE: 1.0\\
    \end{spacing}
\end{flushright}


% Restore original margins from the second page onwards
\restoregeometry

\pagebreak

% Optional TOC
% \tableofcontents
% \pagebreak

%--Paper--

\section{\Large ORARIO}
\begin{spacing}{1.5}
    {\large Inizio incontro: 19.00}\\
    {\large Fine incontro: 19.45}
\end{spacing}

\section{PARTECIPANTI}
% Define minimal spacing at the top and bottom of cells
\setlength\cellspacetoplimit{6pt}
\setlength\cellspacebottomlimit{6pt}

\begin{table}[ht]
  \begin{tabular}{|Sc|Sc|}
    \hline
    \textbf{Nome Componente} & \textbf{Durata della presenza} \\
    \hline
    Camillo Matteo & 45 minuti \\
    Favaretto Marco & 45 minuti \\
    Gallo Edoardo & 45 minuti \\
    Gobbo Marco & 45 minuti \\
    Moretto Giulio & 45 minuti \\
    Simionato Riccardo & 30 minuti \\
    \hline
  \end{tabular}
  \label{tab:conference}
\end{table}

\section{SINTESI DELL'INCONTRO}
Durante l'incontro sono state suddivise e commentate le task relative allo sviluppo del progetto per lo sprint.\\
Si è in seguito discusso dei dubbi da porre al Professor Cardin durante l'incontro che si terrà mercoledì 17/04.

\section{DECISIONI}
\begin{itemize}
    \item Decisi gli argomenti e i dubbi da portare all'incontro di mercoledì:
    \begin{itemize}
    	\item Ottenere un parere sulla nostra scelta di utilizzare Django, un framework MVT(Model-View-Template);
    	\item È corretto inserire all'interno dei diagrammi UML le componenti del framework o i diagrammi possono essere più generici;
    	\item Dove è più opportuno collocare le classi FileUploader e PromptCreator, se all'interno di Model o View, e come collegarle agli altri elementi.
    \end{itemize}
    \item Assegnata a ciascun membro una parte dello sviluppo dell'applicazione per soddisfare la maggior parte dei requisiti entro la fine dello sprint.
\end{itemize}
%--/Paper--

\end{document}
