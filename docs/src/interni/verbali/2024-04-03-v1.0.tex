\documentclass[5pt]{article}

\usepackage{sectsty}
\usepackage{graphicx}
\usepackage{lipsum} % for generating dummy text
\usepackage[margin=1in]{geometry}
\usepackage{setspace}
\usepackage{array}
\usepackage{cellspace}



\usepackage{hyperref}
\usepackage{scrextend}
\graphicspath{ {../../assets} }


% Margins
\topmargin=-0.45in
\evensidemargin=0in
\oddsidemargin=0in
\textwidth=6.5in
\textheight=9.0in
\headsep=0.25in

\title{ Verbale - data }
\date{\today}

%STARTOF THE DOCUMENT
\begin{document}

%-------------------------

% Reduce top margin only on the first page
\newgeometry{top=0.5in}

%UNIPD LOGO
    \vspace{8pt}
    \includegraphics[scale=0.2]{UNIPDFull.png}
%END UNIPD LOGO

\vspace{25pt}

%COURSE INFO
\begin{minipage}[t]{0.48\textwidth}
    %COURSE TITLE
        \begin{flushleft}
            Informatica\\
            \vspace{5pt}
            \textbf{\LARGE Ingegneria del Software}\\
            Anno Accademico: 2023/2024
        \end{flushleft}
    %END COURSE TITLE
\end{minipage}
%END COURSE INFO


\vspace{5px}


%BLACK LINE
\hrule
%JACKPOT CODING INFO
\begin{minipage}[t]{0.50\textwidth}
    %LOGO JACKPOT CODING
    \begin{flushleft}
        \hspace{10pt}
        \includegraphics[scale=0.65]{jackpot-logo.png} 
    \end{flushleft}
\end{minipage}
\hspace{-60pt} % This adds horizontal space between the minipages
\begin{flushright}
    \begin{minipage}[t]{0.50\textwidth}
        %INFO JACKPOT CODING
        \begin{flushright}
            Gruppo: {\Large Jackpot Coding}\\
            Email: \href{mailto:jackpotcoding@gmail.com}{jackpotcoding@gmail.com}
        \end{flushright}
    \end{minipage}
\end{flushright}
%END JACKPOT CODING INFO

\vspace{15pt}

%TITLE
\begin{center}
    \textbf{\large VERBALE }
    \textbf{\large 18/03/2024} \\
    \textbf{\Large}
\end{center}
%END TITLE

\vspace{13pt}

\begin{flushleft}
    \begin{spacing}{1.5}
        REDATTORE:  G. Moretto\\
        VERIFICATORE:  \\%INSERT HERE THE NAMES
        RESPONSABILE: E. Gallo\\%INSERT HERE THE NAMES
        \vspace{7pt}
        SCRIBA: M. Favaretto\\%INSERT HERE THE NAMES
    \end{spacing}
\end{flushleft}

\begin{flushright}
    \begin{spacing}{1}
        USO: INTERNO\\
        VERSIONE: 1.0\\ % secondo numerazione proposta da Riccardo
    \end{spacing}
\end{flushright}


% Restore original margins from the second page onwards
\restoregeometry

\pagebreak

% Optional TOC
% \tableofcontents
% \pagebreak

\section{\Large ORARIO}
\begin{spacing}{1.5}
    {\large Inizio incontro: 15:35}\\
    {\large Fine incontro: 16:40}
\end{spacing}

\section{PARTECIPANTI}
% Define minimal spacing at the top and bottom of cells
\setlength\cellspacetoplimit{6pt}
\setlength\cellspacebottomlimit{6pt}

\begin{table}[ht]
  \begin{tabular}{|Sc|Sc|}
    \hline
    \textbf{Nome Componente} & \textbf{Durata della presenza} \\
    \hline
    Camillo Matteo & 1 ora 5 minuti \\
    Favaretto Marco &1 ora 5 minuti\\
    Gallo Edoardo & 1 ora 5 minuti  \\
    Gobbo Marco & 1 ora 5 minuti \\
    Moretto Giulio & 1 ora 5 minuti \\
    Simionato Riccardo & 1 ora \\
    \hline
  \end{tabular}
  \label{tab:conference}
\end{table}

\section{SINTESI DELL'INCONTRO}

Si è discussa la necessità di aggiornare in maniera costante lo stato dei compiti assegnati nello strumento di\textit{ Issue Tracking System} (Jira). Inoltre ci si è raccomandati di rendicontare le ore produttive di ogni componente. \\
Si sono discussi alcuni pattern architetturali utilizzabili nel progetto, e mostrata un applicazione di esempio create nel framework Django che illustra il pattern \textit{Model View Template}. \\
Si è inoltre fissato il prossimo incontro per Lunedì 15.

\section{DECISIONI}
\begin{itemize}
	\item Utilizzo dei pattern \textit{Model View Template} e \textit{Strategy} per il progetto;
	\item Divisione dei compiti per la redazione del documento di Specifica Architetturale:
		\begin{itemize}
			\item Introduzione - M. Camillo;
			\item Tecnologie - G. Moretto;
			\item Introduzione dell'architettura - M. Gobbo;
			\item Diagrammi delle classi - M. Camillo e R. Simionato;
			\item Design pattern usati - M. Favaretto;
			\item Riferimenti normativi - E. Gallo;
			\item Sincronizzazione con il Glossario E. Gallo;
			\item Aggiunta elenco figure e tabelle - G. Moretto;
		\end{itemize}
	\item Il gruppo deve diventare familiare con il \textit{framework} Django utilizzando la docuentazione del framework stesso;
	\item La prima versione del documento di Specifica Architetturale deve essere terminata entro il giorno 08/04/2024;
	\item In caso sorgano dubbi sulla specifica architetturale si contatterà il Prof. Cardin per un incontro di chiarimento;
	\item Il documento sopra menzionato deve essere discusso con il proponente appena pronto;
\end{itemize}

\end{document}
