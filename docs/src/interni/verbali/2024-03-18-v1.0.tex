\documentclass[5pt]{article}

\usepackage{sectsty}
\usepackage{graphicx}
\usepackage{lipsum} % for generating dummy text
\usepackage[margin=1in]{geometry}
\usepackage{setspace}
\usepackage{array}
\usepackage{cellspace}



\usepackage{hyperref}
\usepackage{scrextend}
\graphicspath{ {../../assets} }


% Margins
\topmargin=-0.45in
\evensidemargin=0in
\oddsidemargin=0in
\textwidth=6.5in
\textheight=9.0in
\headsep=0.25in

\title{ Verbale - data }
\date{\today}

%STARTOF THE DOCUMENT
\begin{document}

%-------------------------

% Reduce top margin only on the first page
\newgeometry{top=0.5in}

%UNIPD LOGO
    \vspace{8pt}
    \includegraphics[scale=0.2]{UNIPDFull.png}
%END UNIPD LOGO

\vspace{25pt}

%COURSE INFO
\begin{minipage}[t]{0.48\textwidth}
    %COURSE TITLE
        \begin{flushleft}
            Informatica\\
            \vspace{5pt}
            \textbf{\LARGE Ingegneria del Software}\\
            Anno Accademico: 2023/2024
        \end{flushleft}
    %END COURSE TITLE
\end{minipage}
%END COURSE INFO


\vspace{5px}


%BLACK LINE
\hrule
%JACKPOT CODING INFO
\begin{minipage}[t]{0.50\textwidth}
    %LOGO JACKPOT CODING
    \begin{flushleft}
        \hspace{10pt}
        \includegraphics[scale=0.65]{jackpot-logo.png} 
    \end{flushleft}
\end{minipage}
\hspace{-60pt} % This adds horizontal space between the minipages
\begin{flushright}
    \begin{minipage}[t]{0.50\textwidth}
        %INFO JACKPOT CODING
        \begin{flushright}
            Gruppo: {\Large Jackpot Coding}\\
            Email: \href{mailto:jackpotcoding@gmail.com}{jackpotcoding@gmail.com}
        \end{flushright}
    \end{minipage}
\end{flushright}
%END JACKPOT CODING INFO

\vspace{15pt}

%TITLE
\begin{center}
    \textbf{\large VERBALE }
    \textbf{\large 18/03/2024} \\
    \textbf{\Large}
\end{center}
%END TITLE

\vspace{13pt}

\begin{flushleft}
    \begin{spacing}{1.5}
        REDATTORE:  E. Gallo \\
        VERIFICATORE: G. Moretto \\%INSERT HERE THE NAMES
        RESPONSABILE: E. Gallo\\%INSERT HERE THE NAMES
        \vspace{7pt}
        SCRIBA: E. Gallo\\%INSERT HERE THE NAMES
    \end{spacing}
\end{flushleft}

\begin{flushright}
    \begin{spacing}{1}
        USO: INTERNO\\
        VERSIONE: 1.0\\ % secondo numerazione proposta da Riccardo
    \end{spacing}
\end{flushright}


% Restore original margins from the second page onwards
\restoregeometry

\pagebreak

% Optional TOC
% \tableofcontents
% \pagebreak

\section{\Large ORARIO}
\begin{spacing}{1.5}
    {\large Inizio incontro: 21:00}\\
    {\large Fine incontro: 23:00}
\end{spacing}

\section{PARTECIPANTI}
% Define minimal spacing at the top and bottom of cells
\setlength\cellspacetoplimit{6pt}
\setlength\cellspacebottomlimit{6pt}

\begin{table}[ht]
  \begin{tabular}{|Sc|Sc|}
    \hline
    \textbf{Nome Componente} & \textbf{Durata della presenza} \\
    \hline
    Camillo Matteo & - \\
    Favaretto Marco & 2 ore \\
    Gallo Edoardo & 2 ore \\
    Gobbo Marco & -\\
    Moretto Giulio & 2 ore \\
    Simionato Riccardo & 2 ore \\
    \hline
  \end{tabular}
  \label{tab:conference}
\end{table}

\section{SINTESI DELL'INCONTRO}
Nell'incontro sono stati presi in rassegna tutti i documenti ancora incompleti o non ancora inseriti nel ramo principale della repository \textit{GitHub}, modificandoli e verificandoli. In particolare, sono stati completati e inseriti i seguenti documenti:
\begin{itemize}
	\item Verbale interno del 11/03/2024
	\item Norme di Progetto v1.0.2
	\item Analisi dei Requisiti v1.1.0
	\item Glossario v1.0.0
	\item Piano di Progetto v1.0.0
\end{itemize}

\section{DECISIONI}
\begin{itemize}
	\item Ultimare e verificare il Piano di Qualifica
	\item Costruire la presentazione PowerPoint per la seconda parte dell'RTB
	\item Prenotare tramite mail la seconda parte dell'RTB
\end{itemize}

\end{document}
