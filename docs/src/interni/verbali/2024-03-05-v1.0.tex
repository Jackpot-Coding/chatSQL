\documentclass[5pt]{article}

\usepackage{sectsty}
\usepackage{graphicx}
\usepackage{lipsum} % for generating dummy text
\usepackage[margin=1in]{geometry}
\usepackage{setspace}
\usepackage{array}
\usepackage{cellspace}



\usepackage{hyperref}
\usepackage{scrextend}
\graphicspath{ {../../assets} }


% Margins
\topmargin=-0.45in
\evensidemargin=0in
\oddsidemargin=0in
\textwidth=6.5in
\textheight=9.0in
\headsep=0.25in

\title{ Verbale - data }
\date{\today}

%STARTOF THE DOCUMENT
\begin{document}

%-------------------------

% Reduce top margin only on the first page
\newgeometry{top=0.5in}

%UNIPD LOGO
    \vspace{8pt}
    \includegraphics[scale=0.2]{UNIPDFull.png}
%END UNIPD LOGO

\vspace{25pt}

%COURSE INFO
\begin{minipage}[t]{0.48\textwidth}
    %COURSE TITLE
        \begin{flushleft}
            Informatica\\
            \vspace{5pt}
            \textbf{\LARGE Ingegneria del Software}\\
            Anno Accademico: 2023/2024
        \end{flushleft}
    %END COURSE TITLE
\end{minipage}
%END COURSE INFO


\vspace{5px}


%BLACK LINE
\hrule
%JACKPOT CODING INFO
\begin{minipage}[t]{0.50\textwidth}
    %LOGO JACKPOT CODING
    \begin{flushleft}
        \hspace{10pt}
        \includegraphics[scale=0.65]{jackpot-logo.png} 
    \end{flushleft}
\end{minipage}
\hspace{-60pt} % This adds horizontal space between the minipages
\begin{flushright}
    \begin{minipage}[t]{0.50\textwidth}
        %INFO JACKPOT CODING
        \begin{flushright}
            Gruppo: {\Large Jackpot Coding}\\
            Email: \href{mailto:jackpotcoding@gmail.com}{jackpotcoding@gmail.com}
        \end{flushright}
    \end{minipage}
\end{flushright}
%END JACKPOT CODING INFO

\vspace{15pt}

%TITLE
\begin{center}
    \textbf{\large VERBALE }
    \textbf{\large 05/03/2024} \\
    \textbf{\Large}
\end{center}
%END TITLE

\vspace{13pt}

\begin{flushleft}
    \begin{spacing}{1.5}
        REDATTORE:  E. Gallo \\
        VERIFICATORE:  \\%INSERT HERE THE NAMES
        RESPONSABILE: M. Camillo \\%INSERT HERE THE NAMES
        \vspace{7pt}
        SCRIBA: E. Gallo\\%INSERT HERE THE NAMES
    \end{spacing}
\end{flushleft}

\begin{flushright}
    \begin{spacing}{1}
        USO: INTERNO\\
        VERSIONE: 1.0\\ % secondo numerazione proposta da Riccardo
    \end{spacing}
\end{flushright}


% Restore original margins from the second page onwards
\restoregeometry

\pagebreak

% Optional TOC
% \tableofcontents
% \pagebreak

\section{\Large ORARIO}
\begin{spacing}{1.5}
    {\large Inizio incontro: 14:30}\\
    {\large Fine incontro: 16:00}
\end{spacing}

\section{PARTECIPANTI}
% Define minimal spacing at the top and bottom of cells
\setlength\cellspacetoplimit{6pt}
\setlength\cellspacebottomlimit{6pt}

\begin{table}[ht]
  \begin{tabular}{|Sc|Sc|}
    \hline
    \textbf{Nome Componente} & \textbf{Durata della presenza} \\
    \hline
    Camillo Matteo & 1 ora \\
    Favaretto Marco & 1 ora e 30 minuti \\
    Gallo Edoardo & 1 ora e 30 minuti \\
    Gobbo Marco & - \\
    Moretto Giulio & 1 ora e 30 minuti \\
    Simionato Riccardo & - \\
    \hline
  \end{tabular}
  \label{tab:conference}
\end{table}

\section{SINTESI DELL'INCONTRO}
La prima parte dell'incontro è servita al gruppo per comprendere e condividere le conoscenze apprese sul funzionamento di \textit{Jira}. È stato completato il primo sprint e deciso quali task portare avanti durante il secondo periodo.\\
\\
Si è poi parlato della divisione delle ore e discusso sul preventivo stilato in precedenza, dividendolo nei periodi decisi, per renderlo più adatto all'inserimento nei documenti di progetto.

\section{DECISIONI}
In preparazione alla seconda parte dell'RTB con il prof. Vardanega sono state individuate le seguenti task:
\begin{itemize}
	\item Rifinire il Piano di Progetto con le decisioni prese durante la riunione
	\item Rifinire il Piano di Qualifica
	\item Aggiornare le Norme di Progetto per rispecchiare le modifiche implementate nel \textit{Way of Working}
\end{itemize}

\end{document}
