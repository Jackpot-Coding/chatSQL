\documentclass{article}
\usepackage{graphicx} % Required for inserting images

\usepackage{hyperref}
\hypersetup{
    colorlinks,
    citecolor=black,
    filecolor=black,
    linkcolor=black,
    urlcolor=black
}

\title{Norme di Progetto}
\author{Jackpot Coding}
\renewcommand*\contentsname{Indice}

\begin{document}

\maketitle

\pagebreak

\begin{table}[ht]
\centerline{%
  \begin{tabular}{|c|c|c|c|c|}
    \hline
    \textbf{Versione} & \textbf{Data} & \textbf{Autore} & \textbf{Verificatore} & \textbf{Modifica} \\
    \hline
    \hline
    V0.0.7 & 04/01/2024 & M. Gobbo & - & \shortstack{Stesura iniziale parte 1.4, 3.2, 3.3, 3.4} \\
    \hline
    % versione & data & autore & verificatore & descrizioneModifica \\
    % \hline
    V0.0.6 & 28/11/2023 & E. Gallo & - & \shortstack{Stesura delle sezioni 1.1, 1.2, 1.3 \\ dell'Introduzione} \\
    \hline
    V0.0.5 & 27/11/2023 & G.Moretto & - & \shortstack{Aggiornamento sezioni da 3.1.8 a 3.1.12 } \\
    \hline
    V0.0.4 & 19/11/2023 & G.Moretto & - & \shortstack{Aggiunti link all'indice \\ Aggiornamento sezioni 3.1.4, 3.1.5,\\ 3.1.6 , 3.1.7} \\
    \hline
    V0.0.3 & 17/11/2023 & R.Simionato & G.Moretto & \shortstack{Aggiunta sezione 4.2 Infrastruttura \\ e stesura} \\
    \hline
    V0.0.2 & 16/11/2023 & G. Moretto & - & \shortstack{Aggiunta sezione 3.1 documentazione \\ e stesura preliminare} \\
    \hline
    V0.0.1 & 15/11/2023 & G. Moretto & R. Simionato & Creata struttura del documento \\
    \hline
  \end{tabular}%
  }
  \label{tab:conference}
\end{table}



\pagebreak
\tableofcontents
\pagebreak

\section{Introduzione}

\subsection{Scopo del documento}
Il presente documento ha lo scopo di individuare le best practices di progetto e definire correttamente il Way of Working (G) dell'attività produttiva, in modo tale da permettere un lavoro omogeneo, coeso e attuabile in modo asincrono. Per garantire un approccio incrementale, vengono indicate nel registro delle modifiche, tutte le versioni e lo storico del documento, seguendo le regole riportate nel documento stesso. Nei prossimi punti vengono precisamente descritte le convenzioni sull'uso degli strumenti utilizzati, sia di comunicazione sia di realizzazione dei processi di lavoro interno ed esterno.

\subsection{Scopo del prodotto}
L'obiettivo del progetto è quello di realizzare un'applicazione che esplori a fondo la capacità dei Large Language Models (LLM) di generare codice più o meno corretto. In questo particolare caso, il capitolato (G) cerca di approfondire l'uso degli LLM e la tecnica di prompting (G) per generare codice SQL (G).
L'applicativo deve:
\begin{enumerate}
    \item Archiviare la descrizione della struttura di un database.
    \item Fornire un modo per richiedere, tramite linguaggio naturale, una visione di alcuni dati del database a scelta.
    \item Combinare la richiesta con le varie informazioni necessarie della struttura del database, creando un prompt che verrà poi sottoposto ad un sistema di AI (G).
\end{enumerate}
Sia il linguaggio di programmazione sia l'LLM utilizzato dalla piattaforma, sono a discrezione del gruppo. Come linguaggio di programmazione è stato scelto Python (G), per semplicità e la presenza di numerose librerie.
Oltre ai requisiti obbligatori, l'azienda Zucchetti S.p.A., da la possibilità di sviluppare l'applicazione in maniere più approfondita. I requisiti opzionali sono:
\begin{enumerate}
    \item Visualizzare la frase SQL prodotta dal sistema AI.
    \item Sviluppare l'input vocale della frase in linguaggio naturale.
    \item Verificare la correttezza della frase SQL prodotta dal sistema di AI.
    \item Implementare la gestione di più basi di dati.
    \item Utilizzare modelli LLM alternativi a ChatGPT.
    \item Addestrare un modello in modo specifico per la traduzione di frasi di interrogazione in italiano a frasi SQL.
\end{enumerate}

\subsection{Glossario}
Per evitare incomprensioni relative ai vari termini utilizzati nel documento, viene fornito un Glossario, nell'omonimo file, in modo da dare una definizione precisa di ogni termine che possa risultare ambiguo. Ogni termine presente nel Glossario verrà indicato nel documento con una lettera G come apice.

\subsection{Riferimenti}
\subsubsection{Riferimenti normativi}
Capitolato C9 ChatSQL: creare frasi SQL da linguaggio naturale
\begin{itemize}
    \item \url{https://www.math.unipd.it/~tullio/IS-1/2023/Progetto/C9.pdf}
\end{itemize}

\subsubsection{Riferimenti informativi}
Slide dell’insegnamento del corso di Ingegneria del Software:
\begin{itemize}
\item \url{https://www.math.unipd.it/~tullio/IS-1/2023/Progetto/T3.pdf}
\item \url{https://www.math.unipd.it/~tullio/IS-1/2023/Dispense/T4.pdf}
\item \url{https://www.math.unipd.it/~tullio/IS-1/2023/Dispense/T5.pdf}
\end{itemize}
\section{Processi primari}


\section{Processi di supporto}

\subsection{Documentazione}
\subsubsection{Descrizione, Scopo e Aspettative }
Si descrive come documentazione software come illustrazioni e/o testo che accompagnano il progetto software, sia come documenti separati che come commenti nel codice sorgente.\\ \\
Il suo scopo è quello di rendere comprensibile il progetto nella sua interezza per chi lo sviluppa, chi lo mantiene ed il suo utilizzatore finale. Questo viene fatto descrivendo le attività effettuate nel ciclo di vita del software nel modo più chiaro possibile.\\ \\
Le aspettative per questo processo sono:
\begin{itemize}
    \item Documentare tramite verbale ogni incontro interno ed esterno
    \item Avere una struttura fissata per ogni tipo di documento
    \item Utilizzare una procedura per la verifica dei documenti
    \item Facilitare il lavoro autonomo per la gestione dei documenti
\end{itemize}

\subsubsection{Ciclo di vita di un documento} \label{cicloDoc}
Il ciclo di vita di un documento è diviso nelle seguenti fasi:
\begin{itemize}
    \item Pianificazione: La struttura ed il contenuto vengono decisi dai componenti del gruppo, varie sezioni sono divise e assegnate tramite la piattaforma Trello.
    \item Impostazione: Viene implementata la struttura del documento.
    \item Scrittura: Le persone incaricate scrivono il contenuto del documento seguendo la struttura e le indicazioni per il tipo di documento. Il documento viene scritto nel linguaggio Latex e caricato nella repository in un branch dedicato.
    \item Verifica: Le persone incaricate verificano il contenuto e, se necessario, apportano modifiche allo stesso.
    \item Approvazione: Il responsabile di progetto approva il documento dopo la verifica. Una volta approvato il documento viene aggiunto al ramo main della repository.
\end{itemize}

\subsubsection{Modelli dei documenti}
Viene creato un modello (o template) in formato Latex per ogni tipo di documento, in questo vengono definite l'intestazione ed le sezioni del documento.

\subsubsection{Struttura di un documento}

\paragraph{Intestazione}~\\
L'intestazione contiene il logo dell'università di Padova, indicando il corso di laurea, il corso relativo al progetto e l'anno accademico. Inoltre è presente il logo, il nome e l'indirizzo email del gruppo (Jackpot Coding).
Sono prensenti i dati che identificano il documento: data del documento, oggetto, redattori, verificatori, il responsabile, lo scriba ed in fine l'uso (interno o esterno) e se esterno i destinatari del documento.

\paragraph{Cambiamenti}~\\
La sezione cambiamenti contiene una tabella con le seguenti colonne:
\begin{itemize}
    \item Versione: la versione raggiunta
    \item Data: quando sono state effettuate le modifiche
    \item Autore: Chi ha effettuato le modifiche
    \item Verificatore: Chi ha verificato le modifiche
    \item Modifica: Quali modifiche sono state apportate
\end{itemize}

\paragraph{Indice}~\\
L'indice riporta l'elenco delle sezioni e sotto-sezioni del documento riportando a quale pagina si trovano e contenendo un link per la navigazione rapida.

\subsubsection{Struttura di un verbale}
Oltre alle sezioni sopra citate vengono inserite le seguenti sezioni.

\paragraph{Orario}~\\
Contiene l'ora di inizio e fine dell'incontro.

\paragraph{Partecipanti}~\\
Contiene una tabella elencante il nome e cognome dei partecipanti e la durata della loro presenza all'incontro.

\paragraph{Sintesi dell'incontro}~\\
Contiene una versione riassuntiva degli argomenti trattati durante l'incontro.

\paragraph{Decisioni}~\\
Contiene le decisioni prese durante l'incontro, queste si traducono i compiti assegnati ai membri del gruppo.

\subsubsection{Nome dei file}
Il nome del file deve essere esplicativo, indicando il contenuto dello stesso, e deve indicarne la versione.

\subsubsection{Stile del testo}
\textbf{Grassetto}: Applicato ai titoli e a parole di particolare importanza. \\
\texttt{Monospace}: Applicato a frammenti di codice e percorsi di cartelle e file. \\
\underline{Sottolineato}: Applicato a link presenti nel documento.

\subsubsection{Norme tipografiche}
\begin{itemize}
    \item Gli elementi di un elenco finiscono con il carattere punto.
    \item La prima lettera degli elementi di un elenco è sempre maiuscola.
    \item I nomi delle sezioni e sottosezioni iniziano con una lettera maiuscola.
\end{itemize}

\subsubsection{Glossario}
Il glossario è un documento che contiene termini e/o sigle il quale significato è importante per il progetto.

Il documento viene caricato sulla repository per facilitare la consultazione e la stesura da parte del gruppo.

\subsubsection{Sigle}

\subsubsection{Immagini}
Le immagini presenti nei documenti sono caricate nella cartella \texttt{/docs/src/assets} della repository.


\subsubsection{Strumenti utilizzati}
\paragraph{Overleaf}~\\
Viene utilizzato per la scrittura collaborativa di documenti in formato Latex.
\paragraph{Github}~\\
Nella repository ufficiale vengono caricate le versioni dei documenti utilizzando la procedura descritta dal ciclo di vita come da sezione \ref{cicloDoc}.

\subsection{Gestione della configurazione}
    \subsubsection{Descrizione e scopo}
    La gestione della configurazione è un processo cruciale per mantenere la coerenza del software nel tempo, garantendo il corretto funzionamento del sistema nonostante le eventuali modifiche apportate. Problemi nella configurazione potrebbero causare incoerenze o problemi di aderenza alle normative, compromettendo le operazioni aziendali, è quindi importante perseguire questo punto.
    
    \subsubsection{Sistema di versionamento}
    Il versionamento offre la possibilità di registrare tutte le modifiche apportate a un documento. Questo sistema consente di ripristinare una versione precedente del documento e visualizzare chiaramente tutte le modifiche fatte nel corso del tempo, incluso l'autore di ciascuna modifica.\\
    Nel nostro gruppo, abbiamo adottato il seguente codice per identificare le diverse versioni di un documento:
    \begin{center}
    \textbf{[x].[y].[z]}
    \end{center}
    La versione viene indicata con tre numeri separati da un punto, come visto sopra, indicati in questa sezione come x.y.z.
    \begin{itemize}
    \item[\textbf{x)}] Il numero viene aumentato solo quando il documento è considerato pronto.
    \item[\textbf{y)}] Il numero viene aumentato solo quando le modifiche sono verificate.
    \item[\textbf{z)}] Il numero viene aumentato quando il documento subisce una modifica.
    \end{itemize}  
    
    \subsubsection{Repository}
    \paragraph{Tecnologie utilizzate}~\\
    Il nostro team ha scelto di utilizzare GitHub come strumento per gestire la configurazione, facendo affidamento sul sistema di controllo di versione distribuito di Git.\\
    La repository è accessibile pubblicamente tramite il seguente link:\\
    \begin{center}
    \url{https://github.com/Jackpot-Coding}\\\\
    \end{center}
    Per agevolare la consultazione e la navigazione dei documenti, abbiamo creato un sito dedicato tramite GitHub. È possibile accedervi utilizzando il seguente riferimento:\\
    \begin{center}
    \url{https://jackpot-coding.github.io/chatSQL/}\\\\
    \end{center}
    \paragraph{Struttura Repo}~\\
    Nella struttura del nostro repository i file sono sistemati in questo modo:\\
    


\subsection{Gestione della qualità}
    \subsubsection{Descrizione, scopo}
    La gestione della qualità in un progetto è fondamentale per assicurare che i processi e i prodotti soddisfino le richieste del cliente con la massima qualità possibile. È altrettanto importante perseguire un miglioramento continuo attraverso monitoraggi e valutazioni retrospettive. Questo processo coinvolge una serie di attività volte a garantire che i risultati e le performance del progetto siano allineati agli obiettivi e ai requisiti stabiliti. Questa sezione si concentra sull'impegno del gruppo nella gestione della qualità del progetto, cercando di assicurare un'implementazione accurata e coerente di tali processi
    \subsubsection{Piano di qualifica}

\subsection{Verifica}
    \subsubsection{Descrizione, scopo}
    La verifica del software è il processo valutativo che assicura l'accuratezza della fase di sviluppo per costruire il prodotto desiderato. Si esegue durante lo sviluppo per rilevare difetti precocemente e garantire che il software soddisfi i requisiti del cliente. L'obiettivo della verifica è controllare la correttezza e completezza del prodotto, garantendo che sia conforme alle aspettative. Si basa su analisi e test per assicurare l'assenza di errori nel software e nella documentazione. Questa sezione mira a definire l'approccio scelto dal gruppo per attuare il processo di verifica.
    \subsubsection{Analisi statica}
    L'analisi statica consiste nell'esaminare il codice o la documentazione prima dell'esecuzione del software, garantendo il soddisfacimento dei requisiti specificati. Questo metodo si applica non solo al codice ma anche ai documenti, focalizzandosi sugli aspetti statici del sistema software, come le convenzioni del codice o le metriche del software. Include sia test manuali che automatizzati, come l'analisi di coerenza.
    Questo tipo di analisi si divide in due metodi principali: 

    \begin{itemize}
    \item WALKTHROUGH: prevedono una lettura ampia per individuare possibili errori senza conoscere la loro posizione esatta.
    \item INSPECTION: mirano a individuare errori specifici con un approccio mirato.
    \end{itemize}
    In particolare, nell'analisi dei documenti, si utilizzano tecniche simili di Walkthrough e Inspection.\\
    Inizialmente, l'attività di Walkthrough è prevalente rispetto all'Inspection, ma con il progredire del progetto e la ripetizione delle verifiche sulla documentazione, si sviluppa una lista degli errori comuni (Lista di Controllo), consentendo l'adozione più ampia dell'Inspection, che si rivela più efficiente.
    \subsubsection{Analisi dinamica}
    L'analisi dinamica si svolge simultaneamente all'esecuzione del software ed è principalmente costituita dalla fase di test. A differenza dell'analisi statica, coinvolge l'esecuzione effettiva del sistema e dei suoi componenti.\\
    Per garantire che il prodotto funzioni correttamente durante l'esecuzione, il gruppo utilizza una serie di test eseguibili in tempo reale. È essenziale che tali test siano ripetibili, pertanto si identificano strumenti per automatizzarne l'esecuzione.\\
    Questo approccio dinamico è cruciale per rilevare problemi e difetti durante l'esecuzione effettiva del software, fornendo una maggiore sicurezza sul corretto funzionamento del prodotto.
    
\section{Processi organizzativi}

\subsection{Gestione di Processo}
    \subsubsection{Pianificazione}
    \subsubsection{Coordinamento}

\subsection{Infrastruttura}
L'infrastruttura organizzativa comprende tutti gli strumenti utilizzati per la comunicazione, la divisione dei compiti e il coordinamento tra i componenti per svolgere in modo efficace e strutturato i processi di organizzazione

\subsubsection{Strumenti}

\paragraph{Github}~\\
Servizio di versionamento e hosting della repository scelto per mantenere la documentazione e il codice del progetto.

\paragraph{Discord}~\\
Strumento utilizzato per la comunicazione sincrona e asincrona interna al gruppo. All'interno del server sono presenti due tipologie di canali:
\begin{itemize}
    \item\textbf{Canali Testuali}: utilizzati principalmente per lo scambio di risorse e le comunicazioni testuali sincrone e asincrone;
    \item\textbf{Canali Vocali}: utilizzati per la comunicazione vocale sincrona durante le riunioni interne, consentono inoltre di condividere il proprio schermo con gli altri membri presenti.
\end{itemize}

In base al periodo e alle necessità il numero di ciascun tipo di canale può variare, dando la possibilità di creare canali adibiti ad argomenti specifici.

\paragraph{Telegram}~\\
Principale strumento di comunicazione interna testuale asincrona che avviene tramite una chat condivisa. Le comunicazioni sono per lo più informali e volte all'organizzazione interna. La chat permette di mettere in evidenza i messaggi importanti, menzionare altri membri del gruppo e condividere sondaggi e/o file.

\paragraph{Trello}~\\
Strumento utilizzato per la gestione delle issues e la suddivisione dei compiti. Sono presenti cinque liste che rappresentano lo stato di avanzamento di una issue:
\begin{itemize} 
    \item\textbf{Da Fare}: stato iniziale, qui la issue attende fino a quando non viene presa in carico da un membro del gruppo. Verrà successivamente spostata alla lista "In Corso" appena inizierà la sua lavorazione;
    \item\textbf{In Corso}: la issue è in lavorazione. Potrà essere spostata alla lista "In Pausa", oppure alla lista "Da Verificare" se è stata terminata;
    \item\textbf{In Pausa}: la issue non è attualmente in lavorazione, potrà essere spostata solamente alla lista "In Corso";
    \item\textbf{Da Verificare}: è compito dei Verificatori controllare le issue che vengono spostate in questa lista, scegliendo se terminarle spostandole nella apposita lista "Fatto", o rimandarle in lavorazione alla lista "In Corso";
    \item\textbf{Fatto}: stato finale, la issue è stata completata e verificata.
\end{itemize}
Le uniche due liste che possono avere delle issue non assegnate ad un membro sono "Da Fare" e "Fatto".
È compito del \textit{Responsabile di Progetto} verificare che le issue vengano divise correttamente ai diversi membri del gruppo. 
\paragraph{Overleaf}~\\
Editor online per i file LaTeX, principalmente usato per la stesura dei file in maniera condivisa.
\paragraph{Google Drive}~\\
Strumento utilizzato dal gruppo per la condivisione di file non ufficiali contenenti bozze e appunti con Google Docs, tabelle per la divisione interna dei ruoli e tracciamento del tempo con Google Sheets e presentazioni per i Diari di Bordo con Google Slides. 
\paragraph{Zoom}~\\
Strumento utilizzato principalmente per videochiamate esterne con committente e proponenete.
\paragraph{Google Mail}~\\
Utilizzato per le comunicazioni esterne asincrone verso il proponente e il committente con l'indirizzo mail del gruppo jackpotcoding@gmail.com.

\subsection{Formazione}

\end{document}
